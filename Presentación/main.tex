\documentclass[10pt]{beamer}

\usetheme{Feather}

\usepackage[utf8]{inputenc}
\usepackage[spanish,mexico]{babel}
\usepackage[T1]{fontenc}

\theoremstyle{plain}
\newtheorem{proposición}{Proposición}
\newtheorem{teorema}{Teorema}
\theoremstyle{definition}
\newtheorem{definición}{Definición}
\newtheorem{ejemplo}{Ejemplo}

\newcommand{\Var}{\mathrm{Var}}

\title{\textbf{Estabilidad y transiciones de fase en el sistema de Lotka-Volterra generalizado}}
\subtitle{Examen profesional}
\author{Rodrigo Vega Vilchis \\[4pt] \small Asesor: Sergio A. Alcalá Corona}
\institute{\large Facultad de Ciencias \\ Universidad Nacional Autónoma de México}
\date{}

\begin{document}

\begin{frame}
	\vspace*{1.5cm}
	\titlepage
\end{frame}
\begin{frame}
	\frametitle{Contenido}
	\tableofcontents
\end{frame}
\section{Antecedentes}

\begin{frame}
	\frametitle{Antecedentes}
	Sea $\textbf{F}:\mathbb{R}^n\to\mathbb{R}^n$ no lineal, entonces se define el sistema de ecuaciones diferenciales no lineal
	$$\frac{dX(t)}{dt}=\textbf{F}(X(t))$$
	y se busca caracterizar su estabilidad. Usando teoría de perturbaciones alrededor de un punto crítico $X^*(t)$ que cumple $\textbf{F}(X^*(t))=0$ se encuentra:
	\begin{equation}\label{eqn:Linealizado}
		\frac{d\textbf{p}(t)}{dt}=A\textbf{p}(t)
	\end{equation}
	donde $\textbf{p}(t)$ es el conjunto de perturbaciones alrededor de $X^*$ y $a_{ij}=\left. \frac{\partial f_i(X)}{\partial X_j}\right  |_{X^*}$ con $f_i\in\textbf{F}$ y $a_{ij}\in A$.
\end{frame}

\begin{frame}
	\frametitle{Antecedentes}
	Robert May en 1972 realizó esta propuesta de linealización de sistemas no lineales multivariados para poder conocer su estabilidad en términos de la matriz $A$ que él denomina como \textit{community matrix} o \textit{matriz de interacciones.} Las soluciones de un sistema lineal son:
	$$\textbf{p}(t)=\sum_{j=1}^N c_je^{\lambda_j t}\vec{v}_j$$
	Las cuales pueden ser reales o complejas. Por lo tanto el sistema linealizado será estable si todos sus valores propios son negativos.
\end{frame}

\begin{frame}
	\frametitle{Antecedentes}
	May propone una forma de construir sus matrices de interacciones:
	\newline
	\begin{itemize}
		\item La diagonal queda fija en un valor $-d\in\mathbb{R}^-$ 
		\item Los elementos $a_{ij}\in A$ fuera de la diagonal serán muestreados de una distribución normal de varianza finita $\sigma^2$ centrada en cero.
		\item La existencia de cada $a_{ij}\neq 0$ será por medio de una probabilidad $C$, de modo que para $1-C$ se tendrán $a_{ij}=0$.
	\end{itemize}

	Al parámetro $\sigma$ (desviación estándar de la distribución normal), lo denomina como \textit{fuerza de interacción} y al parámetro $C$ como \textit{conectancia.}
\end{frame}

\begin{frame}
	\frametitle{Antecedentes}
	May propone una Ley Circular que ajusta la distribución de valores propios de $A$. El centro de dicho círculo es el valor $-d$ de la diagonal y el radio esta dado por $\sigma\sqrt{NC}$. De aquí se deriva el parámetro de May que ajusta la estabilidad de estos sistemas
	$$\sigma\sqrt{NC}<|-d|$$
	Si esta relación se cumple, entones se puede asegurar con gran probabilidad que el sistema será estable.
\end{frame}

\section{Planteamiento del problema}

\begin{frame}
	\frametitle{Planteamiento del problema}
	Un ejemplo de sistema no lineal que representa una dinámica ecológica semejante a la de Robert May es el siguiente:
	$$\frac{dx_i}{dt}=r_ix_i\left (1-\frac{\sum_{j=1}^N\alpha_{ij}x_j}{K_i}\right )$$
	Conocido como \textit{Sistema de Lokta-Volterra generalizado.} Considera una tasa de crecimiento $r_i$, una capacidad de carga $K_i$ y un conjunto de interacciones $\alpha_{ij}\in\Lambda$.
\end{frame}

\begin{frame}
	\frametitle{Planteamiento del problema}
	Se define la \textit{matriz de incidencias} $\Lambda$ como aquella que posee las interacciones del sistema de poblaciones. Esta matriz se genera de forma similar a las matrices de May:
	\newline
	\begin{itemize}
		\item La diagonal en este caso se fija a un valor positivo $d\in\mathbb{R}^+$.
		\item Cada $\alpha_{ij}\in\Lambda$ proviene de una distribución normal de varianza finita $\sigma^2$ y centrada en cero.
		\item Cada interacción tiene una probabilidad $p$ de existir, y $1-p$ de ser igual a cero.
	\end{itemize}
	Esta matriz se puede construir con base en la matriz de adyacencia de una \textit{red aleatoria} de Erdös–Rényi.
\end{frame}

\begin{frame}
	\frametitle{Planteamiento del problema}
	El sistema de Lotka-Volterra generalizado para $N\gg1$ es complejo de resolver analíticamente, sin embargo, se puede aproximar su solución mediante un proceso de integración numérica, por ejemplo con Runge-Kutta de orden 4. Por consiguiente, se plantean las siguientes preguntas:
	\newline
	\begin{itemize}
		\item ¿Qué resultado genera la integración del sistema?
		\item ¿Existe una matriz equivalente a la \textit{community matrix} de May?
		\item Si existe: ¿Cual es la distribución de sus valores propios?
		\item ¿Cómo se puede medir la estabilidad de los sistemas LV?
	\end{itemize}
\end{frame}

\begin{frame}
	\frametitle{Planteamiento del problema: Hipótesis}
	Para ello se presentan las siguientes hipótesis del planteamiento del problema:
	\newline
	\begin{itemize}
		\item Las series de tiempo divergen o se estabilizan entorno a un punto fijo del sistema.
		\item Los sistemas LV se linealizan a partir de su matriz Jacobiana, la cual se conforma de la combinación del producto entre variables normales y entradas del punto fijo.
		\item La distribución de valores propios es heterogénea debido a la diagonal de la matriz Jacobiana.
		\item La estabilidad depende de los parámetros $\sigma$, $p$ y $N$, a su vez depende de configuraciones de $\Lambda$ que resulten en puntos fijos no negativos.
	\end{itemize}
\end{frame}

\section{Metodología}

\begin{frame}
  \frametitle{Sistema de Lotka-Volterra generalizado}
  Se define el sistema:
  \begin{equation}\label{eqn:LK}
  	\frac{dx_i}{dt}=r_ix_i\left(1-\frac{\sum_{j=1}^N\alpha_{ij}x_j}{K_i}\right)
  \end{equation}
  \begin{itemize}
  	\item donde $r_i$ es la tasa de crecimiento,
  	\item $K_i$ es la capacidad de carga del sistema,
  	\item $x_i$ es la población de una especie y su derivada marca la dinámica de su crecimiento,
  	\item $\alpha_{ij}x_j$ con las especies que interaccionan con $x_i$ con cierto valor $\alpha_{ij}$.
  \end{itemize}  
\end{frame}

\begin{frame}
  \frametitle{Sistema de Lotka-Volterra generalizado}
  Cada ecuación del sistema equivale a una función de $\textbf{F}:\mathbb{R}^n\to\mathbb{R}^n$, es decir, $\dot{x}_i=f_i(X)$. Entonces:
	$$\textbf{F}(X)=\begin{pmatrix}
	  	f_1(X)\\
	  	\vdots\\
	  	f_n(X)
	\end{pmatrix},\qquad\text{donde } X=(x_1(t),...,x_n(t))$$   
	por lo tanto, el sistema puede ser re-escrito de la siguiente forma
	  $$\dot{X}=\textbf{F}(X)$$
\end{frame}

\begin{frame}
  \frametitle{Sistema de Lotka-Volterra generalizado}
  El resultado de Robert May ha sido proponer la matriz Jacobiana como la \textit{community matrix} (ver ec. \ref{eqn:Linealizado}). Por lo tanto es de interés determinar la matriz Jacobiana del sistema y evaluarlo en el punto crítico correspondiente:
  \begin{equation}
  	\mathbb{J}_{\textbf{F}}(X)=\begin{pmatrix}
  		\frac{\partial f_1(X)}{\partial x_1}&\cdots&\frac{\partial f_1(X)}{\partial x_n}\\
  		\vdots&\ddots&\vdots\\
  		\frac{\partial f_{n}(X)}{\partial x_1}&\cdots&\frac{\partial f_n(X)}{\partial x_n}
  	\end{pmatrix}
  \end{equation}  
  Un punto crítico $X^*$ es aquel que cumple $\textbf{F}(X^*)=0$. Linealizar el sistema significa poder conocer y acceder a las propiedades del mismo de forma local alrededor de este $X^*$.
\end{frame}

\begin{frame}
  \frametitle{Matriz de incidencias}
  A continuación se presentan los pasos para construir la \textit{matriz de incidencias} $\Lambda$ que contiene las $\alpha_{ij}$ de las ecuaciones (\ref{eqn:LK}).
  \newline
  \newline
  Se toma una red aleatoria de Erdös–Rényi y se extrae su matriz de adyacencia asociada $\mathcal{E}\in\mathrm{M}_n(\mathbb{R})$ cuyos elementos son de la forma
  $$\mathcal{E}_{ij}=\begin{cases}
  	1,\qquad\text{si }r<p\\
  	0,\qquad\text{si }r\geq p
  \end{cases},\qquad\text{para cada }i\neq j$$  
  donde $r$ es un número aleatorio entre 0 y 1 y $p$ una probabilidad en el mismo rango. $\mathcal{E}$ puede ser simétrica, o puede ser de una red dirigida de modo que $\mathcal{E}_{ij}\neq \mathcal{E}_{ji}$.
\end{frame}
\begin{frame}
  \frametitle{Matriz de incidencias}
  Teniendo la red de Erdös–Rényi (no dirigida o dirigida), se define una matriz $\mathcal{M}$ cuyas entradas estarán muestreadas de una distribución Normal con $\mu=0$ y $\sigma=\{0.1,0.2,...,1.0\}$.
  \newline
  \newline
  Con ello se define la \textbf{matriz de incidencias} asociada a una red de incidencias que guarda las interacciones del sistema (\ref{eqn:LK})
  $$\Lambda=(\mathcal{E}\odot \mathcal{M})+I$$
  Dependiendo de si $\mathcal{E}$ es dirigida o no, $\Lambda$ será estructuralmente simétrica o puramente aleatoria. 	
\end{frame}
\begin{frame}
  \frametitle{Matriz de incidencias}
  ¿Por qué sumar la identidad al producto $A\odot R$? Extendiendo la ecuación \ref{eqn:LK} se tiene
  $$\frac{dx_i}{dt}=r_ix_i\left (1-\frac{\alpha_{ii}x_i+\alpha_{i1}x_1+\cdots+\alpha_{iN}x_N}{K_i}\right )$$
  considerando que $\alpha_{ii}=1$. Se factoriza la ecuación:
  $$\frac{dx_i}{dt}=r_ix_i\left (1-\frac{x_i}{K_i}\right )-r_ix_i\left (\frac{\alpha_{i1}x_1+\cdots+\alpha_{iN}x_N}{K_i}\right )$$
  Independientemente de los signos de $\alpha_{ij}$ con $i\neq j$, el carácter logístico del sistema se guarda en $\alpha_{ii}=1$. La restricción no es tan fuerte, únicamente se pide $\alpha_{ii}>0$.
\end{frame}
\begin{frame}
  \frametitle{Tipos de interacciones}
  Para la red de incidencias estructuralmente simétricas se tienen las siguientes interacciones (nodo $i\leftrightarrow j$)
  \newline
  \begin{itemize}
  	\item[1.] Cooperación $(--)$
  	\item[2.] Competencia $(++)$
  	\item[3.] Presa-Depredador $(-+)$ ó $(+-)$
  \end{itemize}
\end{frame}
\begin{frame}
  \frametitle{Tipos de interacciones}
   Para la red de incidencias puramente aleatoria se agregan a las anteriores las siguientes interacciones (nodo $i\to j$ y $j\to i$)
  \newline
  \begin{itemize}
  	\item[4.] Comensalismo $(+0)$ (una de ellas obtiene un beneficio)
  	\item[5.] Amensalismo $(-0)$ (una de ella sale perjudicada)
  \end{itemize}
  y vendrán determinadas a partir de como se de la combinación $A\odot R$.
\end{frame}
\begin{frame}
	\frametitle{Tipos de interacciones}
	Por lo tanto las interacciones $(--)$, $(++)$, $(-+)$, $(+0)$ y $(0-)$ pesan en la dinámica del sistema
	$$\frac{dx_i}{dt}=r_ix_i\left (1-\frac{x_i}{K_i}\right )-r_ix_i\left (\frac{\alpha_{i1}x_1+\cdots+\alpha_{iN}x_N}{K_i}\right )$$
\end{frame}

\begin{frame}
	\frametitle{Interacciones de cooperación}
		\begin{figure}[h!]
		\centering
		\includegraphics[scale=0.11]{../Texto/Imagenes/Cooperacion de especies}
		\caption{Sistema de Lotka-Volterra con interacciones de cooperación dados por las ecuaciones (\ref{eqn:LK}). Tasas de crecimiento y capacidades de carga: $r_x=K_x=2$ y $r_y=K_y=3$. \textbf{A}) Series de tiempo del sistema para las especies $x(t)$ y $y(t)$ bajo la condición inicial $(1,2)$. (\textbf{B}) Espacio fase del sistema con sus puntos fijos asociados, se muestra solamente un único punto fijo estable.}
		\label{fig:CooperacionEspecies}
	\end{figure}
\end{frame}

\begin{frame}
	\frametitle{Punto Fijo}
	Para hallar puntos fijos se debe resolver $\textbf{F}(X)=0$ o lo que es equivalente bajos las ecuaciones (\ref{eqn:LK}):
	$$1-\frac{\sum_{j=1}^N \alpha_{ij}x_j}{K_i}=0$$
	que es equivalente a encontrar una solución del sistema 
	$$\Lambda X^*=\mathrm{K}$$
	Este ensamble tiene múltiples soluciones con la forma $X^*=\Lambda^{-1}\mathrm{K}$, cada forma irá en función del número de $x_k=0$ que se contemplen para las diversas soluciones
\end{frame}

\begin{frame}
	\frametitle{Punto Fijo}
	La distribución del punto fijo se posicionará alrededor de $\mathrm{K}$. Mientras $\sigma$ y $p$ sean $\ll 1$ respectivamente, la distribución estará muy próxima a $\mathrm{K}$, y mientras vayan aumentando sus valores, dicha distribución se ensanchará.
	\newline
	\newline
	Si por alguna razón $\det(\Lambda)=0$, entonces la matriz es singular y las entradas de $\Lambda^{-1}$ divergen a $\infty$. Si $\det(\Lambda)<1$ entonces
	$$\text{Adj}(\Lambda)\cdot\Lambda=\det(\Lambda)\cdot \mathrm{I}$$
	implicando que Adj$(\Lambda)$ tiene entradas de magnitud semejante con respecto de $\Lambda$. Si $\det(\Lambda)>1$, las entradas de $\Lambda$ son las mismas implicando que los elementos Adj$(\Lambda)$ se van amplificando cada vez más.
	$$\Lambda^{-1}=\frac{1}{\det(\Lambda)}\text{Adj}(\Lambda)$$
\end{frame}

\begin{frame}
  \frametitle{Jacobiano del sistema de Lotka-Volterra}
  \begin{definición}\label{def:MatrizJacobiana}
  	Sea $\mathcal{J}\in\mathrm{M}_N(\mathbb{R})$ donde $N$ es el número de especies del sistema LV generalizado. Se define la matriz \textit{Jacobiana} asociada al sistema (\ref{eqn:Fmatricial}) evaluado en un punto fijo $X^*$ de la siguiente forma
  	\begin{equation}\label{eqn:MatrizJacobiana}
  		\mathcal{J}_{ij}=\begin{cases}
  			-\frac{r_ix_i^*}{K_i}\alpha_{ii},\qquad&\text{para }i=j\\
  			-\frac{r_ix_i^*}{K_i}\alpha_{ij},\qquad&\text{para }i\neq j
  		\end{cases}
  	\end{equation}
  	considerando que $\alpha_{ii}>0$ y particularmente se ha fijado en $\alpha_{ii}=1$ para todo $i\in\{1,...,N\}$. Si se considera una matriz diagonal cuyos elementos son $-\frac{r_ix_i^*}{K_i}$, entonces la Jacobiana se puede reescribir de la siguiente manera
  	\begin{equation}\label{eqn:Jacobiana}
  		\mathcal{J}=-\text{diag}\left (\frac{r_ix_i^*}{K_i}\right )\Lambda
  	\end{equation}
  \end{definición}
\end{frame}

\begin{frame}
	\frametitle{Proposición}
	\begin{proposición}\label{prop:DiagonalI}
		Los elementos de la diagonal de la matriz Jacobiana determinan en gran medida la estabilidad del sistema
	\end{proposición}
	\begin{proof}
		Para demostrar esta proposición se hará uso del \textit{Teorema de Gershgorin} \cite{GershgorinTheorem}. Para ello se asume que la matriz Jacobiana tendrá valores propios complejos. Se define el radio de Gershgorin como
		\begin{equation}\label{proof1:RGershgorin}
			R_i=\sum_{i\neq j}|\mathcal{J}_{ij}|
		\end{equation}
		con ello se define $D(\mathcal{J}_{ii},R_i)\subset\mathbb{C}$ como el disco de Gershgorin centrado en $\mathcal{J}_{ii}$ con radio $R_i$. El teorema establece que cada valor propio de $\mathcal{J}$ estará contenido en alguno de estos discos. 
	\end{proof}
\end{frame}

\begin{frame}
	\frametitle{Proposición}
	\begin{proof}
		Demostrando que si todo $D(\mathcal{J}_{ii},R_i)$ se encuentra contenido en el semiplano negativo de $\mathbb{C}$, entonces todos los valores propios de la Jacobiana serán negativos también y por lo tanto el sistema (\ref{eqn:LK}) será estable en $X^*$. El elemento más importante de esta prueba es considerar el centro de los discos; si todos los $\mathcal{J}_{ii}$ son negativos entonces el centro de todos los discos se encuentra en el semiplano negativo de $\mathbb{C}$, y para garantizar que todos los valores propios sean negativos habría que probar $|\mathcal{J}_{ii}|>R_i$ para todo $i\in\{1,...,N\}$, es decir:
		\begin{equation}\label{proof1:Desigualdad}
			\begin{split}
				\left |-\frac{r_ix_i^*}{K_i}\alpha_{ii}\right |&>\sum_{j\neq i}\left |-\frac{r_ix_i^*}{K_i}\alpha_{ij}\right |\\
				1&>\sum_{j\neq i}|\alpha_{ij}|
			\end{split}
		\end{equation}
		
		
	\end{proof}
	
\end{frame}

\begin{frame}
	\frametitle{Proposición}
	\begin{proof}
		De la ec. (\ref{eqn:MatrizJacobiana}) se puede ver que $\alpha_{ii}>0$ y en particular $\alpha_{ii}=1$, por lo tanto es importante que todo $x_i^*\in X^*$ sea positivo para que el centro de los discos se encuentre en el semiplano negativo de $\mathbb{C}$. En caso contrario habrá al menos un disco con centro en el sempiplano positivo de $\mathbb{C}$ que pueda alojar valores propios positivos que devengan en una dinámica inestable. Es conveniente poder saber cual es el tamaño promedio de los discos de Gershgorin y para ello se define la siguiente variable aleatoria
		\begin{equation}\label{proof1:VarAlW}
			W = 
			\begin{cases}
				0,\qquad \text{Si }1-p\\
				Y,\qquad \text{Si }p>0	
			\end{cases}
		\end{equation}
		donde $Y\sim N(0,\sigma^2)$ es una variable aleatoria normal. Ya que $R_i$ considera suma de valores absolutos.
	\end{proof}
\end{frame}

\begin{frame}
	\frametitle{Proposición}
	\begin{proof}
		Es de interés saber el valor esperado de $|W|$ 
		\begin{equation}\label{proof1:ValorEsperadoW}
			\mathbb{E}[|W|]=(1-p)\cdot 0 + p\cdot \mathbb{E}[|Y|] = p\cdot\mathbb{E}[|Y|] = p\sigma\sqrt{\frac{2}{\pi}}
		\end{equation}
		el término $\sqrt{\frac{2}{\pi}}$ viene de considerar la variable aleatoria $Z\sim N(0,1)$ y la distribución \textit{half-normal} $|Z|$, entonces su valor esperado es $$\mathbb{E}[|Z|]=\frac{1}{\sqrt{2\pi}}\int_{-\infty}^{\infty}|z|e^{-z^2/2}\, dz=\frac{2}{\sqrt{2\pi}}\int_0^\infty ze^{-z^2/2}\, dz=\sqrt{\frac{2}{\pi}}$$
		
	\end{proof}
\end{frame}

\begin{frame}
	\frametitle{Proposición}
	\begin{proof}
		al considerar la variable aleatoria $Y\sim N(0,\sigma^2)$ entonces basta con escalar $Y=\sigma Z$ y en consecuencia su valor esperado será $$\mathbb{E}[|Y|]=\sigma \mathbb{E}[|Z|]=\sigma\sqrt{\frac{2}{\pi}}$$
		Por lo tanto, el valor esperado del radio de Gershgorin es
		$$\mathbb{E}[R_i]=(N-1)p\sigma\sqrt{\frac{2}{\pi}}$$
		Si este radio promedio es menor a 1 se puede garantizar en cierta medida que los discos de Gershgorin estarán contenidos en el semiplano negativo de $\mathbb{C}$. Sin embargo, pueden existir discos de Gershgorin que se alejen de la media ¿qué tanto se pueden alejar para que se siga cumpliendo $\mathcal{J}_{ii}>R_i$? 
	\end{proof}
\end{frame}

\begin{frame}
	\frametitle{Proposición}
	\begin{proof}
		Será conveniente calcular la varianza de los radios de Gershgorin para averiguarlo. Tomando la variable aleatoria $W$, se calcula su segundo momento
		$$\mathbb{E}(|W|^2)=(1-p)\cdot 0+p\cdot\mathbb{E}(|Y|^2)=p\cdot\mathbb{E}(Y^2)=p\sigma^2$$
		por otro lado el cuadrado del valor esperado (\ref{proof1:ValorEsperadoW}) es
		$$\mathbb{E}(|W|)^2=p^2\sigma^2\frac{2}{\pi}$$
		Finalmente la varianza de $|W|$ es
		$$\Var(|W|)=p\sigma^2-p^2\sigma^2\frac{2}{\pi}$$
	
	\end{proof}
\end{frame}

\begin{frame}
	\frametitle{Proposición}
	\begin{proof}
		Entonces la varianza de los radios de Gershgorin queda envuelta en la siguiente expresión
		\begin{equation}\label{proof1:paramEstabilidad}
			\Var(R_i)=(N-1)p\sigma^2\left (1-\frac{2p}{\pi}\right )
		\end{equation}
		y mientras sea menor a 1 se puede garantizar que los discos de Gershgorin van a cumplir (\ref{proof1:Desigualdad}) y en consecuencia los valores propios de $\mathcal{J}$ estarán contenidos en el semiplano negativo de $\mathbb{C}$.
	\end{proof}
\end{frame}

\begin{frame}
  \frametitle{Resultados de May}
  Ley Circular
  \begin{figure}[h!]
  	\centering
  	\includegraphics[scale=0.13]{../Texto/Imagenes/LeyCircularMay}
  	\caption{Distribución de eigenvalores que cumplen la Ley Circular de May. Para ambos sistemas se consideró $N=100$, una FDP normal centrada en $\mu=0$ y con $\sigma=0.2$ para una conectancia $C=\frac{1}{\sigma^2 N}-0.03$. (\textbf{A}) Considerando una matriz de interacciones estructuralmente simétrica. (\textbf{B}) Considerando una matriz de interacciones puramente aleatoria.}
  	\label{fig:LeyCircularMay}
  \end{figure}
\end{frame}
\begin{frame}
  \frametitle{Resultado de Allesina}
  Ley Elíptica
  \begin{figure}[h!]
  	\centering
  	\includegraphics[scale=0.10]{../Texto/Imagenes/LeyElipticaAllesina}
  	\caption{Distribución de eigenvalores que cumplen la Ley Elíptica de Allesina. Para ambos sistemas se consideró $N=100$, dos FDP normal respectivamente, una conectancia $C=0.12$ y en ambas se debe de considerar a la matriz de interacciones como estructuralmente simétrica. (\textbf{A}) Se considera una FDP normal para la parte triangular superior con $\mu_1=0.1$ y $\sigma_1 = 0.1$ y para la parte inferior se considera otra FDP normal con $\mu_2=0.3$ y $\sigma_2 = 0.2$. (\textbf{B}) Se considera una FDP normal para la parte triangular superior con $\mu_1=-0.1$ y $\sigma_1=0.1$ y para la parte inferior se considera otra FDP normal con $\mu_2=0.3$ y $\sigma_2=0.2$. }
  	\label{fig:LeyElipticaAllesina}
  \end{figure}
\end{frame}
\begin{frame}
  \frametitle{Transiciones de May}
  Define un parámetro de transición como: $\sigma<(nC)^{-1/2}$
  \begin{figure}[h!]
  	\centering
  	\includegraphics[scale = 0.10]{../Texto/Imagenes/TransicionDirvsNoDir}
  	\caption{Transición entre redes de May dirigidas vs No dirigidas. (\textbf{A}) Se considera para $\sigma = 0.2$ (\textbf{B}) Se considera para $\sigma=0.6$.}
  	\label{fig:TransicionDirvsNoDir}
  \end{figure}
\end{frame}
\begin{frame}
  \frametitle{Transiciones de May}
  Transición en función de $\sigma$
  \begin{figure}[h!]
  	\centering
	\includegraphics[width=0.52\textwidth]{../Texto/Imagenes/TransicionσDirvsNoDir} 
  	\caption{Variaciones en la transición para la matriz de May estructuralmente simétrica y para matriz puramente aleatoria. Se consideró el valor de la conectancia $C=0.6$.} 
  	\label{fig:TransicionσDirvsNoDir}
  \end{figure} 
\end{frame}


\end{document}
