\chapter{El modelo de Lotka-Volterra generalizado}

%En este capítulo solo se define y se construye el modelo, se deja listo para que en el siguiente capítulo pueda ser analizado.
\setlength{\parindent}{0cm} En este capítulo se introduce el modelo de Lotka-Volterra generalizado que será utilizado a lo largo de esta tesis. Se define el sistema dinámico, se especifican sus parámetros y la estructura de las interacciones, y posteriormente se analizan los puntos de equilibrio del sistema y su matriz Jacobiana.
\begin{equation}\label{eqn:LK}
	\frac{dx_i}{dt}=r_ix_i\left(1-\frac{\sum_{j=1}^N \alpha_{ij}x_j}{K_i}\right),\qquad i=1,...,N
\end{equation}
En el sistema \eqref{eqn:LK}, $x_i(t)$ representa la abundancia de la especie $i$, $r_i$ denota su tasa intrínseca de crecimiento y $K_i$ su capacidad de carga, heredados del modelo logístico \eqref{eqn:EqLogistica}. Los coeficientes $\alpha_{ij}$ codifican la intensidad y el signo de las interacciones entre las especies $i$ y $j$; dependiendo de su signo, estas interacciones pueden corresponder a escenarios de competencia, cooperación o a un régimen mixto generalizado.\\
\\
A continuación, se analiza el comportamiento del sistema en el caso particular de dos especies, con el fin de fijar intuiciones antes de abordar el régimen de alta dimensión. En este último caso, debido al carácter no lineal y la alta dimensionalidad del sistema, el análisis se apoya en herramientas computacionales.

\section{Caso particular para $N=2$}

En el caso particular $N=2$, el sistema \eqref{eqn:LK} se reduce a un par de ecuaciones diferenciales que conservan la misma estructura funcional y parámetros del modelo general,
\begin{align*}
	\frac{dx_1}{dt}&=r_1 x_1\left(1-\frac{\alpha_{11}x_1}{K_1}-\frac{\alpha_{12}x_2}{K_1}\right),\\
	\frac{dx_2}{dt}&=r_2 x_2\left(1-\frac{\alpha_{21}x_1}{K_2}-\frac{\alpha_{22}x_2}{K_2}\right).
\end{align*}

Dada la estructura del sistema, los coeficientes de interacción entre especies pueden organizarse en una \textbf{matriz de coeficientes de interacción $\Lambda$}, cuyos elementos $\alpha_{ij}$ cuantifican la relación entre las especies $j$ con $i$.
\begin{equation}\label{eqn:mIncidencias}
	\Lambda=
	\begin{pmatrix}
		\alpha_{11} & \alpha_{12}\\
		\alpha_{21} & \alpha_{22}
	\end{pmatrix}
\end{equation}
En esta representación, los coeficientes de la diagonal se fijan a un valor positivo, en particular $\alpha_{ii}=1$, lo que induce un término de autorregulación para cada especie. Este término actúa como un mecanismo disipativo que limita el crecimiento no acotado del sistema y garantiza la existencia de una escala de saturación en las abundancias. Cabe destacar que esta matriz no corresponde con la Jacobiana el sistema linealiado alrededor de un estado estacionario, sino que representa los coeficientes de interacción que aparecen directamente en el modelo no lineal. En capítulos posteriores se introducirá la matriz Jacobiana, la cual se construye a partir de $\Lambda$ pero posee una interpretación dinámica diferente.

\begin{ejemplo}\label{eg:2x2}
	A continuación se considera un sistema de Lotka-Volterra para $N=2$ especies, con coeficientes de interacción constantes y positivos, de modo que el sistema muestre un escenario competitivo.
	\begin{equation}\label{eqn:Sist2x2Comp}
		\begin{split}
			\frac{dx}{dt}&=2x\left(1-\frac{x}{2}\right)-xy\\
			\frac{dy}{dt}&=3y\left(1-\frac{y}{3}\right)-2xy
		\end{split}
	\end{equation}
	Para revelar su dinámica, se procede con la determinación de sus puntos de equilibrio, regiones en donde se mantiene constante. Para ello es necesario igualar a cero y resolver las ecuaciones del sistema \eqref{eqn:Sist2x2Comp}. El caso trivial esta asociado a $\vec{0}$, mientras que para el resto de casos es necesario aplicar un manejo algebráico para poder llegar a las soluciones. Respectivamente, el resto de puntos de equilibrio son $(2,0)$, $(0,3)$ y $(1,1)$ y se encomienda al lector corroborar estos resultados.\\
	\\
	Como se puede apreciar, el sistema (\ref{eqn:Sist2x2Comp}) tiene asociados un número multiple de puntos de equilibrio en contraste con los sistemas lineales que solo poseen uno. Es importante destacarlo, ya que en sistemas de mayor dimensión se tendrá un gran conjunto de estados estacionarios no triviales de determinar, de los cuales al menos uno es de interés. En este contexto, se conocen los estados estacionarios y por lo tanto se puede determinar su estabilidad a través de su matriz Jacobiana evaluada en dichos puntos. 
	\begin{equation}\label{eqn:Jacobiano2}
		\mathbb{J}(X)=\begin{pmatrix}
			r_1-\frac{2r_1\alpha_{11}x_1+r_1\alpha_{12}x_2}{K_1} & -\frac{r_1a_{12}x_1}{K_1}\\
			-\frac{r_2a_{21}x_2}{K_2} & r_2-\frac{2r_2\alpha_{22}x_2+r_2\alpha_{21}x_1}{K_2}
		\end{pmatrix}
	\end{equation}
\end{ejemplo}
Algo fundamental de notar es que la elección de condiciones iniciales es crucial para sentenciar si la dinámica va a converger o diverger. En la Figura (\ref{fig:CompetenciaEspecies}) se puede ver gráficamente; de acuerdo al punto de partida en el plano de soluciones, la tendencia hacia uno de los dos atractores será inexorable. Si por ejemplo se escoge la condición inicial $(1,1)$ se sabe que el sistema permanecerá constante para cierto periodo de tiempo, pero si por alguna razón existe una ligera perturbación que lo afecte, entonces ahora la dinámica va a diverger de este punto fijo y converger a cualquiera de los otros dos 
\begin{wrapfigure}{r}{0.5 \textwidth} \vspace{-30pt} \begin{center}
		\includegraphics[width=0.5\textwidth]{../Imagenes/Competencia de especies} 
	\end{center} 
	\vspace{-20pt} 
	\caption{Campo vectorial de las soluciones del sistema (\ref{eqn:Sist2x2Comp}) de dos especies.} 
	\vspace{10pt}
	\label{fig:CompetenciaEspecies}
\end{wrapfigure} 
(dependiendo de la dirección y magnitud de la perturbación).
\setlength{\parindent}{0cm} Además de obtener información visual acerca de la estabilidad de los puntos críticos, es posible visualizar las isoclinas del sistema que no son más que el conjunto de puntos donde se satisface:
\begin{equation*}
	\dot{x}=f(x,y)=0,\qquad\dot{y}=g(x,y)=0
\end{equation*}
de modo que en estas rectas una de las dinámicas permanece constante mientras que la otra tiene se mueve hacia arriba o abajo, izquierda o derecha según sea el caso. Como se puede observar, de la representación visual se pueden deducir muchos elementos importantes del sistema, ¿qué pasa cuando el número de ecuaciones no lineales ya no permite la representación visual del espacio fase? Una alternativa sería generar sub-espacios fase que puedan brindar una idea de como es la dinámica a nivel local, pero para ello se tendría que contar con algunos puntos fijos de interés. Sin embargo generar un gran número de gráficas solo para obtener conclusiones puede no ser una buena práctica, para ello existe una técnica analítica de la que ya se ha hablado, que servirá para determinar la estabilidad sin tener que recurrir a una gráfica.
\\
\\
En la sección (\ref{sec:Antecedentes}) se había mencionado que el proceso que llevó a cabo Robert May para poder explorar el sistema a nivel local, consistía en \textit{linealizar} el mismo a través de contar con una función vectorial $\textbf{F}$ no lineal y definir una matriz $A$ tal que sus entradas son $a_{ij}=\left .\frac{\partial f_i(X)}{\partial x_j}\right |_{X^*}$ que es equivalente a definir la matriz Jacobiana de $\textbf{F}$ evaluada en $X^*$.
\newpage
\begin{definición}\label{def:LKVectorial}
	Sea la función vectorial $\mathbf{F}:\mathbb{R}^n\to\mathbb{R}^n$ no lineal y asociada al sistema de Lotka-Volterra de especies en competencia. Se definen sus entradas de la siguiente manera
	\begin{equation}\label{eqn:Fmatricial}
		\mathbf{F}(X)=\begin{pmatrix}
			f_1(X)=r_1x_1\left(1-\frac{\sum_{j=1}^n \alpha_{1j}x_j}{K_1}\right)\\
			\vdots\\
			f_n(X)=r_nx_n\left(1-\frac{\sum_{j=1}^n \alpha_{nj}x_j}{K_n}\right)
		\end{pmatrix},\qquad\text{donde $X(t)=\left(x_1(t),...,x_n(t))\right)\in\mathbb{R}^n$.}
	\end{equation}
	Por lo tanto el sistema (\ref{eqn:LK}) puede ser re-escrito de una forma más compacta considerando
	\begin{equation}\label{eqn:LKmatricial}
		\dot{X}(t) = \mathbf{F}(X(t))
	\end{equation}
	Teniendo ahora la función vectorial que define al sistema de Lotka-Volterra, es directo definir la matriz Jacobiana para poder linealizar el sistema
	\begin{equation}\label{eqn:Jacobiano}
		\mathbb{J}_\mathbf{F}(X^*) = \begin{pmatrix}
			\frac{\partial f_1(X^*)}{\partial x_1} & \cdots &\frac{\partial f_1(X^*)}{\partial x_n}\\
			\vdots & \ddots & \vdots\\
			\frac{\partial f_n(X^*)}{\partial x_1} & \cdots &\frac{\partial f_n(X^*)}{\partial x_n}
		\end{pmatrix}
	\end{equation}
\end{definición}

\setlength{\parindent}{0cm}	Se necesitarán puntos fijos de interés y evaluarlos para poder obtener el sistema linealizado. Validando esta aseveración, se continúa con el ejemplo \ref{eg:2x2} determinando cada matriz jacobiana y su estabilidad asociada. Se comparará con lo que se muestra en la Figura (\ref{fig:CompetenciaEspecies}). La matriz jacobiana para este sistema es:
\begin{equation}\label{eqn:Jacobiano2}
	\mathbb{J}_\mathbf{F}(X)=\begin{pmatrix}
		r_1-\frac{2r_1\alpha_{11}x_1+r_1\alpha_{12}x_2}{K_1} & -\frac{r_1a_{12}x_1}{K_1}\\
		-\frac{r_2a_{21}x_2}{K_2} & r_2-\frac{2r_2\alpha_{22}x_2+r_2\alpha_{21}x_1}{K_2}
	\end{pmatrix}
\end{equation}
considerando que las entradas $\alpha_{ii}=1$ debido a como se define la matriz de incidencias (\ref{eqn:mIncidencias}). Sustituyendo y operando sobre nuestro sistema, al evaluar los puntos fijos antes encontrados se tienen las siguientes matrices de interacciones:
$$
\mathbb{J}_{(2,0)} = \begin{pmatrix}
	-2 & -2\\
	0 & -1
\end{pmatrix},\qquad \mathbb{J}_{(0,3)}=\begin{pmatrix}
	-1 & 0\\
	-6 & -3
\end{pmatrix},\qquad \mathbb{J}_{(1,1)}=\begin{pmatrix}
	-1 & -1\\
	-2 & -1
\end{pmatrix},\qquad \mathbb{J}_{(0,0)}=\begin{pmatrix}
	2 & 0 \\
	0 & 3
\end{pmatrix}
$$
%%%%%%%%Checkpoint
Teniendo las matrices, solo resta determinar sus valores propios y evaluar sus partes reales para concluir con el tipo de estabilidad asociado a cada uno de los puntos fijos. Realizando el álgebra correspondiente se encuentra lo siguiente
%Por lo tanto los valores propios de las primeras dos matrices de interacciones deben ser negativos para que sustenten los atractores de la figura (\ref{fig:CompetenciaEspecies}). Mientras que los valores propios de $\mathbb{J}_{(1,1)}$ deben ser uno negativo y otro positivo para sustentar al punto silla. Para el caso de la matriz $\mathbb{J}_{(0,0)}$ sus valores propios deben tener parte real positiva para que sustenten el repulsor. 
\begin{align*}
	\mathbb{J}_{(2,0)}&\Longrightarrow\ \lambda_1 = -2,\quad\lambda_2 = -1\\
	\mathbb{J}_{(0,3)}&\Longrightarrow\ \lambda_1 = -3,\quad\lambda_2 = -1\\
	\mathbb{J}_{(1,1)}&\Longrightarrow\ \lambda_1 = -1+\sqrt{2},\quad\lambda_2 = -1-\sqrt{2}\\
	\mathbb{J}_{(0,0)}&\Longrightarrow\ \lambda_1 = 2,\quad\lambda_2 = 3\\
\end{align*}
Se puede observar que las partes reales coinciden con la estabilidad de cada punto fijo de la Figura (\ref{fig:CompetenciaEspecies}). Esta técnica resultará muy útil para cuando se tengan que resolver los sistemas con $N\gg 1$ ecuaciones diferenciales. La parte real de los valores propios de $\mathbb{J}_{\textbf{F}}(X^*)$ determinarán su estabilidad.
%%%%%%%%% Checkpoint
\section{Generalizando a $N\gg 1$ especies.}

Con base en la sección anterior, se tiene una idea de como resolver el sistema dinámico de Lotka-Volterra (\ref{eqn:LK}); en principio se requiere hallar puntos fijos que satisfagan $\textbf{F}(X^*)=\vec{0}$ para poder hallar la matriz Jacobiana del sistema (\ref{eqn:Jacobiano}) y así poder explorar su estabilidad. Todo este proceso se debe de implementar mediante algoritmos los cuales se irán revisando en la sección (apéndice tal\footnote{agregar seccion.}). Sin embargo, en lo que corresponde a esta sección será en definir las interacciones entre especies del sistema mediante la \textit{matriz de incidencias} y ver que parámetros la gobiernan para poder plantear una serie de experimentos/simulaciones que resuelvan las conjeturas planteadas.
\\
\\
Para comenzar el modelado de las interacciones de (\ref{eqn:LK}), es necesario comenzar a hablar de redes/grafos, ya que con este artilugio matemático se podrán representar a las especies de manera conveniente, sobre todo para investigar las implicaciones de probar con diversas topologías de red por ejemplo: libres de escala, de mundo pequeño o como en nuestro caso: aleatorias. Una red es considerada una colección de \textit{nodos} que se encuentran unidos por \textit{enlaces} \cite{newman2018networks}. Para definir redes siempre es necesario establecer que es lo que representan los nodos y los enlaces, en nuestro caso los nodos representan directamente las especies que participan en el sistema mientras que los enlaces (con pesos) serían sus interacciones.\\
\\
Si se define un conjunto de especies $x_i$ y sabemos que se relacionan por medio de los coeficientes $\alpha_{ij}$ que
\begin{wrapfigure}{l}{0.45 \textwidth} \vspace{-30pt} \begin{center}
		\includesvg[width=.5\textwidth]{../Imagenes/karateLu} 
	\end{center} 
	\vspace{-20pt} 
	\caption{Red de Karate de Zachary} 
	\vspace{-20pt}
	\label{fig:RedKarate}
\end{wrapfigure} 
aparecen en las ecuaciones de (\ref{eqn:LK}), entonces las interacciones entre las especies $x_i$ y $x_j$ se dan para cuando $\alpha_{ij}\neq 0$, y éste representaría el peso del enlace $j\to i$ de la red que se tiene en mente. En el mundo es posible encontrar diferentes tipos de redes con cierto significado, tales como la red de energética de un país, redes de amistades en una universidad o redes de acciones que cotizan en la bolsa de valores; en el caso de la Figura (\ref{fig:RedKarate}) es una red conocida como \href{https://en.wikipedia.org/wiki/Zachary%27s_karate_club}{Red del club Karate de Zachary} \cite{newman2018networks}. Todo tipo de red tiene una representación algebraica que conviene mucho tener en cuenta para la construcción del modelo, es conocida como la \textit{matriz de adyacencia.}\\
\\
\begin{definición}\label{def:matrizdeadyacencia}
	Sea $\mathcal{A}\in\mathrm{M}_n(\mathbb{R}) $. Se define la matriz de adyacencia tal que sus entradas son de la siguiente forma
	$$\mathcal{A}_{ij}= 
	\begin{cases}
		1, \ \text{si $\exists$ un enlace entre el nodo $i$ y el nodo $j$.}\\
		0, \ \text{en otro caso}.
	\end{cases}$$
	De esta forma se puede acceder a la estructura de la red sin tener que dibujarla como en la Figura (\ref{fig:RedKarate}), cada renglón de la matriz es un nodo y sus columnas son los nodos disponibles que tiene para conectarse, incluso consigo mismo. Si se le agregan pesos al enlace, entonces $\mathcal{A}_{ij}\in\mathbb{R}$. Al pasar a la siguiente etapa de la construcción de la matriz de incidencias $\Lambda$, estos elementos serán escogidos de una distribución estadística. 
\end{definición}
%%%%%%% Checkpoint
\begin{ejemplo}
	Para poder apreciar la matriz de adyacencia definamos una red de 10 nodos y veamos la matriz de adyacencia que le corresponde. Cada nodo ha sido marcado para poderlo identificar y relacionar con la matriz de adyacencia. Los renglones y columnas de la matriz representan los nodos, siendo el primer renglón el primer nodo, el quinto renglón será el quinto nodo; esto pasa de manera equivalente con las columnas, la cuarta columna corresponde con el cuarto nodo, la octava columna corresponde con el octavo nodo. Por tanto, mediante la matriz de adyacencia sabemos que el primer nodo (renglón 1) esta enlazado con el noveno nodo (columna 9) ya que existe un uno, mientras que el primer renglón y la quinta columna hay un cero lo que indica que no existe un enlace entre estos nodos. 
\end{ejemplo}
\begin{wrapfigure}{r}{0.5 \textwidth} \vspace{-30pt} \begin{center}
		\includesvg[width=0.5\textwidth]{../Imagenes/red10Lu} 
	\end{center} 
	\vspace{-20pt} 
	\caption{Red no dirigida de 10 nodos.} 
	\vspace{-150pt}
	\label{fig:Red10}
\end{wrapfigure} 

$$
A=\begin{pmatrix}
	0 & 1 & 0 & 1 & 0 & 0 & 0 & 0 & 1 & 0 \\
	1 & 0 & 1 & 0 & 0 & 1 & 0 & 0 & 0 & 0 \\
	0 & 1 & 0 & 0 & 0 & 0 & 1 & 1 & 0 & 0 \\
	1 & 0 & 0 & 0 & 1 & 1 & 1 & 0 & 0 & 1 \\
	0 & 0 & 0 & 1 & 0 & 0 & 0 & 0 & 1 & 1 \\
	0 & 1 & 0 & 1 & 0 & 0 & 1 & 1 & 0 & 1 \\
	0 & 0 & 1 & 1 & 0 & 1 & 0 & 0 & 1 & 0 \\
	0 & 0 & 1 & 0 & 0 & 1 & 0 & 0 & 1 & 0 \\
	1 & 0 & 0 & 0 & 1 & 0 & 1 & 1 & 0 & 1 \\
	0 & 0 & 0 & 1 & 1 & 1 & 0 & 0 & 1 & 0 \\
\end{pmatrix}
$$\\
\\
También hay que destacar qué la matriz es simétrica y que la diagonal es igual a cero: para el primer punto se debe notar que la relación de los enlaces entre nodos no tiene dirección, es decir, que exista un enlace entre nodos significa que el nodo $i$ se conecta con $j$ y viceversa. Por tanto la red de la figura (\ref{fig:Red10}) es \textit{no dirigida} puesto que no hay una dirección preferencial en el enlace. Para el segundo punto se puede deducir que los nodos podrían relacionarse consigo mismo, en este caso particular no lo hacen pero si es posible la existencia de \textit{autoenlaces}. Para el sistema (\ref{eqn:LK}) estos autoenlaces representan las auto-interacciones que dan el carácter logístico (\ref{eqn:EqLogistica}). Más adelante se hablará de la importancia de estos autoenlaces.
\\
\\
Cuando se tiene el caso en que los enlaces tienen una dirección preferencial de nodo a nodo, se dice que corresponde a una \textit{red dirigida}. En este caso el enlace podrá ir del nodo $i$ al nodo $j$ pero no necesariamente lo hará en sentido contrario, deberá definirse explícitamente. En el mundo también existe un gran conjunto de redes dirigidas como lo son las citaciones académicas, la propia WWW (World Wide Web), incluso redes tróficas. Y para este caso también se tiene asociada una matriz de adyacencia con una ligera modificación con respecto de la definición \ref{def:matrizdeadyacencia}.
\begin{definición}
	Sea $\mathcal{D}\in\mathrm{M}_n(\mathbb{R})$, matriz de adyacencia de una red dirigida. Se definen sus elementos de la siguiente manera:
	$$
	\mathcal{D}_{ij}=\begin{cases}
		1,\qquad\text{Si existe un enlace del nodo $i$ al nodo $j$}\\
		0,\qquad\text{otro caso.}
	\end{cases}
	$$
	Los enlaces de las redes dirigidas van a estar representados por flechas para que puedan mostrar adecuadamente las direcciones correspondientes entre los nodos. Las redes ecológicas que Robert May definió en su trabajo son consideradas dirigidas, ya que el sentido de sus enlace tienen una dirección preferencial.
\end{definición}

\begin{ejemplo}
	Se tiene la siguiente red dirigida de 10 nodos con exactamente 14 enlaces. La matriz de adyacencia asociada es la siguiente
\end{ejemplo}
\begin{wrapfigure}{l}{0.5 \textwidth} \vspace{-44pt} \begin{center}
		\includesvg[width=0.52\textwidth]{../Imagenes/Red10DirLu} 
	\end{center} 
	\vspace{-20pt} 
	\caption{Red dirigida de 10 nodos.} 
	\vspace{-200pt}
	\label{fig:Red10Dir}
\end{wrapfigure} 
$$
D = \begin{pmatrix}
	0 & 0 & 0 & 0 & 0 & 0 & 0 & 0 & 0 & 1 \\
	1 & 0 & 1 & 0 & 0 & 0 & 0 & 1 & 0 & 0 \\
	0 & 0 & 0 & 1 & 1 & 0 & 0 & 0 & 0 & 1 \\
	1 & 0 & 0 & 0 & 0 & 0 & 0 & 0 & 0 & 0 \\
	0 & 0 & 1 & 0 & 0 & 0 & 0 & 1 & 0 & 0 \\
	0 & 1 & 0 & 0 & 0 & 0 & 0 & 0 & 0 & 0 \\
	0 & 0 & 1 & 0 & 0 & 0 & 0 & 0 & 0 & 0 \\
	0 & 0 & 0 & 0 & 0 & 0 & 0 & 0 & 0 & 0 \\
	0 & 1 & 0 & 0 & 0 & 0 & 0 & 0 & 0 & 0 \\
	0 & 0 & 0 & 0 & 0 & 0 & 1 & 0 & 0 & 0 \\
\end{pmatrix}
$$
\newpage
Ahora se puede notar que la matriz de adyacencia no es simétrica y que en consecuencia los enlaces presentan una dirección preferencial. Ambas visiones se van a tomar en cuenta para modelar al sistema (\ref{eqn:LK}) generalizado, sin embargo, tendrá mayor protagonismo la \textit{red no dirigida} sobre la otra, ya que en un principio los experimentos fueron diseñados bajo esta lógica. Habiendo definido las redes y sus matrices de adyacencia, se debe seguir el camino definiendo que tipo de red se ocupará para el modelo.

\subsection{Red de incidencias}

La \textit{red de incidencias} será la representación de la relación entre especies bajo la lógica del sistema de Lotka-Volterra generalizado; dicha red tiene asociada su matriz de adyacencia que se denominará \textit{matriz de incidencias} la cual tendrá tres principales características: puede ser de una red dirigida o no dirigida, definir la dirección de las interacciones hace más completo al sistema, sin embargo se usarán mayormente redes no dirigidas. La topología de la red que se va a utilizar es la de Erdös–Rényi \cite{posfai2016network} conocida por tener un carácter aleatorio con base en una probabilidad $p$. Y aunado a todo ello se considerará un peso para cada enlace cuyo valor vendrá de una distribución normal centrada en $\mu=0$ y desviación estándar $\sigma$.\\
\\
La red de Erdös–Rényi es ampliamente utilizada para aprender sobre la estructura y propiedades de la redes y con ello poder extender ese aprendizaje al estudio de las llamadas \textit{redes libres de escala}. La diferencia sustancial entre la red aleatoria y la red libre de escala es que la primera tiene muy pocas (o ninguna) aplicaciones en escenarios de la naturaleza \cite{posfai2016network}. Las estructuras en la naturaleza están representadas matemáticamente por las redes libres de escala, siguiendo leyes de potencia y otra variedad de propiedades. Sin embargo este trabajo únicamente se centrará en el uso del modelo de Erdös–Rényi aplicado al sistema (\ref{eqn:LK}) para tener un primer acercamiento a la dinámica que produce, con intenciones de extender el estudio al caso de las redes libres de escala\footnote{Un modelo teórico que rescata todas las propiedades de las redes libres de escala es el de Albert-Barabasi \cite{posfai2016network}.}. 
\begin{definición}\label{def:redAleatoria}
	Sea un conjunto de $N$ nodos sin enlaces asociados. Se define la red aleatoria $\mathcal{G}_\mathcal{E}$ con su matriz de adyacencia asociada $\mathcal{E}\in\mathrm{M}_n(\mathbb{R})$ si para cada par de nodos $n_i$ y $n_j$ de las $\binom{N}{2}$ posibles combinaciones se definen sus enlaces aleatorios dada una probabilidad $p$ y un número aleatorio $r\in[0,1]$ con base en la siguiente regla:
	$$
	L_{ij}=\begin{cases}
		\exists,\qquad\text{Si }r<p\\
		\nexists,\qquad\text{Si }r\geq p
	\end{cases},\qquad L_{ii}=0
	$$
	desde luego que entre mayor sea la $p$ se tendrá mayor cantidad de enlaces. En este caso no se anuncia una dirección preferencial, por lo que la red aleatoria de Erdös–Rényi es \textit{no dirigida}\footnote{En esta referencia \cite{posfai2016network} el lector podrá conocer más sobre sus propiedades.}. Ya que $\mathcal{G}_\mathcal{E}$ es no dirigida, su matriz de adyacencia $\mathcal{E}$ es simétrica. \\
	\\
	Sin embargo, es posible extender esta definición a redes aleatorias dirigidas suponiendo que la conexión tiene dirección preferencial, por lo tanto ahora existirán $N(N-1)$ posibles combinaciones de nodos. Si solo se considerara uno de los dos casos, es decir, para $\binom{N}{2}$ combinaciones de nodos entonces se tendría la mitad de posibles conexiones, la matriz de adyacencia sería triangular superior por haber considerado únicamente la dirección de $n_i$ a $n_j$. Para llenar la parte triangular inferior de esta matriz se deben considerar las conexiones de $n_j$ a $n_i$, como resultado ahora se tiene una red aleatoria dirigida cuya matriz de adyacencia es no simétrica. En la sección (\ref{sec:RedesAleatorias}) se encuentra la implementación computacional de ambas visiones de la red aleatoria.
\end{definición}
\setlength{\parindent}{0cm}En este punto solamente falta de considerar la magnitud de las interacciones entre especies para terminar de construir la matriz de incidencias. Para ello se va a tomar en cuenta una matriz de $\mathcal{M}\in\mathrm{M}_n(\mathbb{R})$ que estará mapeada con valores que vienen de una distribución normal centrada en $\mu=0$ y desviación estándar $\sigma$, es decir, $\mathcal{M}_{ij}\in\mathcal{N}(0,\sigma)$.
\begin{definición}\label{def:MatrizIncidencias}
	Sea una red aleatoria $\mathcal{G}_\mathcal{E}$ dirigida o no dirigida con matriz de adyacencia $\mathcal{E}\in\mathrm{M}_n(\mathbb{R})$. Sea una matriz $\mathcal{M}\in\mathrm{M}_n(\mathbb{R})$ de entradas aleatorias provenientes de una distribución normal $\mathcal{N}(0,\sigma)$. Se define a $\Lambda$ la \textit{matriz de incidencias} como el producto de Hadamard \cite{HadamardProduct} de la matriz de adyacencia con la matriz de entradas aleatorias sumada con la matriz identidad:
	\begin{equation}\label{eqn:MatrizIncidencias}
		\Lambda=(\mathcal{E}\odot \mathcal{M}) + I
	\end{equation}
	El producto agrega pesos de la distribución normal a cada posible enlace de la red aleatoria y la suma con de la identidad es para poder agregar autoenlaces de peso $1$ a cada nodo de la misma.
\end{definición}
\setlength{\parindent}{0cm} La matriz de incidencias $\Lambda$ puede ser \textit{estructuralmente simétrica}\footnote{Es una matriz cuyas entradas cumplen $B_{ij}\neq B_{ji}\neq 0$ ó $B_{ij}=B_{ji}=0$ para toda $B_{ij}\in B$, quiere decir que aunque $B$ no sea simétrica, relativo a las posiciones de sus entradas si lo es.} si se escoge una red $\mathcal{G}_\mathcal{E}$ no dirigida, o puramente aleatoria si se escoge la opción  de red dirigida. Las implicaciones más importantes de esta distinción es que para $\Lambda$ estructuralmente simétrica únicamente podrán existir 3 tipos de interacciones mientras que la puramente aleatoria considera hasta 5 de ellas posibles. Los elementos $\alpha_{ij}\in\Lambda$ serán los coeficientes de interacción que aparecen en las ecuaciones de (\ref{eqn:LK}) y en este punto se ha concluido un paso más de la construcción del modelo.\\
\\
En la figura (\ref{fig:RedIncidencias}) se muestra una representación visual de las interacciones entre especies de los sistemas por resolver (\ref{eqn:LK}). En la sección (\ref{sec:redIncidencias}) se muestra la implementación computacional de la red de incidencias. En la siguiente sección se hablará acerca de los tipos de interacciones posibles que se pueden tener, para cada caso descrito anteriormente.
\newpage
\begin{wrapfigure}{r}{0.5 \textwidth} \vspace{-70pt} \begin{center}
		\includesvg[width=0.52\textwidth]{../Imagenes/RedIncidencias} 
	\end{center} 
	\vspace{-52pt} 
	\caption{Red de incidencias de 8 nodos bajo la topología de una red aleatoria dirigida con $p=0.15$ y una matriz aleatoria con $\mu=0$ y $\sigma=0.2$.} 
	\vspace{-10pt}
	\label{fig:RedIncidencias}
\end{wrapfigure} 


\subsection{Tipos de interacciones}

La red de incidencias es de tipo dirigida independientemente del tipo de red aleatoria que se decida elegir para su construcción, a lo mucho podrá ser estructuralmente simétrica pero los pesos de cada enlace $i\to j$ y $j\to i$ son diferentes lo que implica que $\alpha_{ij}\neq\alpha_{ji}$ para toda $\alpha_{ij}\in\Lambda$. Puede considerarse como un efecto natural puesto que el nivel de interacción entre especies difícilmente va a ser igual. Existen más interacciones además de la competencia que generan dinámicas interesantes, en esta sección se explorarán dichas interacciones y su aplicación en la matriz de incidencias.
\\
\\
Hasta ahora únicamente se han discutido las ecuaciones de (\ref{eqn:LK}) bajo el escenario de la competencia, es decir para toda $\alpha_{ij}\in\Lambda$ positiva. El caso particular $N=2$ ha develado que este sistema es incapaz de superar sus capacidades de carga, por lo que tienen un crecimiento controlado. Pero ¿qué pasa si se comienzan a considerar $\alpha_{ij}<0$ provenientes de la distribución normal? ¿Cómo podría afectar en la dinámica resultante y qué pasaría con las capacidades de carga del sistema? Para comenzar a explorar las respuestas a estas preguntas hay que definir el tipo de interacciones posibles en $\Lambda$ y May presenta un abanico de 5 de ellas \cite{may2019stability}.\\
\\
Las interacciones a las que May refiere, únicamente se aplican a su \textit{community matrix} que sería lo equivalente a las matrices Jacobianas de nuestro sistema evaluadas en algún punto fijo (\ref{eqn:Jacobiano}). Estas interacciones no necesariamente aplican a la matriz de incidencias $\Lambda$, sin embargo podrán aplicarse  si se les voltean el signo tal y como se verá más adelante. Las posibles interacciones se reparten de la siguiente forma: para sistemas de May con matrices estructuralmente simétricas únicamente podrán acceder a tres tipos de interacción, las de competencia $(--)$, las de cooperación\footnote{Conocidas también por mutualismo o simbiosis.} $(++)$ y las de presa-depreador $(+-)$ ó $(-+)$.
\\
\\
En sistemas de May puramente aleatorios se tienen 5 tipos de interacción, ya que su matriz de interacciones no necesariamente es estructuralmente simétrica. Las que se agregan son: comensalismo $(+0)$ y amensalismo $(-0)$. \\
\\
La relación de cooperación implica que ambas especies se verán beneficiadas de su interacción mutua, en contraste con la competencia en la que ambas se van a ver perjudicadas. Para el caso de la interacción presa-depredador, una de las especies se verá beneficiada mientras que la otra saldrá perjudicada de dicho beneficio. En el  comensalismo una de las especies se beneficia mientras que la otra no presenta ningún impacto. Por el contrario, en el amensalismo una de las especies se perjudica mientras que la otra no presenta algún cambio.

\subsubsection*{Interacciones de May aplicadas a la red de incidencias}

Las interacciones que se aplican a la matriz de incidencias son las mismas que May estipula para sus matrices de interacción solamente que con el signo contrario. Para el caso de la competencia se tendrán $(++)$, en cooperación serán $(--)$, presa-depredación $(-+)$ ó $(+-)$, comensalismo $(-0)$ y por último amensalismo $(+0)$. Para poder ver de que forma afectan los coeficientes $\alpha_{ij}\in\Lambda$  a la dinámica del sistema, extendemos sus ecuaciones (\ref{eqn:LK}) de la siguiente manera:
\begin{equation}\label{eqn:LKextendida}
	r_ix_i\left (1 - \frac{\alpha_{ii}x_i+\alpha_{i1}x_1+\cdots+\alpha_{iN}x_N}{K_i}\right )\quad\Longleftrightarrow\quad r_ix_i\left (1-\frac{\alpha_{ii}x_i}{K_i}\right )-r_ix_i\left (\frac{\sum_{j\neq i}\alpha_{ij}x_j}{K_i}\right )
\end{equation}
El sistema se puede separar en una componente logística y en una suma de términos no lineales. Para que la componente logística mantenga regulado su crecimiento (omitiendo el resto de términos), es necesario que $\alpha_{ii}>0$, por simplicidad y convenciencia se ha definido en la matriz de incidencias que todos estos elementos sean $\alpha_{ii}=1$, pero en esencia es suficiente con que sean mayor a cero. Observe que si $\alpha_{ii}\leq 0$, entonces el crecimiento será exponencial y sin límites.
\\
\\
Los términos de la suma de (\ref{eqn:LKextendida}) van a aportar valor al término logístico y va a depender del resto de coeficientes $\alpha_{ij}$ junto con su signo. Tomando alguna $\alpha_{ij}$ arbitraria, si su signo es negativo entonces se vuelve positivo con el signo menos de la izquierda y aportará crecimiento a la dinámica de $x_i$, este escenario puede suceder en interacciones de cooperación, en la interacción depredador únicamente y en el comensalismo. Por el contrario si $\alpha_{ij}>0$ entonces va a restar crecimiento a la dinámica de $x_i$ (tal y como se aprecia en el ejemplo \ref{eg:2x2}). En contraparte, este escenario se va a dar en la competencia, en la presa únicamente y en el amensalismo. Nótese que cuando $\alpha_{ij}=0$ no impacta en la evolución de $x_i$. \\
\\
Las interacciones negativas que agregan valor al crecimiento de cada especie, generan la oportunidad de que dicho incremento logre sobrepasar la(s) capacidad(es) de carga del sistema, lo que puede generar que el punto de estabilidad de las especies quede por arriba de dichas $K_i$, algo que no se permite en el sistema puramente de competencia. Explorar el balance de las interacciones (positivas y negativas) puede ser un elemento importante que defina el tipo de estabilidad del sistema.\\
\\
%Sin embargo, debe de existir un balance entre las interacciones positivas y negativas, ya que si las segundas se sobreponen a las primeras, puede ocurrir un colapso del sistema traducido en crecimiento desmedido. Este escenario corresponderá a aquellos sistemas (\ref{eqn:LK}) inestables resultantes de ciertas configuraciones en $p$, $\sigma$ y $N$ de la matriz de incidencias y se irá revisando más adelante.\\\\
Dependiendo de la $p$ que forma la matriz de incidencias, cada especie podrá tener a lo mucho $N$ interacciones posibles con diversas magnitudes en función de $\sigma$, lo que implica que cada especie puede tener interacciones de cooperación, competencia, etc, de forma aleatoria (dependiendo si la matriz es estructuralmente simétrica o no). La dinámica resultante se acompleja conforme $N$ es mayor debido a todas las posibles interacciones que afectan significativamente al desarrollo de cada especie. Partiendo de la estructura aleatoria de la red y el peso de las interacciones provenientes de $\mathcal{N}(0,\sigma)$, es complejo de idear tan si quiera un bosquejo de la dinámica resultante del sistema (\ref{eqn:LK}).

\subsubsection*{Interacciones de $\Lambda$ para $N=2$.}

\begin{ejemplo}\label{eg:2x2CoopyDemás}
	Se explorará un caso particular de un sistema (\ref{eqn:LK}) para $N=2$ con interacciones de cooperación. Se corroborará si efectivamente su dinámica es capaz de sobrepasar sus capacidades de carga. A través del espacio fase y las series de tiempo se observará dicho fenómeno. El sistema en concreto es
	\begin{equation}\label{eqn:Sist2x2Coop}
		\begin{split}
			\frac{dx}{dt} &= 2x\left (1-\frac{x}{2}\right )+\frac{1}{2}xy\\
			\frac{dy}{dt} &= 3y\left (1-\frac{y}{3}\right )+xy
		\end{split}
	\end{equation}
	La matriz de incidencias asociada a este caso sería 
	$$
	\Lambda = \begin{pmatrix}
		1 & -\frac{1}{2}\\
		-1 & 1
	\end{pmatrix}
	$$
	A diferencia del sistema del Ejemplo \ref{eg:2x2} (Ec. \ref{eqn:Sist2x2Comp}) su anti-diagonal es negativa lo que indica que son coeficientes de interacción $(--)$ que propician la cooperación entre ambas especies y fomentan sus crecimientos. Los puntos fijos para este sistema ahora son: $(0,0)$, $(2,0)$, $(0,3)$ y $(7,10)$; las matrices jacobianas asociadas son:
	$$
	\mathbb{J}_{(0,0)}=\begin{pmatrix}
		2 & 0 \\
		0 & 3
	\end{pmatrix},\qquad\mathbb{J}_{(2,0)}=\begin{pmatrix}
	-2 & 1\\
	0 & 5
	\end{pmatrix},\qquad\mathbb{J}_{(0,3)}=\begin{pmatrix}
	3.5 & 0 \\
	3 & -3
	\end{pmatrix},\qquad\mathbb{J}_{(7,10)}=\begin{pmatrix}
	-7 & 3.5\\
	10 & -10
	\end{pmatrix}
	$$
	Realizando el cálculo de los valores propios de cada una de las matrices de interacción se encuentra que para el primer punto fijo se tiene el conocido y trivial repulsor. En el caso de los puntos fijos de los ejes ahora su estabilidad ha cambiado con respecto del sistema (\ref{eqn:Sist2x2Comp}), se tienen puntos sillas que resultan ser inestables, ya que para $t\to\infty$ las soluciones terminan divergiendo. Por último se encuentra que los valores propios del punto fijo restante son negativos, lo que implica que todas las soluciones del sistema irán a converger a este punto.
	\begin{figure}[h!]
		\centering
		\includegraphics[scale=0.24]{../Imagenes/Cooperacion de especies}
		\caption{Sistema de Lotka-Volterra con interacciones de cooperación dados por las ecuaciones (\ref{eqn:Sist2x2Coop}). Tasas de crecimiento y capacidades de carga: $r_x=K_x=2$ y $r_y=K_y=3$. \textbf{A}) Series de tiempo del sistema para las especies $x(t)$ y $y(t)$ bajo la condición inicial $(1,2)$. (\textbf{B}) Espacio fase del sistema con sus puntos fijos asociados, se muestra solamente un único punto fijo estable.}
		\label{fig:CooperacionEspecies}
	\end{figure}\\
	\\
	 Esto es cuanto menos interesante, ya que se puede observar que efectivamente la cooperación entre las especies $x$ y $y$ es capaz de superar sus propias capacidades de carga que fungían como límites. Bajo este escenario el concepto de la capacidad de carga podría tomar otro significado: ahora será un parámetro que regule el crecimiento de las especies, cuanto mayor sea la capacidad de carga será más difícil el crecimiento de las $x_i$; en caso contrario podrá haber una fácil apertura para el crecimiento desmedido debido a que las $K_i$ no serán capaces de contener las interacciones $\alpha_{ij}<0$, aunque por su puesto también dependerá de la magnitud de las $\alpha_{ij}\in\Lambda$.\\
	 \\
 	La cooperación entre especies fomenta su crecimiento y ahora su estabilidad se posiciona en puntos que quedan por arriba de sus capacidades de carga. Cuando se tienen las interacciones comensalismo (-0) y amensalismo (+0), solo una de las especies seguirá su comportamiento logístico mientras que la otra encontrará su estabilidad arriba o debajo de su capacidad de carga. La interacción de depredación $(-+)$ la especie depredadora se estabiliza arriba de su $K_i$ y la presa por debajo de la misma.\\
 	\\
 	Modificando el sistema (\ref{eqn:Sist2x2Coop}) de tal modo que se considere cada una de estas interacciones. En este caso se elige que $\alpha_{21}=0$ de la ecuación $\dot{y}$ y $\alpha_{12}=\pm\frac{1}{2}$ de la ecuación $\dot{x}$ para cubrir los casos de comensalismo y amensalismo. La ecuación $\dot{y}$ queda reducida a una ecuación logística, la cual no se verá afectada por la dinámica de $\dot{x}$ y se mantendrá estable en su capacidad de carga. 	Para el caso de depredación se utiliza $\alpha_{12}=\frac{1}{2}$ y $\alpha_{21}=-1$ y es como una combinación de los anteriores, una de las especies logra estabilizarse en un punto superior a su capacidad de carga mientras que la otra se establece en algún punto menor a su capacidad de carga pero sin llegar a extinguirse. En estos casos particulares los coeficientes $\alpha_{ij}<0$ fueron soportados sus capacidades de carga, pero dependiendo de su magnitud es que pudo desenvolver en sistemas estables. Hasta el momento, la estabilidad va a depender de: la magnitud de las capacidades de carga y los coeficientes de interacción en relación a las anteriores.
 	\begin{figure}[h!]
 		\centering
 		\includegraphics[scale=0.2]{../Imagenes/STrestoInteracciones}
 		\caption{Series de tiempo para las interacciones comensalismo, amensalismo y depredación. (\textbf{A}) Para el comensalismo se definió $\alpha_{21}=0$ y $\alpha_{12}=-\frac{1}{2}$. (\textbf{B}) Para el amensalismo se consideró $\alpha_{21}=0$ y $\alpha_{12}=\frac{1}{2}$ (\textbf{C}) Para la depredación se consideró $\alpha_{21}=-1$ y $\alpha_{12}=\frac{1}{2}$.}
 		\label{fig:RestoInteraccionesST}
 	\end{figure}
 	

\end{ejemplo}
En esta sección se ha construido la matriz de incidencias $\Lambda$ necesaria para definir las interacciones del sistema (\ref{eqn:LK}). Se hizo con base en una red de Erdös–Rényi dirigida o no dirigida, contemplando que el peso de los enlaces provienen de una distribución normal centrada en cero y desviación estándar $\sigma$, y agregando autoenlaces con peso $1$. Posteriormente se revisaron las posibles interacciones que se pueden obtener de $\Lambda$ siendo 3 o hasta 5 dependiendo del tipo de red aleatoria escogida. Se ha analizado que los coeficientes $\alpha_{ii}$ deben ser positivos para que respeten el comportamiento logístico de cada especie, y que los coeficientes $\alpha_{ij}<0$ son capaces de agregar valor al crecimiento de las especies. \\
\\
Por último se ha explorado en un primer nivel que la estabilidad puede depender del balance entre coeficientes positivos y negativos, y que la capacidad de carga juega un papel importante en la contención de los crecimientos. Por lo tanto ya se tiene todo para comenzar a integrar numéricamente el sistema para cualquier $N$, el siguiente paso será determinar los puntos fijos y la matriz Jacobiana del sistema para analizar aspectos alrededor de ella. En la sección (\ref{sec:poblacionesLK}) se muestra la implementación computacional de este sistema.
\newpage
\section{Puntos fijos}\label{sec:PuntosFijos}

En este punto conviene realizar un análisis sobre los puntos fijos del sistema, ya que  es el puente entre $\Lambda$ con la tupla $(p,\sigma,N)$ y la matriz Jacobiana con su distribución espectral, misma que define la estabilidad del sistema. Las diferentes configuraciones de la tupla generarán a su vez diversas configuraciones de $\Lambda$ que devengan en puntos fijos estables o inestables. Por lo tanto será indispensable hallar dichas configuraciones para obtener una noción del mapa de estabilidad del sistema. Para comenzar, es necesario hallar los puntos fijos que se obtienen al resolver el conjunto de ecuaciones que se encuentra entre paréntesis del sistema (\ref{eqn:LK})
\begin{equation}\label{eqn:EcsPuntoFijo}
	1-\frac{\sum_{j=1}^N\alpha_{ij}x_j}{K_i}=0
\end{equation}
Es posible resolverlo bajo diversas restricciones, por ejemplo asumiendo que algunas especies no sobreviven y resolver para el subconjunto que no contempla dicha especie, es decir, asumiendo $x_k=0$ para alguna (o varias) $k\in\{1,...,N\}$. Pero el caso de interés es asumiendo que todas las poblaciones sobreviven; de ese modo las ecuaciones se pueden reescribir de la siguiente manera
$$\Lambda X^*=\mathrm{K},\qquad\text{con }\mathrm{K}=(K_1,...,K_n)$$
%Aqui hay un argumento fuerte para la estabilidad en función de la tupla.
entonces hallar el punto fijo es resolver el sistema $X^*=\Lambda^{-1}\mathrm{K}$. Gracias a que la diagonal esta fijada al valor $d=1$, mitiga la probabilidad de que sea singular y garantiza que sea invertible con alta probabilidad. Es de interés que el resultado del punto fijo sea positivo puesto que no tiene sentido físico considerar poblaciones negativas (que probablemente devengan en una dinámica inestable). El sistema hasta ahora no garantiza que $x_i^*>0$ para toda $x_i^*\in X^*$, puesto que aunque $\mathrm{K}>0$, las entradas de $\Lambda$ pueden ser positivas y negativas con una misma probabilidad. Si se considera a $\mathrm{K}$ con entradas constantes, es decir, $\mathrm{K}=k\vec{1}$ con $k>0$ entonces la distribución de $X^*$ puede darse a su alrededor con base en $(p,\sigma,N)$. \\
\\
Se proponen las siguientes suposiciones: Cuando $p\ll 1$ entonces $\Lambda\approx I$ y por lo tanto $\Lambda^{-1}\approx I$, en consecuencia $X^* \approx \mathrm{K}$. Sin embargo cuando $p$ llega a valores intermedios o grandes, existe mayor cantidad de interacciones que ensancha la distribución alrededor de $\mathrm{K}$ hacia valores negativos, haciendo que $X^*$ contenga entradas negativas. Analizando a $\sigma$ se tiene el caso trivial cuando $\sigma=0$ entonces $\Lambda=I$ y por tanto $X^*=K$. Cuando $\sigma\ll 1$ se generan pequeñas fluctuaciones y la distribución comienza a ensancharse alrededor de $\mathrm{K}$. Cuando $\sigma$ toma valores intermedios $0.3<\sigma<0.5$ entonces la distribución se ensancha de tal forma que comienzan a emerger términos negativos en las entradas de $X^*$. Finalmente cuando $\sigma>0.5$ la cantidad de entradas negativas se amplifica de modo que va disminuyendo la probabilidad de que existan puntos fijos con entradas positivas.\\
\\
Finalmente el papel de $N$ en la tupla que gobierna a $\Lambda$, será el de amplificar los efectos que produce la combinación de $p$ y $\sigma$, ya que al tener mayor cantidad de entradas: se amplifica la posibilidad de que $\Lambda^{-1}\mathrm{K}$ sea más dispersa. ¿Cuál será el efecto de $\Lambda^{-1}$ en la distribución del punto fijo? Esta matriz como tal no será una combinación lineal de normales, sino que dependerá del determinante de $\Lambda$ y sus valores propios\footnote{Considerando que el producto de sus valores propios es igual a su determinante ya que si $\lambda=0$ en el polinomio característico se tiene $p(0)=\det(A-0I)=\det(A)=\lambda_1\cdots\lambda_n$.}. Si por alguna razón $\det(\Lambda)=0$ entonces la matriz es singular y todas las entradas de $\Lambda^{-1}$ divergen a $\infty$. Si $\det(\Lambda)<1$ y además se sabe que 
\begin{equation}\label{eqn:determinante}
	\text{Adj}(\Lambda)\cdot \Lambda = \det(\Lambda)\cdot I
\end{equation}
se puede conjeturar que las entradas de Adj$(\Lambda)$ son comparables en magnitud con las entradas de $\Lambda$. Por lo tanto $\Lambda^{-1}=\frac{1}{\det(\Lambda)}\text{Adj}(\Lambda)$ tendrá entradas ligeramente amplificadas con respecto de $\Lambda$ debido a $\frac{1}{\det(\Lambda)}>1$. Cuando $\det(\Lambda)>1$ y $\Lambda$ mantiene las mismas condiciones que antes, entonces las entradas de su matriz adjunta deberán ser uno o varios ordenes de magnitud mayor que las entradas de $\Lambda$ para poder satisfacer el producto (\ref{eqn:determinante}). Dichas magnitudes irán en función de que tan grande es el el valor del determinante. En este caso la matriz inversa $\Lambda^{-1}$ tendrá entradas amplificadas debido a Adj$(\Lambda)$ pero con un efecto amortiguador proveniente de $\frac{1}{\det(\Lambda)}<1$.
\\
\\
Entonces en cualquiera de los dos casos mencionados, la matriz inversa $\Lambda^{-1}$ presenta entradas amplificadas con respecto de $\Lambda$ en función de la magnitud del determinante. Al realizar la multiplicación $\Lambda^{-1}\mathrm{K}$ se obtiene una distribución alrededor de $\mathrm{K}$ considerablemente más dispersa que si se tomara el producto $\Lambda\mathrm{K}$. Esta dispersión estará dada en función de $p$ y $\sigma$ haciendo que la distribución se quede en valores positivos, cuando estos parámetros son pequeños, o se disperse hacia valores negativos cuando dichos parámetros aumenten.\\
\\
Viendo como el determinante de $\Lambda$ afecta en $\Lambda^{-1}$ y en la distribución de $X^*$, queda preguntarse qué sucede cuando existe un número impar de valores propios negativos que propicie $\det(\Lambda)<0$, ¿devendrán en puntos fijos con entradas negativas? Más aún, ¿existirá una relación entre el espectro de valores propios de $\Lambda$ con la determinación de los puntos fijos? de modo que el signo de la parte real de los valores propios de $\Lambda$ sea determinante del carácter de las entradas del punto fijo. Estas son conjeturas que llegan al límite de esta tesis.
\newpage
\section{Jacobiana del sistema}

¿Existirán puntos fijos con entradas positivas que devengan en una dinámica inestable? En esta sección se explorará sobre la estabilidad en términos de la matriz Jacobiana. Anteriormente se ha comentado sobre la \textit{linealización} del sistema para conocer la estabilidad alrededor de un punto fijo. A estas alturas ya se cuenta con lo necesario para calcular la matriz Jacobiana explícita del sistema (\ref{eqn:LKmatricial}) para cualquier número de especies. El resultado deberá coincidir con una forma semejante a las interacciones que May definió en \cite{may2019stability}. Una vez hallado el punto fijo de interés, se podrá determinar si el sistema es estable o no con base en los valores propios de esta matriz Jacobiana, considerando diversas configuraciones de la tupla $(p,\sigma,N)$ que gobiernan a $\Lambda$.\\
\\
Con base en la definición \ref{def:LKVectorial} se define el sistema de Lotka-Volterra generalizado como $\dot{X}=\textbf{F}(X)$, siendo \textbf{F} la función vectorial no lineal del sistema. Para determinar su matriz Jacobiana se tiene que aplicar las derivadas parciales a cada $f_i(X)\in\mathbf{F}(X)$ y para ello separamos en dos casos: considerando elementos de la diagonal $\frac{\partial f_i(X)}{\partial x_i}$ y elementos fuera de la diagonal: $\frac{\partial f_i(X)}{\partial x_k}$. Para comenzar el cálculo considere las ecuaciones del sistema (\ref{eqn:LK}) 
$$
f_i(X) = r_ix_i\left (1-\frac{\sum_{j=1}^N \alpha_{ij}x_j}{K_i}\right )
$$
las derivadas parciales para los términos de la diagonal son
\begin{equation}\label{eqn:EcsJacobiano_ii}
	\begin{split}
			\frac{\partial f_i(X)}{\partial x_i} &= r_i \left (1-\frac{2\alpha_{ii}x_i+\sum_{j\neq i}\alpha_{ij}x_j}{K_i}\right )
	\end{split}
\end{equation}
Sin embargo, hay que tener en cuenta que se evalúa sobre un punto fijo $X^*$ el cual se genera de la siguiente ecuación  
$$\alpha_{ii}x_i+\sum_{j\neq i}\alpha_{ij}x_j=K_i$$
Sustituyendo $\sum_{j\neq i}\alpha_{ij}x_j=K_i-\alpha_{ii}x_i$ en (\ref{eqn:EcsJacobiano_ii}) finalmente se tiene
$$\frac{\partial f_i(X^*)}{\partial x_i} = -\frac{r_ix_i^*}{K_i}\alpha_{ii}$$
Para los términos que quedan fuera de la diagonal evaluados en $X^*$ se tiene
$$\frac{\partial f_i(X^*)}{\partial x_k} = -\frac{r_ix_i^*}{K_i}\alpha_{ik} $$
\newpage
\begin{definición}\label{def:MatrizJacobiana}
	Sea $\mathcal{J}\in\mathrm{M}_N(\mathbb{R})$ donde $N$ es el número de especies del sistema LV generalizado. Se define la matriz \textit{Jacobiana} asociada al sistema (\ref{eqn:Fmatricial}) evaluado en un punto fijo $X^*$ de la siguiente forma
	\begin{equation}\label{eqn:MatrizJacobiana}
		\mathcal{J}_{ij}=\begin{cases}
			-\frac{r_ix_i^*}{K_i}\alpha_{ii},\qquad&\text{para }i=j\\
			-\frac{r_ix_i^*}{K_i}\alpha_{ij},\qquad&\text{para }i\neq j
		\end{cases}
	\end{equation}
	considerando que $\alpha_{ii}>0$ y particularmente se ha fijado en $\alpha_{ii}=1$ para todo $i\in\{1,...,N\}$. Si se considera una matriz diagonal cuyos elementos son $-\frac{r_ix_i^*}{K_i}$, entonces la Jacobiana se puede reescribir de la siguiente manera
	\begin{equation}\label{eqn:Jacobiana}
		\mathcal{J}=-\text{diag}\left (\frac{r_ix_i^*}{K_i}\right )\Lambda
	\end{equation}
\end{definición}
De la matriz Jacobiana se puede observar como todas sus entradas voltean el signo con respecto de $\Lambda$, lo que parece ajustarse con las interacciones que May define en su \textit{community matrix} \cite{may2019stability}, incluso obedece el signo negativo en la diagonal que será sumamente importante para determinar la estabilidad del sistema. La ec. (\ref{eqn:Jacobiana}) deja ver que la matriz Jacobiana ¡es un re-escalamiento de la matriz de Incidencias! indicando que la estabilidad únicamente depende de $\Lambda$\footnote{Ya que son las interacciones de $\Lambda$ las que inducen la naturaleza del punto fijo.}. ¿Como se podría verificar la estabilidad del sistema con base en el punto fijo y los valores propios de $\mathcal{J}$?

\begin{proposición}\label{prop:DiagonalI}
	Los elementos de la diagonal de la matriz Jacobiana determinan en gran medida la estabilidad del sistema
	\begin{proof}
		Para demostrar esta proposición se hará uso del \textit{Teorema de Gershgorin} \cite{GershgorinTheorem}. Para ello se asume que la matriz Jacobiana tendrá valores propios complejos. Se define el radio de Gershgorin como
		\begin{equation}\label{proof1:RGershgorin}
			R_i=\sum_{i\neq j}|\mathcal{J}_{ij}|
		\end{equation}
		con ello se define $D(\mathcal{J}_{ii},R_i)\subset\mathbb{C}$ como el disco de Gershgorin centrado en $\mathcal{J}_{ii}$ con radio $R_i$. El teorema establece que cada valor propio de $\mathcal{J}$ estará contenido en alguno de estos discos. Demostrando que si todo $D(\mathcal{J}_{ii},R_i)$ se encuentra contenido en el semiplano negativo de $\mathbb{C}$, entonces todos los valores propios de la Jacobiana serán negativos también y por lo tanto el sistema (\ref{eqn:LK}) será estable en $X^*$. El elemento más importante de esta prueba es considerar el centro de los discos; si todos los $\mathcal{J}_{ii}$ son negativos entonces el centro de todos los discos se encuentra en el semiplano negativo de $\mathbb{C}$, y para garantizar que todos los valores propios sean negativos habría que probar $|\mathcal{J}_{ii}|>R_i$ para todo $i\in\{1,...,N\}$, es decir:
		\begin{equation}\label{proof1:Desigualdad}
			\begin{split}
					 \left |-\frac{r_ix_i^*}{K_i}\alpha_{ii}\right |&>\sum_{j\neq i}\left |-\frac{r_ix_i^*}{K_i}\alpha_{ij}\right |\\
				1&>\sum_{j\neq i}|\alpha_{ij}|
			\end{split}
		\end{equation}
		\newpage
		De la ec. (\ref{eqn:MatrizJacobiana}) se puede ver que $\alpha_{ii}>0$ y en particular $\alpha_{ii}=1$, por lo tanto es importante que todo $x_i^*\in X^*$ sea positivo para que el centro de los discos se encuentre en el semiplano negativo de $\mathbb{C}$. En caso contrario habrá al menos un disco con centro en el sempiplano positivo de $\mathbb{C}$ que pueda alojar valores propios positivos que devengan en una dinámica inestable. Es conveniente poder saber cual es el tamaño promedio de los discos de Gershgorin y para ello se define la siguiente variable aleatoria
		\begin{equation}\label{proof1:VarAlW}
			W = 
			\begin{cases}
				0,\qquad \text{Si }1-p\\
				Y,\qquad \text{Si }p>0	
			\end{cases}
		\end{equation}
		donde $Y\sim N(0,\sigma^2)$ es una variable aleatoria normal. Ya que $R_i$ considera suma de valores absolutos, es de interés saber el valor esperado de $|W|$ 
		\begin{equation}\label{proof1:ValorEsperadoW}
			\mathbb{E}[|W|]=(1-p)\cdot 0 + p\cdot \mathbb{E}[|Y|] = p\cdot\mathbb{E}[|Y|] = p\sigma\sqrt{\frac{2}{\pi}}
		\end{equation}
		el término $\sqrt{\frac{2}{\pi}}$ viene de considerar la variable aleatoria $Z\sim N(0,1)$ y la distribución \textit{half-normal} $|Z|$, entonces su valor esperado es $$\mathbb{E}[|Z|]=\frac{1}{\sqrt{2\pi}}\int_{-\infty}^{\infty}|z|e^{-z^2/2}\, dz=\frac{2}{\sqrt{2\pi}}\int_0^\infty ze^{-z^2/2}\, dz=\sqrt{\frac{2}{\pi}}$$
		al considerar la variable aleatoria $Y\sim N(0,\sigma^2)$ entonces basta con escalar $Y=\sigma Z$ y en consecuencia su valor esperado será $$\mathbb{E}[|Y|]=\sigma \mathbb{E}[|Z|]=\sigma\sqrt{\frac{2}{\pi}}$$
		Por lo tanto, el valor esperado del radio de Gershgorin es
		\begin{equation}\label{proof1:E[RadioG]}
			\mathbb{E}[R_i]=(N-1)p\sigma\sqrt{\frac{2}{\pi}}
		\end{equation}
		Si este radio promedio es menor a 1 se puede garantizar en cierta medida que los discos de Gershgorin estarán contenidos en el semiplano negativo de $\mathbb{C}$. Sin embargo, pueden existir discos de Gershgorin que se alejen de la media ¿qué tanto se pueden alejar para que se siga cumpliendo $\mathcal{J}_{ii}>R_i$? será conveniente calcular la varianza de los radios de Gershgorin para averiguarlo. Tomando la variable aleatoria $W$, se calcula su segundo momento
		$$\mathbb{E}(|W|^2)=(1-p)\cdot 0+p\cdot\mathbb{E}(|Y|^2)=p\cdot\mathbb{E}(Y^2)=p\sigma^2$$
		por otro lado el cuadrado del valor esperado (\ref{proof1:ValorEsperadoW}) es
		$$\mathbb{E}(|W|)^2=p^2\sigma^2\frac{2}{\pi}$$
		Finalmente la varianza de $|W|$ es
		$$\Var(|W|)=p\sigma^2-p^2\sigma^2\frac{2}{\pi}$$
		Entonces la varianza de los radios de Gershgorin queda envuelta en la siguiente expresión
		\begin{equation}\label{proof1:VarRadioG}
			\Var(R_i)=(N-1)p\sigma^2\left (1-\frac{2p}{\pi}\right )
		\end{equation}
		y mientras sea menor a 1 se puede garantizar que los discos de Gershgorin van a cumplir (\ref{proof1:Desigualdad}) y en consecuencia los valores propios de $\mathcal{J}$ estarán contenidos en el semiplano negativo de $\mathbb{C}$. Sin embargo esta prueba tiene algunas limitantes; los discos de Gershgorin únicamente garantizan que contienen valores propios, forzar la condición (\ref{proof1:Desigualdad}) es suficiente para garantizar estabilidad pero no es determinante, ya que este método no menciona nada acerca de la posición de los valores propios. Si dicha desigualdad no se cumple pueden ocurrir dos escenarios: que $\max(|\lambda_\mathcal{J}|)<0$ y el sistema siga siendo estable a pesar de algún $R_i>\mathcal{J}_{ii}$ ó $\max(|\lambda_\mathcal{J}|)>0$ en cuyo caso el sistema será inestable, y eso dependerá de $\Lambda$ y la tupla $(p,\sigma,N)$. En conclusión, esta prueba funciona en un primer nivel para mostrar que los elementos de la diagonal de $\mathcal{J}$ y por consiguiente las entradas de $X^*$ juegan un papel importante en la estabilidad del sistema. 
	\end{proof}
\end{proposición}

Es importante recordar que las interacciones de auto-regulación se mantienen en la diagonal de $\Lambda$ (Ver ec. (\ref{eqn:LKextendida})); cuando la matriz de incidencias se transforma en la Jacobiana, el punto fijo debe de contener entradas positivas para poder respetar las interacciones de auto-regulación que May define en \cite{may1972will}. Si el punto fijo contiene entradas negativas es casi un hecho que el sistema será inestable porque existirá al menos un disco de Gershgorin con centro en el semiplano positivo de $\mathbb{C}$ que independientemente de como sea su radio, es altamente probable que contenga valores propios positivos.\\
\\
Para poder determinar un parámetro crítico de transición en la estabilidad de los sistemas, es imprescindible conocer donde se ubica el $\max(\lambda_\mathcal{J})$ y encontrar la forma de controlar su posición para poder saber las condiciones que lo llevan a ser negativo, cero o positivo. Aunque existe una forma dada para resolver esta conjetura, la cual se abordará en la siguiente sección, tiene un gran inconveniente: la Jacobiana tiene en su diagonal toda una distribución de valores que dificulta la definición de un radio espectral que determine la posición de Re$(\lambda_{\max}(\mathcal{J}))$. Esta conjetura ligada con la correspondiente de la sección anterior también llega al límite de esta tesis y más adelante se presentará un análisis detallando sus razones.
\newpage
