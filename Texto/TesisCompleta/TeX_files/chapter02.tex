\chapter{El modelo de Lotka-Volterra generalizado}

%En este capítulo solo se define y se construye el modelo, se deja listo para que en el siguiente capítulo pueda ser analizado.
\setlength{\parindent}{0cm} En este capítulo se introduce el modelo de Lotka-Volterra generalizado que será utilizado a lo largo de esta tesis. Se define el sistema dinámico, se especifican sus parámetros y la estructura de las interacciones, y posteriormente se analizan los puntos de equilibrio del sistema y su matriz Jacobiana.
\begin{equation}\label{eqn:LK}
	\frac{dx_i}{dt}=r_ix_i\left(1-\frac{\sum_{j=1}^N \alpha_{ij}x_j}{K_i}\right),\qquad i=1,...,N
\end{equation}
En el sistema \eqref{eqn:LK}, $x_i(t)$ representa la abundancia de la especie $i$, $r_i$ denota su tasa intrínseca de crecimiento y $K_i$ su capacidad de carga, heredados del modelo logístico \eqref{eqn:EqLogistica}. Los coeficientes $\alpha_{ij}$ codifican la intensidad y el signo de las interacciones entre las especies $i$ y $j$; dependiendo de su signo, estas interacciones pueden corresponder a escenarios de competencia, cooperación o a un régimen mixto generalizado.\\
\\
A continuación, se analiza el comportamiento del sistema en el caso particular de dos especies, con el fin de fijar intuiciones antes de abordar el régimen de alta dimensión. En este último caso, debido al carácter no lineal y la alta dimensionalidad del sistema, el análisis se apoya en herramientas computacionales.

\section{Caso particular para $N=2$}

En el caso particular $N=2$, el sistema \eqref{eqn:LK} se reduce a un par de ecuaciones diferenciales que conservan la misma estructura funcional y parámetros del modelo general,
\begin{align*}
	\frac{dx_1}{dt}&=r_1 x_1\left(1-\frac{\alpha_{11}x_1}{K_1}-\frac{\alpha_{12}x_2}{K_1}\right),\\
	\frac{dx_2}{dt}&=r_2 x_2\left(1-\frac{\alpha_{21}x_1}{K_2}-\frac{\alpha_{22}x_2}{K_2}\right).
\end{align*}

Dada la estructura del sistema, los coeficientes de interacción entre especies pueden organizarse en una \textbf{matriz de coeficientes de interacción $\Lambda$}, cuyos elementos $\alpha_{ij}$ cuantifican la relación entre las especies $j$ con $i$.
\begin{equation}\label{eqn:mIncidencias}
	\Lambda=
	\begin{pmatrix}
		\alpha_{11} & \alpha_{12}\\
		\alpha_{21} & \alpha_{22}
	\end{pmatrix}
\end{equation}
En esta representación, los coeficientes de la diagonal se fijan a un valor positivo, en particular $\alpha_{ii}=1$, lo que induce un término de autorregulación para cada especie. Este término actúa como un mecanismo disipativo que limita el crecimiento no acotado del sistema y garantiza la existencia de una escala de saturación en las abundancias. Cabe destacar que esta matriz no corresponde con la Jacobiana el sistema linealiado alrededor de un estado estacionario, sino que representa los coeficientes de interacción que aparecen directamente en el modelo no lineal. En secciones posteriores se introducirá la matriz Jacobiana, la cual se construye a partir de $\Lambda$ pero posee una interpretación dinámica diferente.

\begin{ejemplo}\label{eg:2x2}
	A continuación se considera un sistema de Lotka-Volterra para $N=2$ especies, con coeficientes de interacción constantes y positivos, de modo que el sistema muestre un escenario competitivo.
	\begin{equation}\label{eqn:Sist2x2Comp}
		\begin{split}
			\frac{dx}{dt}&=2x\left(1-\frac{x}{2}\right)-xy\\
			\frac{dy}{dt}&=3y\left(1-\frac{y}{3}\right)-2xy
		\end{split}
	\end{equation}
	Para analizar la dinámica del sistema se procede a la determinación de sus puntos de equilibrio, definidos como aquellos estados del sistema donde su dinámica se anula. Esto se logra igualando a cero y resolviendo las ecuaciones del sistema \eqref{eqn:Sist2x2Comp}. El equilibrio trivial corresponde con el estado $\vec{0}$, el cual representa la extinción de ambas especies. Para los casos no triviales, la resolución del sistema algebráico conduce a los puntos de equilibrio $(2,0)$, $(0,3)$ y $(1,1)$.\\
	\\
	Como se puede apreciar, el sistema (\ref{eqn:Sist2x2Comp}) presenta múltiples puntos de equilibrio, en contraste con los sistemas lineales, que admiten a lo sumo uno. Este hecho es relevante, ya que en sistemas no lineales de mayor dimensión la determinación explícita de los puntos de equilibrio se vuelve cada vez más compleja. En el caso presente, se conocen los puntos de equilibrio y por tanto es posible estudiar su estabilidad local mediante la evaluación de la matriz Jacobiana del sistema en dichos estados.\\
	\\
	Al evaluar el Jacobiano en el estado coexistente se obtiene 
	\begin{equation}\label{eqn:Jacobiano(1,1)}
		\mathbb{J}_{(1,1)}=\begin{pmatrix}
			-1 & -1\\
			-2 & -1
		\end{pmatrix}
	\end{equation}
	Los valores propios asociados a esta matriz son
	$$\lambda_{1}=-1+\sqrt{2}>0,\qquad\lambda_2=-1-\sqrt{2}<0$$
\end{ejemplo}

\begin{wrapfigure}{r}{0.5 \textwidth} \vspace{-30pt} \begin{center}
		\includegraphics[width=0.5\textwidth]{../Imagenes/Competencia de especies} 
	\end{center} 
	\vspace{-20pt} 
	\caption{Campo vectorial de las soluciones del sistema (\ref{eqn:Sist2x2Comp}) de dos especies.} 
	\vspace{10pt}
	\label{fig:CompetenciaEspecies}
\end{wrapfigure} 
Lo que caracteriza al punto de equilibrio $(1,1)$ como un \textbf{equilibrio tipo silla}, y por tanto localmente inestable. En consecuencia, las trayectorias se alejan de este punto y la dinámica converge hacia los atractores del sistema. Esta estructura puede apreciarse en la representación del espacio de fases del sistema (Figura (\ref{fig:CompetenciaEspecies})), donde se observan las regiones de atracción asociadas a los estados $(2,0)$ o $(0,3)$. Este procedimiento, aunque no admite una generalización analítica directa a sistemas de gran dimensión, puede extenderse de manera sistemática y será desarrollado progresivamente en las secciones siguientes. Como se ha señalado, el análisis de estabilidad se basa en el proceso de linealización local del sistema no lineal alrededor de un punto de equilibrio. La matriz Jacobiana constituye la representación del sistema linealizado, por lo que resulta conveniente introducir a continuación su forma general.
\begin{definición}\label{def:LKVectorial}
	Sea $\mathbf{F}:\mathbb{R}^N\to\mathbb{R}^N$ el campo vectorial no lineal asociado al sistema de Lotka-Volterra generalizado, cuyas componentes están definidas por 
	\begin{equation}\label{eqn:Fmatricial}
		\mathbf{F}(X)=\begin{pmatrix}
			f_1(X)=r_1x_1\left(1-\frac{\sum_{j=1}^N \alpha_{1j}x_j}{K_1}\right)\\
			\vdots\\
			f_N(X)=r_Nx_N\left(1-\frac{\sum_{j=1}^N \alpha_{Nj}x_j}{K_N}\right)
		\end{pmatrix},
	\end{equation}
	donde $X(t)=\left(x_1(t),\dots,x_N(t))\right)\in\mathbb{R}^N$. Sea $X^*$ un punto de equilibrio del sistema, es decir, cumple $\mathbf{F}(X^*)=\mathbf{0}$. La matriz Jacobiana asociada al sistema evaluada en $X^*$ esta dada por
	\begin{equation}\label{eqn:Jacobiano}
		\mathbb{J}_\mathbf{F}(X^*) = \begin{pmatrix}
			\frac{\partial f_1(X^*)}{\partial x_1} & \cdots &\frac{\partial f_1(X^*)}{\partial x_N}\\
			\vdots & \ddots & \vdots\\
			\frac{\partial f_N(X^*)}{\partial x_1} & \cdots &\frac{\partial f_N(X^*)}{\partial x_N}
		\end{pmatrix}
	\end{equation}
\end{definición}

Con esta formulación queda establecido el marco general para el análisis de la dinámica y la estabilidad del sistema de Lotka-Votlerra generalizado. En lo que sigue, el estudio se centrará en la estructura de la matriz Jacobiana con base en las interacciones entre especies. En particular, será necesario especificar cómo se modela la conectividad del sistema y la distribución de las intensidades de interacción, aspectos que resultan clave para abordar el régimen de gran dimensión $N\gg1$.

\section{Generalizando a $N\gg 1$ especies.}

En esta sección se establece la estructura de las interacciones del sistema Lotka-Votlerra generalizado mediante la \textbf{matriz de coeficientes de interacción}. Se introducirán los parámetros que gobiernan su construcción, con el propósito de especificar un régimen estructural del sistema y formular el esquema de simulaciones que permitirán analizar los objetivos planteados.
\\
\\
Para modelar la estructura de las interacciones se recurre al formalismo de \textit{redes}. En este contexto, el sistema puede representarse como una red dirigida ponderada $G=(V,E)$, donde los \textit{nodos} $V$ corresponden a las especies y los enlaces $E$ describen las interacciones entre ellas. A modo ilustrativo, en la Figura (\ref{fig:RedKarate}) se muestra la red del club de karate estudiada por Wayne Zachary (1977), un ejemplo clásico en teoría de redes en donde los nodos representan individuos y los enlaces sus relaciones sociales.
\begin{wrapfigure}{l}{0.45 \textwidth} \vspace{-30pt} \begin{center}
		\includesvg[width=.5\textwidth]{../Imagenes/karateLu} 
	\end{center} 
	\vspace{-20pt} 
	\caption{Red de Karate de Zachary} 
	\vspace{10pt}
	\label{fig:RedKarate}
\end{wrapfigure} 
No obstante, la estructura de las redes puede codificarse algebraicamente mediante su \textit{matriz de adyacencia}, la cual describe la conectividad del sistema independientemente de la intensidad de las interacciones que se perciben. En el modelo de Lotka-Volterra generalizado, los coeficientes $\alpha_{ij}$ cuantifican la influencia entre la especie $j$ sobre la especie $i$. En particular, si $\alpha_{ij}=0$, no existe interacción directa entre especies. De esta manera, los coeficientes de interacción $\alpha_{ij}$ pueden interpretarse como los pesos asociados a los enlaces de una red dirigida. En consecuencia, una red ponderada cuya matriz de pesos este dada por $(\alpha_{ij})$ es algebraicamente equivalente a la matriz de coeficientes de interacción del sistema, la cual se formaliza en esta sección.


\begin{definición}
	Una \textbf{red dirigida} se define como un par ordenado $G=(V,E)$, donde $V=\{1,\dots,n\}$ es el conjunto de nodos y $E\subseteq V\times V$ es el conjunto de enlaces ordenados. El par $(i,j)\in E$ representa una interacción dirigida desde el nodo $i$ hasta el nodo $j$. A esta red se le asocia su \textbf{matriz de adyacencia} dirigida $\mathcal{D}\in\mathrm{M}_n(\mathbb{R})$, cuyos elementos se definen como
	$$
	\mathcal{D}_{ij}=\begin{cases}
		1,\quad\text{si } (i,j)\in E\\
		0,\quad\text{en otro caso.}
	\end{cases}
	$$
	En el contexto ecológico, este tipo de estructura es natural, ya que las interacciones entre especies poseen una direccionalidad inherente: la influencia de la especie $i$ sobre $j$ no necesariamente coincide en el sentido contrario. Este es precisamente el tipo de estructura considerada en el análisis de estabilidad de comunidades complejas desarrollado por Robert May. Obsérvese que, en el modelo de Lotka-Volterra generalizado, la matriz de coeficientes $(\alpha_{ij})$ puede interpretarse como una matriz de pesos asociada a esta estructura dirigida, de modo que la conectividad y la intensidad de las interacciones quedan codificadas simultáneamente.
\end{definición}

\begin{ejemplo}
	Considérese una red dirigida con $n=10$ nodos y 14 enlaces.
\end{ejemplo}

\begin{wrapfigure}{l}{0.5 \textwidth} \vspace{-44pt} \begin{center}
		\includesvg[width=0.52\textwidth]{../Imagenes/Red10DirLu} 
	\end{center} 
	\vspace{-20pt} 
	\caption{Red dirigida de 10 nodos.} 
	\vspace{-150pt}
	\label{fig:Red10Dir}
\end{wrapfigure} 

$$	\mathcal{D} = \begin{pmatrix}
	0 & 0 & 0 & 0 & 0 & 0 & 0 & 0 & 0 & 1 \\
	1 & 0 & 1 & 0 & 0 & 0 & 0 & 1 & 0 & 0 \\
	0 & 0 & 0 & 1 & 1 & 0 & 0 & 0 & 0 & 1 \\
	1 & 0 & 0 & 0 & 0 & 0 & 0 & 0 & 0 & 0 \\
	0 & 0 & 1 & 0 & 0 & 0 & 0 & 1 & 0 & 0 \\
	0 & 1 & 0 & 0 & 0 & 0 & 0 & 0 & 0 & 0 \\
	0 & 0 & 1 & 0 & 0 & 0 & 0 & 0 & 0 & 0 \\
	0 & 0 & 0 & 0 & 0 & 0 & 0 & 0 & 0 & 0 \\
	0 & 1 & 0 & 0 & 0 & 0 & 0 & 0 & 0 & 0 \\
	0 & 0 & 0 & 0 & 0 & 0 & 1 & 0 & 0 & 0 \\
\end{pmatrix}$$
\newline
Su matriz de adyacencia $\mathcal{D}$ no es simétrica como lo sería en un contexto de \textit{red no dirigida}, los enlaces presentan una dirección preferencial entre nodos. Esta representación codifica únicamente la existencia de enlaces $(\mathcal{D}_{ij}\in\{0,1\})$, sin incorporar información sobre su intensidad. En este ejemplo se tiene $\mathcal{D}_{ii}=0$, es decir, no se consideran autoenlaces. Sin embargo, en el modelo dinámico subyacente los términos de la diagonal $\Lambda_{ii}$ desempeñan un papel fundamental al representar los mecanismos de regulación, los cuales serán incorporados explícitamente en la construcción de la martiz de coeficientes de interacción. Las \textbf{redes de interacción ponderadas} consideradas en esta tesis se construyen a partir de una matriz de adyacencia que podrá ser simétrica o dirigidaX, determinando así la estructura de conectividad del sistema.

\subsection{Red de interacciones ponderada}

La red de interacciones ponderada constituye el objeto matemático que codifica la estructura de interacciones del modelo de Lotka-Volterra generalizado. Su representación matricial corresponde a la matriz de coeficientes de interacción, la cual determina tanto la conectividad de las especies como la intensidad de dichas interacciones. En general, esta red se construye a partir de un soporte estructural que puede ser simétrico o dirigido, junto con un conjunto de pesos que no necesariamente satisfacen simetría. \\
\\
Su formulación depende de un conjunto de parámetros estructurales que serán especificados más adelante. Para modelar la estructura de conectividad se emplea la \textit{red aleatoria} de Erdös–Rényi, en la cual cada posible enlace entre pares de nodos se establece de manera independiente con probabilidad $p$. Por otra parte, la intensidad de las interacciones se modela mediante variables aleatorias independientes e idénticamente distribuidas con media cero y varianza finita. \\
\\
Como resultado, la red de interacciones ponderada considerada en este trabajo corresponde a una \textbf{red aleatoria ponderada}, donde la topología y los pesos se introducen de manera independiente. Este enfoque puede extenderse a otras topologías, como redes libres de escala o de mundo pequeño, aunque tales generalizaciones no se contemplan en este trabajo.

\begin{definición}\label{def:redAleatoria}
	Sea un conjunto de $N$ nodos. Una red aleatoria \textbf{no dirigida} $\mathcal{G}_\mathcal{E}$ se define por su matriz de adyacencia  $\mathcal{E}\in\mathrm{M}_N(\{0,1\})$ tal que
	$$
	\mathcal{E}_{ij}=\begin{cases}
		X_{ij},\quad &i<j,\\
		X_{ji},\quad &i>j,\\
		0,\quad &i=j
	\end{cases},\qquad X_{ij}\sim \text{Bernoulli}(p)
	$$
	donde las variables $X_{ij}$ son independientes. En consecuencia, $\mathcal{E}$ es simétrica y cada enlace aparece con probabilidad $p$. En contraste, su versión \textbf{dirigida} contempla
	$$
	\mathcal{E}_{ij}=\begin{cases}
		X_{ij},\quad &i\neq j,\\
		0,\quad &i=j
	\end{cases},\qquad X_{ij}\sim\text{Bernoulli}(p)
	$$
	con $X_{ij}$ independientes, por lo que $\mathcal{E}$ no es necesariamente simétrica y existen $N(N-1)$ enlaces posibles.
\end{definición}

Para completar la construcción de la estructura de las interacciones, resta incorporar la magnitud de los enlaces. Para ello se introduce una matriz aleatoria $\mathcal{M}\in \mathrm{M}_N(\mathbb{R})$, cuyas entradas se suponen independientes e idénticamente distribuidas con $\mathcal{M}_{ij}\sim\mathcal{N}(0,\sigma^2)$. Esta matriz representará la intensidad de las interacciones y, combinada con la matriz de adyacencia previamente definida, permitirá construir la matriz de coeficientes de interacción $\Lambda$.
\begin{definición}\label{def:MatrizIncidencias}
	Sea una red aleatoria $\mathcal{G}_\mathcal{E}$ dirigida o no dirigida con matriz de adyacencia $\mathcal{E}\in\mathrm{M}_N(\{0,1\})$. Sea una matriz $\mathcal{M}\in\mathrm{M}_N(\mathbb{R})$ con entradas i.i.d. con media cero y varianza finita $\mathcal{M}_{ij}\sim\mathcal{N}(0,\sigma^2)$. La \textbf{matriz de coeficientes de interacción} $\Lambda$ se define como el producto de Hadamard \cite{HadamardProduct} de la matriz de adyacencia con la matriz de entradas aleatorias sumada con la matriz identidad
	\begin{equation}\label{eqn:MatrizIncidencias}
		\Lambda=(\mathcal{E}\odot \mathcal{M}) + I
	\end{equation}
	El producto de Hadamard restringe los coeficientes aleatorios a los enlaces de la red, mientras que la suma con la identidad introduce términos de auto-interacción que funcionan como mecanismo de autoregulación. Se dirá que la matriz $\Lambda$ es \textbf{estructuralmente simétrica} si su matriz de adyacencia $\mathcal{E}$ es simétrica, es decir, si existe interacción entre $i$ con $j$ en ambos sentidos $(\mathcal{E}_{ij}=\mathcal{E}_{ji})$, aunque las intensidades correspondientes puedan diferir ($\Lambda_{ij}\neq\Lambda_{ji}$). Por el contrario, si $\mathcal{E}$ tiene un soporte dirigido entonces se dirá que $\Lambda$ es \textbf{estructuralmente no simétrica}. Esta distinción permite separar y distinguir entre la simetría del soporte topológico y la simetría dinámica de las interacciones. 
	\newline
\end{definición}

Por lo tanto, las matrices de coeficientes de interacción quedan parametrizadas por la probabilidad de conectividad $p$, la intensidad típica de las interacciones $\sigma$ y el tamaño del sistema $N$. En la Figura (\ref{fig:RedIncidencias}) se muestra una representación visual de una red de interacciones ponderada de 8 nodos, donde se ilustran tanto la estructura de conectividad como los pesos asociados. A continuación, se procede a caracterizar los distintos tipos de interacciones que pueden presentarse en este sistema dinámico.

\begin{wrapfigure}{r}{0.51 \textwidth} \vspace{-70pt} \begin{center}
		\includesvg[width=0.52\textwidth]{../Imagenes/RedIncidencias} 
	\end{center} 
	\vspace{-52pt} 
	\caption{Red de interacciones ponderada de 8 nodos con topología de red aleatoria dirigida $p=0.15$ y una matriz aleatoria con $\mu=0$ y $\sigma=0.2$.} 
	\vspace{-10pt}
	\label{fig:RedIncidencias}
\end{wrapfigure} 


\subsection{Tipos de interacciones}

La red de interacciones ponderada induce, en general, una estructura dirigida en la dinámica independientemente del soporte estructural que posea. Esto implica que las interacciones entre especies no necesariamente están balanceadas y pueden existir influencias preferenciales entre especies. Esta perspectiva resulta natural en contextos ecológicos, donde las interacciones suelen presentar una direccionalidad efectiva. En esta sección se explorarán los distintos tipos de interacciones que pueden surgir de sistemas de Lotka-Volterra generalizados bajo contextos estructuralmente simétricos y con interacciones aleatorias.\\
\\
Hasta ahora solo se han discutido interacciones de competencia en el Ejemplo \ref{eg:2x2}, donde los coeficientes $\alpha_{ij}\in\Lambda$ se consideran positivos. Bajo esta hipótesis, el crecimiento de las especies se ve limitado por sus respectivas capacidades de carga, lo que induce una dinámica acotada. Resulta entonces natural preguntarse como cambia el comportamiento del sistema cuando se permiten coeficientes $\alpha_{ij}$ negativos, los cuales representan distintos tipos de interacción biológica. Siguiendo la clasificación clásica de Robert May \cite{may2019stability}, pueden distinguirse cinco tipos fundamentales de interacciones. \\
\\
Estas interacciones se clasifican en función de los signos de las matrices de comunidad (\textit{community matrices}), introducidas previamente en el capítulo 1. En consecuencia, no se definen directamente sobre la matriz de coeficientes de interacción $\Lambda$, sino sobre la Jacobiana del sistema evaluada en un estado estacionario. En sistemas estructuralmente simétricos, las posibles interacciones son: competencia $(--)$, cooperación\footnote{También conocida como mutualismo o simbiosis} $(++)$ y presa-depredador $(-+)$ ó $(+-)$. Cuando se consideran sistemas sin esta restricción estructural, pueden presentarse además $(+0)$ y amensalismo $(-0)$, dando lugar a cinco tipos de interacción en total, en contraste con los tres del caso estructuralmente simétrico.
\\
\\
La cooperación implica que ambas especies se benefician de la interacción $(++)$, mientras que en la competencia ambas resultan perjudicadas $(--)$. En la interacción presa-depredador, una especie se beneficia y la otra se ve perjudicada $(+-)$ ó $(-+)$. En el comensalismo, una especie obtiene un beneficio sin afectar a la otra $(+0)$, mientras que en el amensalismo una especie resulta perjudicada y la otra permanece sin efecto apreciable $(-0)$. 
\newline
\begin{table}[h!]
	\centering
	\begin{tabular}{lll}
		\toprule
		\rowcolor{gray!15}
		Tipo de interacción & Signo & Interpretación ecológica \\
		\midrule
		Cooperación        & $(++)$ & Ambas especies aumentan sus poblaciones.\\
		Competencia        & $(--)$ & Ambas especies limitan o disminuyen sus poblaciones.\\
		Presa--depredador  & $(+-)$ ó $(-+)$ & Una especie aumenta su población y la otra la disminuye.\\
		Comensalismo       & $(+0)$ & Una especie aumenta su población sin afectar a la otra. \\
		Amensalismo        & $(-0)$ & Una especie disminuye su población sin afectar a la otra. \\
		\bottomrule
	\end{tabular}
	\caption{Tipos de interacciones posibles en sistemas de Lotka--Volterra generalizados.}
	\label{tab:Interacciones}
\end{table}


\subsubsection*{Interacciones en $\Lambda$}

Las interacciones en la matriz de coeficientes $\Lambda$ se relacionan con la clasificación de May (Tabla \ref{tab:Interacciones}), pero bajo una convención de signo opuesta. En esta formulación, la cooperación se representa como $(--)$, la competencia como $(++)$, la depredación como $(-+)$ ó $(+-)$, y el comensalismo y amensalismo como $(-0)$ y $(+0)$ respectivamente. Esta inversión se debe a que los coeficientes $\alpha_{ij}$ aparecen dentro del término de limitación del crecimiento; por tanto, valores positivos incrementan la presión limitante sobre la especie $i$, mientras que valores negativos la reducen. Para analizar cómo éstos coeficientes afectan la dinámica, reescribimos las ecuaciones \eqref{eqn:LK} separando la contribución de autoregulación de las interacciones interespecíficas:

\begin{equation}\label{eqn:LKextendida}
r_ix_i\left (1-\frac{\sum_{j=1}^N\alpha_{ij}x_j}{K_i}\right ) = r_ix_i\left (1-\frac{\alpha_{ii}x_i}{K_i}\right )-r_ix_i\left (\frac{\sum_{j\neq i}\alpha_{ij}x_j}{K_i}\right )
\end{equation}

El segundo término correspondiente a las interacciones interespecíficas, produce un efecto neto sobre la tasa de crecimiento de la especie $i$: sera positivo si $\alpha_{ij}<0$ y negativo si $\alpha_{ij}>0$. En consecuencia, desde la perspectiva de la especie $i$, las interacciones de cooperación, depredación (cuando $i$ es el depredador) y comensalismo favorecen su crecimiento poblacional, mientras que  la competencia, la depredación  (cuando $i$ es la presa) y amensalismo lo reducen. Nótese que si $\alpha_{ij}=0$, la especie $j$ no genera impacto sobre la dinámica de $i$.\\
\\
En este contexto, los coeficientes $\alpha_{ii}$ deben ser positivos; por convención, en este trabajo se fija $\alpha_{ii}=1$. Si se tomara $\alpha_{ii}\leq0$, se perdería el mecanismo intrínseco de regulación logística asociada a la especie, eliminando la auto-limitación de su crecimiento. Por esta razón, la contribución de la identidad en $\Lambda$ representa auto-interacciones que inducen un mecanismo de autoregulación del sistema. Por otro lado, coeficientes $\alpha_{ij}<0$ pueden favorecer el crecimiento poblacional de la especie $i$, desplazando el equilibrio efectivo por encima de su capacidad de carga $K_i$, fenómeno que no se aprecia en sistemas competitivos.
\\
\\
%Explorar el balance de las interacciones (positivas y negativas) puede ser un elemento importante que defina el tipo de estabilidad del sistema. Sin embargo, debe de existir un balance entre las interacciones positivas y negativas, ya que si las segundas se sobreponen a las primeras, puede ocurrir un colapso del sistema traducido en crecimiento desmedido. Este escenario corresponderá a aquellos sistemas (\ref{eqn:LK}) inestables resultantes de ciertas configuraciones en $p$, $\sigma$ y $N$ de la matriz de incidencias y se irá revisando más adelante.\\\\
Dependiendo de los parámetros asociados a $\Lambda$, la probabilidad de conectividad $p$ controla el número esperado de interacciones por especie, mientras que $\sigma$ fija la escala típica de sus magnitudes. En general, cada especie puede interactuar con hasta $N-1$ especies adicionales, dando lugar a cualquiera de los tipos de interacción antes descritos, de acuerdo con el soporte topológico de la red. A medida que el tamaño del sistema \eqref{eqn:LK} crece, su dinámica resultante se vuelve difícil de visualizar de manera directa, por lo que resulta conveniente analizar ejemplos de baja dimensión. En particular, a continuación se extiende el ejemplo bidimensional previo para ilustrar los casos con interacciones de cooperación. 

\begin{ejemplo}\label{eg:2x2CoopyDemás}
	En este ejemplo se ajustan los signos del sistema \eqref{eqn:Sist2x2Comp} como a continuación se muestra 
	\begin{equation}\label{eqn:Sist2x2Coop}
		\begin{split}
			\frac{dx}{dt} &= 2x\left (1-\frac{x}{2}\right )+\frac{1}{2}xy\\
			\frac{dy}{dt} &= 3y\left (1-\frac{y}{3}\right )+xy
		\end{split}
	\end{equation}
	La matriz de coeficientes de interacción asociada es
	$$
	\Lambda = \begin{pmatrix}
		1 & -\frac{1}{2}\\
		-1 & 1
	\end{pmatrix}
	$$
	Se puede apreciar una interacción de cooperación entre las especies $x$ y $y$; en este caso, dicha interacción favorece el crecimiento de ambas poblaciones, permitiendo que el equilibrio de coexistencia se ubique por encima de sus respectivas capacidades de carga. Los estados estacionarios del sistema son $(0,0)$, $(2,0)$, $(0,3)$ y $(7,10)$. Evaluando la matriz Jacobiana en el estado coexistente se obtiene 
	$$
	\mathbb{J}_{(7,10)}=\begin{pmatrix}
	-7 & 3.5\\
	10 & -10
	\end{pmatrix}
	$$
	cuyos valores propios son $\lambda_{1,2}=\frac{-17\pm\sqrt{149}}{2}<0$. Por lo tanto, el punto $(7,10)$ es localmente asintóticamente estable, lo que implica que trayectorias cercanas convergen a este punto de equilibrio conforme $t\to\infty$. En la Figura \eqref{fig:CooperacionEspecies} se muestran las series de tiempo y el espacio fase, donde se observa que las poblaciones se estabilizan en valores superiores a sus respectivas capacidades de carga.
	\begin{figure}[h!]
		\centering
		\includegraphics[scale=0.24]{../Imagenes/Cooperacion de especies}
		\caption{Sistema de Lotka-Volterra con interacciones de cooperación dados por las ecuaciones (\ref{eqn:Sist2x2Coop}). Tasas de crecimiento y capacidades de carga: $r_x=K_x=2$ y $r_y=K_y=3$. (\textbf{A}) Series de tiempo del sistema para las especies $x(t)$ y $y(t)$ bajo la condición inicial $(1,2)$. (\textbf{B}) Espacio fase del sistema con sus puntos fijos asociados, se muestra solamente un único punto fijo estable.}
		\label{fig:CooperacionEspecies}
	\end{figure}
	
	En este contexto, el concepto de capacidad de carga deja de interpretarse como una cota rígida para la población y pasa a entenderse como un parámetro regulador de crecimiento. En ausencia de interacciones interespecíficas, $K_i$ fija el nivel de saturación logística; sin embargo, cuando se incorporan las interacciones, el término $\sum_j \alpha_{ij}x_j$ puede modificar de forma sustancial esta regulación. Si las contribuciones efectivas asociadas a interacciones que favorecen el crecimiento dominan sobre el término autoregulatorio, el sistema puede presentar equilibrios por encima de $K_i$ o incluso dinámicas no acotadas. \\
	 \\
 	En sistemas de baja dimensión pueden obtenerse condiciones explícitas que relacionen la estabilidad con la intensidad de las interacciones; este tipo de resultados se retoman más adelante. \\
 	\\
 	Para ilustrar el papel del signo de los coeficientes, se modifica nuevamente al sistema \eqref{eqn:Sist2x2Coop} para considerar interacciones de comensalismo, amensalismo y depredación. Las simulaciones correspondientes muestran que en el comensalismo la especie beneficiada puede alcanzar niveles superiores a su capacidad de carga efectiva mientras la otra permanece prácticamente inalterada; en el amensalismo, una especie ve reducido su nivel poblacional mientras la otra no experimenta cambios apreciables; finalmente, la depredación combina ambos efectos, con una especie favorecida y la otra perjudicada. Las series de tiempo se muestran en la Figura \eqref{fig:RestoInteraccionesST}
 	\begin{figure}[h!]
 		\centering
 		\includegraphics[scale=0.2]{../Imagenes/STrestoInteracciones}
 		\caption{Series de tiempo para las interacciones comensalismo, amensalismo y depredación modificando el sistema \eqref{eqn:Sist2x2Coop}. (\textbf{A}) Para el comensalismo se definió $\alpha_{21}=0$ y $\alpha_{12}=-\frac{1}{2}$. (\textbf{B}) Para el amensalismo se consideró $\alpha_{21}=0$ y $\alpha_{12}=\frac{1}{2}$ (\textbf{C}) Para la depredación se consideró $\alpha_{21}=-1$ y $\alpha_{12}=\frac{1}{2}$.}
 		\label{fig:RestoInteraccionesST}
 	\end{figure}
\end{ejemplo}
En esta sección se estableció la forma en que la estructura de la conectividad en conjunto con la magnitud de las interacciones determinan la matriz de coeficientes $\Lambda$, así como la clasificación de los distintos tipos de interacción ecológica y su efecto cualitativo sobre la dinámica poblacional. Estos elementos permiten pasar de una descripción estructural del sistema a una análisis dinámico propiamente dicho. En particular, la información obtenida en $\Lambda$ condiciona la existencia y naturaleza de los estados estacionarios del sistema, por lo que en la siguiente sección se abordará el estudio de los puntos de equilibrio y su estabilidad.

\section{Estados estacionarios}\label{sec:PuntosFijos}

El análisis de los estados estacionarios es fundamental para comprender la estabilidad dinámica del sistema. La naturaleza de los puntos de equilibrio depende de las interacciones entre especies, codificadas en la matriz de coeficientes de interacción $\Lambda$. En esta sección se bosquejará un análisis exploratorio centrado en el estado estacionario coexistente, sin perder de vista que constituye solo uno de los múltiples equilibrios posibles. Este estado resulta de particular interés porque representa la coexistencia simultánea en todas las especies y permite examinar, de manera tratable, cómo su estabilidad puede variar en función de los parámetros $(p,\sigma,N)$. Para determinar los estados estacionarios se imponen $\dot{x}_i=0$ en el sistema \eqref{eqn:LK}. Esto implica que para cada $i$,
\begin{equation}\label{eqn:EcsPuntoFijo}
	x_i=0\quad\text{ó}\quad 1-\frac{\sum_{j=1}^N\alpha_{ij}x_j}{K_i}=0
\end{equation}
Por tanto, el sistema admite múltiples soluciones asociadas a la extinción de una o varias especies. El caso de interés es el estado coexistente, en el cual $x_i^*>0$ para toda $i$. En este caso las ecuaciones se reducen a 
$$\sum_{j=1}^N\alpha_{ij}x_j^*=K_i$$
lo cual puede escribirse de forma matricial como
$$\Lambda X^*=\mathrm{K},\qquad\text{con }\mathrm{K}=(K_1,...,K_n)$$
%Aqui hay un argumento fuerte para la estabilidad en función de la tupla.
Siempre que $\Lambda$ sea invertible, el coexistente viene dado por $X^*=\Lambda^{-1}\mathrm{K}$. La fijación de la diagonal en $d=1$ introduce un término de autoregulación que desplaza Gracias a que la diagonal esta fijada al valor $d=1$, mitiga la probabilidad de que sea singular y garantiza que sea invertible con alta probabilidad. El sistema hasta ahora no garantiza que $x_i^*>0$ para toda $x_i^*\in X^*$, puesto que aunque $\mathrm{K}>0$, las entradas de $\Lambda$ pueden ser positivas y negativas con una misma probabilidad. Si se considera a $\mathrm{K}$ con entradas constantes, es decir, $\mathrm{K}=k\vec{1}$ con $k>0$ entonces la distribución de $X^*$ puede darse a su alrededor con base en $(p,\sigma,N)$. \\
\\
Se proponen las siguientes suposiciones: Cuando $p\ll 1$ entonces $\Lambda\approx I$ y por lo tanto $\Lambda^{-1}\approx I$, en consecuencia $X^* \approx \mathrm{K}$. Sin embargo cuando $p$ llega a valores intermedios o grandes, existe mayor cantidad de interacciones que ensancha la distribución alrededor de $\mathrm{K}$ hacia valores negativos, haciendo que $X^*$ contenga entradas negativas. Analizando a $\sigma$ se tiene el caso trivial cuando $\sigma=0$ entonces $\Lambda=I$ y por tanto $X^*=K$. Cuando $\sigma\ll 1$ se generan pequeñas fluctuaciones y la distribución comienza a ensancharse alrededor de $\mathrm{K}$. Cuando $\sigma$ toma valores intermedios $0.3<\sigma<0.5$ entonces la distribución se ensancha de tal forma que comienzan a emerger términos negativos en las entradas de $X^*$. Finalmente cuando $\sigma>0.5$ la cantidad de entradas negativas se amplifica de modo que va disminuyendo la probabilidad de que existan puntos fijos con entradas positivas.\\
\\
Finalmente el papel de $N$ en la tupla que gobierna a $\Lambda$, será el de amplificar los efectos que produce la combinación de $p$ y $\sigma$, ya que al tener mayor cantidad de entradas: se amplifica la posibilidad de que $\Lambda^{-1}\mathrm{K}$ sea más dispersa. ¿Cuál será el efecto de $\Lambda^{-1}$ en la distribución del punto fijo? Esta matriz como tal no será una combinación lineal de normales, sino que dependerá del determinante de $\Lambda$ y sus valores propios\footnote{Considerando que el producto de sus valores propios es igual a su determinante ya que si $\lambda=0$ en el polinomio característico se tiene $p(0)=\det(A-0I)=\det(A)=\lambda_1\cdots\lambda_n$.}. Si por alguna razón $\det(\Lambda)=0$ entonces la matriz es singular y todas las entradas de $\Lambda^{-1}$ divergen a $\infty$. Si $\det(\Lambda)<1$ y además se sabe que 
\begin{equation}\label{eqn:determinante}
	\text{Adj}(\Lambda)\cdot \Lambda = \det(\Lambda)\cdot I
\end{equation}
se puede conjeturar que las entradas de Adj$(\Lambda)$ son comparables en magnitud con las entradas de $\Lambda$. Por lo tanto $\Lambda^{-1}=\frac{1}{\det(\Lambda)}\text{Adj}(\Lambda)$ tendrá entradas ligeramente amplificadas con respecto de $\Lambda$ debido a $\frac{1}{\det(\Lambda)}>1$. Cuando $\det(\Lambda)>1$ y $\Lambda$ mantiene las mismas condiciones que antes, entonces las entradas de su matriz adjunta deberán ser uno o varios ordenes de magnitud mayor que las entradas de $\Lambda$ para poder satisfacer el producto (\ref{eqn:determinante}). Dichas magnitudes irán en función de que tan grande es el el valor del determinante. En este caso la matriz inversa $\Lambda^{-1}$ tendrá entradas amplificadas debido a Adj$(\Lambda)$ pero con un efecto amortiguador proveniente de $\frac{1}{\det(\Lambda)}<1$.
\\
\\
Entonces en cualquiera de los dos casos mencionados, la matriz inversa $\Lambda^{-1}$ presenta entradas amplificadas con respecto de $\Lambda$ en función de la magnitud del determinante. Al realizar la multiplicación $\Lambda^{-1}\mathrm{K}$ se obtiene una distribución alrededor de $\mathrm{K}$ considerablemente más dispersa que si se tomara el producto $\Lambda\mathrm{K}$. Esta dispersión estará dada en función de $p$ y $\sigma$ haciendo que la distribución se quede en valores positivos, cuando estos parámetros son pequeños, o se disperse hacia valores negativos cuando dichos parámetros aumenten.\\
\\
Viendo como el determinante de $\Lambda$ afecta en $\Lambda^{-1}$ y en la distribución de $X^*$, queda preguntarse qué sucede cuando existe un número impar de valores propios negativos que propicie $\det(\Lambda)<0$, ¿devendrán en puntos fijos con entradas negativas? Más aún, ¿existirá una relación entre el espectro de valores propios de $\Lambda$ con la determinación de los puntos fijos? de modo que el signo de la parte real de los valores propios de $\Lambda$ sea determinante del carácter de las entradas del punto fijo. Estas son conjeturas que llegan al límite de esta tesis.
\newpage
\section{Jacobiana del sistema}

¿Existirán puntos fijos con entradas positivas que devengan en una dinámica inestable? En esta sección se explorará sobre la estabilidad en términos de la matriz Jacobiana. Anteriormente se ha comentado sobre la \textit{linealización} del sistema para conocer la estabilidad alrededor de un punto fijo. A estas alturas ya se cuenta con lo necesario para calcular la matriz Jacobiana explícita del sistema (\ref{eqn:LKmatricial}) para cualquier número de especies. El resultado deberá coincidir con una forma semejante a las interacciones que May definió en \cite{may2019stability}. Una vez hallado el punto fijo de interés, se podrá determinar si el sistema es estable o no con base en los valores propios de esta matriz Jacobiana, considerando diversas configuraciones de la tupla $(p,\sigma,N)$ que gobiernan a $\Lambda$.\\
\\
Con base en la definición \ref{def:LKVectorial} se define el sistema de Lotka-Volterra generalizado como $\dot{X}=\textbf{F}(X)$, siendo \textbf{F} la función vectorial no lineal del sistema. Para determinar su matriz Jacobiana se tiene que aplicar las derivadas parciales a cada $f_i(X)\in\mathbf{F}(X)$ y para ello separamos en dos casos: considerando elementos de la diagonal $\frac{\partial f_i(X)}{\partial x_i}$ y elementos fuera de la diagonal: $\frac{\partial f_i(X)}{\partial x_k}$. Para comenzar el cálculo considere las ecuaciones del sistema (\ref{eqn:LK}) 
$$
f_i(X) = r_ix_i\left (1-\frac{\sum_{j=1}^N \alpha_{ij}x_j}{K_i}\right )
$$
las derivadas parciales para los términos de la diagonal son
\begin{equation}\label{eqn:EcsJacobiano_ii}
	\begin{split}
			\frac{\partial f_i(X)}{\partial x_i} &= r_i \left (1-\frac{2\alpha_{ii}x_i+\sum_{j\neq i}\alpha_{ij}x_j}{K_i}\right )
	\end{split}
\end{equation}
Sin embargo, hay que tener en cuenta que se evalúa sobre un punto fijo $X^*$ el cual se genera de la siguiente ecuación  
$$\alpha_{ii}x_i+\sum_{j\neq i}\alpha_{ij}x_j=K_i$$
Sustituyendo $\sum_{j\neq i}\alpha_{ij}x_j=K_i-\alpha_{ii}x_i$ en (\ref{eqn:EcsJacobiano_ii}) finalmente se tiene
$$\frac{\partial f_i(X^*)}{\partial x_i} = -\frac{r_ix_i^*}{K_i}\alpha_{ii}$$
Para los términos que quedan fuera de la diagonal evaluados en $X^*$ se tiene
$$\frac{\partial f_i(X^*)}{\partial x_k} = -\frac{r_ix_i^*}{K_i}\alpha_{ik} $$
\newpage
\begin{definición}\label{def:MatrizJacobiana}
	Sea $\mathcal{J}\in\mathrm{M}_N(\mathbb{R})$ donde $N$ es el número de especies del sistema LV generalizado. Se define la matriz \textit{Jacobiana} asociada al sistema (\ref{eqn:Fmatricial}) evaluado en un punto fijo $X^*$ de la siguiente forma
	\begin{equation}\label{eqn:MatrizJacobiana}
		\mathcal{J}_{ij}=\begin{cases}
			-\frac{r_ix_i^*}{K_i}\alpha_{ii},\qquad&\text{para }i=j\\
			-\frac{r_ix_i^*}{K_i}\alpha_{ij},\qquad&\text{para }i\neq j
		\end{cases}
	\end{equation}
	considerando que $\alpha_{ii}>0$ y particularmente se ha fijado en $\alpha_{ii}=1$ para todo $i\in\{1,...,N\}$. Si se considera una matriz diagonal cuyos elementos son $-\frac{r_ix_i^*}{K_i}$, entonces la Jacobiana se puede reescribir de la siguiente manera
	\begin{equation}\label{eqn:Jacobiana}
		\mathcal{J}=-\text{diag}\left (\frac{r_ix_i^*}{K_i}\right )\Lambda
	\end{equation}
\end{definición}
De la matriz Jacobiana se puede observar como todas sus entradas voltean el signo con respecto de $\Lambda$, lo que parece ajustarse con las interacciones que May define en su \textit{community matrix} \cite{may2019stability}, incluso obedece el signo negativo en la diagonal que será sumamente importante para determinar la estabilidad del sistema. La ec. (\ref{eqn:Jacobiana}) deja ver que la matriz Jacobiana ¡es un re-escalamiento de la matriz de Incidencias! indicando que la estabilidad únicamente depende de $\Lambda$\footnote{Ya que son las interacciones de $\Lambda$ las que inducen la naturaleza del punto fijo.}. ¿Como se podría verificar la estabilidad del sistema con base en el punto fijo y los valores propios de $\mathcal{J}$?

\begin{proposición}\label{prop:DiagonalI}
	Los elementos de la diagonal de la matriz Jacobiana determinan en gran medida la estabilidad del sistema
	\begin{proof}
		Para demostrar esta proposición se hará uso del \textit{Teorema de Gershgorin} \cite{GershgorinTheorem}. Para ello se asume que la matriz Jacobiana tendrá valores propios complejos. Se define el radio de Gershgorin como
		\begin{equation}\label{proof1:RGershgorin}
			R_i=\sum_{i\neq j}|\mathcal{J}_{ij}|
		\end{equation}
		con ello se define $D(\mathcal{J}_{ii},R_i)\subset\mathbb{C}$ como el disco de Gershgorin centrado en $\mathcal{J}_{ii}$ con radio $R_i$. El teorema establece que cada valor propio de $\mathcal{J}$ estará contenido en alguno de estos discos. Demostrando que si todo $D(\mathcal{J}_{ii},R_i)$ se encuentra contenido en el semiplano negativo de $\mathbb{C}$, entonces todos los valores propios de la Jacobiana serán negativos también y por lo tanto el sistema (\ref{eqn:LK}) será estable en $X^*$. El elemento más importante de esta prueba es considerar el centro de los discos; si todos los $\mathcal{J}_{ii}$ son negativos entonces el centro de todos los discos se encuentra en el semiplano negativo de $\mathbb{C}$, y para garantizar que todos los valores propios sean negativos habría que probar $|\mathcal{J}_{ii}|>R_i$ para todo $i\in\{1,...,N\}$, es decir:
		\begin{equation}\label{proof1:Desigualdad}
			\begin{split}
					 \left |-\frac{r_ix_i^*}{K_i}\alpha_{ii}\right |&>\sum_{j\neq i}\left |-\frac{r_ix_i^*}{K_i}\alpha_{ij}\right |\\
				1&>\sum_{j\neq i}|\alpha_{ij}|
			\end{split}
		\end{equation}
		\newpage
		De la ec. (\ref{eqn:MatrizJacobiana}) se puede ver que $\alpha_{ii}>0$ y en particular $\alpha_{ii}=1$, por lo tanto es importante que todo $x_i^*\in X^*$ sea positivo para que el centro de los discos se encuentre en el semiplano negativo de $\mathbb{C}$. En caso contrario habrá al menos un disco con centro en el sempiplano positivo de $\mathbb{C}$ que pueda alojar valores propios positivos que devengan en una dinámica inestable. Es conveniente poder saber cual es el tamaño promedio de los discos de Gershgorin y para ello se define la siguiente variable aleatoria
		\begin{equation}\label{proof1:VarAlW}
			W = 
			\begin{cases}
				0,\qquad \text{Si }1-p\\
				Y,\qquad \text{Si }p>0	
			\end{cases}
		\end{equation}
		donde $Y\sim N(0,\sigma^2)$ es una variable aleatoria normal. Ya que $R_i$ considera suma de valores absolutos, es de interés saber el valor esperado de $|W|$ 
		\begin{equation}\label{proof1:ValorEsperadoW}
			\mathbb{E}[|W|]=(1-p)\cdot 0 + p\cdot \mathbb{E}[|Y|] = p\cdot\mathbb{E}[|Y|] = p\sigma\sqrt{\frac{2}{\pi}}
		\end{equation}
		el término $\sqrt{\frac{2}{\pi}}$ viene de considerar la variable aleatoria $Z\sim N(0,1)$ y la distribución \textit{half-normal} $|Z|$, entonces su valor esperado es $$\mathbb{E}[|Z|]=\frac{1}{\sqrt{2\pi}}\int_{-\infty}^{\infty}|z|e^{-z^2/2}\, dz=\frac{2}{\sqrt{2\pi}}\int_0^\infty ze^{-z^2/2}\, dz=\sqrt{\frac{2}{\pi}}$$
		al considerar la variable aleatoria $Y\sim N(0,\sigma^2)$ entonces basta con escalar $Y=\sigma Z$ y en consecuencia su valor esperado será $$\mathbb{E}[|Y|]=\sigma \mathbb{E}[|Z|]=\sigma\sqrt{\frac{2}{\pi}}$$
		Por lo tanto, el valor esperado del radio de Gershgorin es
		\begin{equation}\label{proof1:E[RadioG]}
			\mathbb{E}[R_i]=(N-1)p\sigma\sqrt{\frac{2}{\pi}}
		\end{equation}
		Si este radio promedio es menor a 1 se puede garantizar en cierta medida que los discos de Gershgorin estarán contenidos en el semiplano negativo de $\mathbb{C}$. Sin embargo, pueden existir discos de Gershgorin que se alejen de la media ¿qué tanto se pueden alejar para que se siga cumpliendo $\mathcal{J}_{ii}>R_i$? será conveniente calcular la varianza de los radios de Gershgorin para averiguarlo. Tomando la variable aleatoria $W$, se calcula su segundo momento
		$$\mathbb{E}(|W|^2)=(1-p)\cdot 0+p\cdot\mathbb{E}(|Y|^2)=p\cdot\mathbb{E}(Y^2)=p\sigma^2$$
		por otro lado el cuadrado del valor esperado (\ref{proof1:ValorEsperadoW}) es
		$$\mathbb{E}(|W|)^2=p^2\sigma^2\frac{2}{\pi}$$
		Finalmente la varianza de $|W|$ es
		$$\Var(|W|)=p\sigma^2-p^2\sigma^2\frac{2}{\pi}$$
		Entonces la varianza de los radios de Gershgorin queda envuelta en la siguiente expresión
		\begin{equation}\label{proof1:VarRadioG}
			\Var(R_i)=(N-1)p\sigma^2\left (1-\frac{2p}{\pi}\right )
		\end{equation}
		y mientras sea menor a 1 se puede garantizar que los discos de Gershgorin van a cumplir (\ref{proof1:Desigualdad}) y en consecuencia los valores propios de $\mathcal{J}$ estarán contenidos en el semiplano negativo de $\mathbb{C}$. Sin embargo esta prueba tiene algunas limitantes; los discos de Gershgorin únicamente garantizan que contienen valores propios, forzar la condición (\ref{proof1:Desigualdad}) es suficiente para garantizar estabilidad pero no es determinante, ya que este método no menciona nada acerca de la posición de los valores propios. Si dicha desigualdad no se cumple pueden ocurrir dos escenarios: que $\max(|\lambda_\mathcal{J}|)<0$ y el sistema siga siendo estable a pesar de algún $R_i>\mathcal{J}_{ii}$ ó $\max(|\lambda_\mathcal{J}|)>0$ en cuyo caso el sistema será inestable, y eso dependerá de $\Lambda$ y la tupla $(p,\sigma,N)$. En conclusión, esta prueba funciona en un primer nivel para mostrar que los elementos de la diagonal de $\mathcal{J}$ y por consiguiente las entradas de $X^*$ juegan un papel importante en la estabilidad del sistema. 
	\end{proof}
\end{proposición}

Es importante recordar que las interacciones de autoregulación se mantienen en la diagonal de $\Lambda$ (Ver ec. (\ref{eqn:LKextendida})); cuando la matriz de incidencias se transforma en la Jacobiana, el punto fijo debe de contener entradas positivas para poder respetar las interacciones de autoregulación que May define en \cite{may1972will}. Si el punto fijo contiene entradas negativas es casi un hecho que el sistema será inestable porque existirá al menos un disco de Gershgorin con centro en el semiplano positivo de $\mathbb{C}$ que independientemente de como sea su radio, es altamente probable que contenga valores propios positivos.\\
\\
Para poder determinar un parámetro crítico de transición en la estabilidad de los sistemas, es imprescindible conocer donde se ubica el $\max(\lambda_\mathcal{J})$ y encontrar la forma de controlar su posición para poder saber las condiciones que lo llevan a ser negativo, cero o positivo. Aunque existe una forma dada para resolver esta conjetura, la cual se abordará en la siguiente sección, tiene un gran inconveniente: la Jacobiana tiene en su diagonal toda una distribución de valores que dificulta la definición de un radio espectral que determine la posición de Re$(\lambda_{\max}(\mathcal{J}))$. Esta conjetura ligada con la correspondiente de la sección anterior también llega al límite de esta tesis y más adelante se presentará un análisis detallando sus razones.
\newpage
