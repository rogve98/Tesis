\chapter{Estabilidad del sistema de Lotka-Volterra generalizado}

\setlength{\parindent}{0cm} El capítulo anterior motiva al actual para presentar los resultados de la dinámica que produce el sistema de Lotka-Volterra generalizado (\ref{eqn:LK}) bajo los coeficientes de interacción de la matriz de incidencias (\ref{eqn:MatrizIncidencias}) para posteriormente linearizarlo mediante la Matriz Jacobiana del sistema (\ref{eqn:MatrizJacobiana}). En este capítulo se presentarán los resultados que produce cada etapa del proceso, así como sus características. El objetivo será dar respuesta a cada una de las hipótesis del planteamiento del problema, sobre todo indagar las condiciones de estabilidad del sistema en términos de $\Lambda$ y reforzar la idea consecuente de la proposición \ref{prop:paramMay} en la que se detallaba que el radio espectral (\ref{eqn:radioMayGirko}) no necesariamente se ajusta a la estabilidad del sistema de Lotka-Volterra.\\
\\
Se consideraron 3 conjuntos diferentes de simulaciones: sistemas para 25, 50 y 100 especies. Para explorar los resultados se dejaron fijos la mayor cantidad de parámetros que hay en (\ref{eqn:LK}) con el fin de observar cambios significativos con la menor cantidad de fluctuaciones posibles; únicamente se han variado las matrices de incidencias $\Lambda$ que dependen de $p$ y $\sigma$, y al igual que en las transiciones de May y semi Allesina se varió una de estas cantidades mientras que la otra permaneció fija. En todas las simulaciones la tasa de crecimiento se dejó fija en $r_i=2$ y la capacidad de carga en $K_i=5$ para cada especie del sistema. Al integrar numéricamente las ecuaciones con RK4 se escogió un intervalo de tiempo entre 0 y 50 con un paso de integración de $h=0.01$.\\
\\
Además de estos parámetros, siempre se inicializó cada simulación con la condición inicial $\vec{x}_0=\vec{1}$ y se consideraron dos escenarios: Matrices de incidencias estructuralmente simétricas y puramente aleatorias\footnote{Solo aplica para $N=100$, el resto de casos se consideraron $\Lambda$ estructuralmente simétricas.}, tal y como se visualizó al final del capítulo anterior. En el apéndice (\ref{ch:Ap}), el lector puede darse una idea de como se realizó el proceso de las simulaciones. En cada escenario se presentó cierta cantidad de ruido en las gráficas de estabilidad, por lo que el número de simulaciones fue establecido en función de la disminución de dicho ruido, siguiendo la ley de los grandes números.

\section{Series de tiempo}

Anteriormente se ha comentado que las interacciones de la matriz de incidencias $\Lambda$ están volteadas con respecto de la matriz de Jacobiana del sistema (\ref{eqn:MatrizJacobiana}); por ejemplo, la cooperación en $\mathcal{J}_\Lambda$ se da para las interacciones (++) mientras que para $\Lambda$ es $(--)$. Si se tuviera un sistema puramente de competencia, es decir para toda $\alpha_{ij}\in\Lambda$ mayor o igual que cero, entonces no hay forma de que ninguna de las poblaciones participantes sobrepasen la capacidad de carga establecida, tal y como se indica en la ecuación (\ref{eqn:LKextendida}) y en el Ejemplo \ref{eg:2x2}. Por lo tanto se obtendrían series de tiempo caóticas para cada una de las especies por debajo de $K=5$.
\begin{figure}[h!]
	\centering
	\includegraphics[scale=0.23]{../Imagenes/Seriesdetiempopositiva}
	\caption{Series de tiempo para el sistema de competencia de especies. Se emplea una matriz de incidencias con $N=100$ cuyas entradas vienen de una distribución uniforme del intervalo $[0,1]$. Se considera a la totalmente conexa, es decir, con $p=1.0$. En este caso la dinámica no sobrepasa la capacidad de carga puesto que las 100 especies se encuentran compitiendo y obedeciendo el comportamiento logístico que se muestra en (\ref{eqn:LKextendida}).}
	\label{fig:Seriesdetiempopostiva}
\end{figure}

Una de las características que se encontró en los sistemas de competencia, es el tiempo que tarda en estabilizarse es considerablemente mayor que en sistemas donde se considerarán todas las interacciones que puede ofrecer una distribución normal de media cero y varianza finita (cooperación, presa-depredador, etc.). Las poblaciones en este caso se encuentran limitadas por la capacidad de carga y serán incapaces de superarla ya que el término de la derecha en (\ref{eqn:LKextendida}) es siempre positivo y por tanto resta al término logístico de la izquierda.
\\
\\
En contraste con el sistema generalizado, éste sí excede la capacidad de carga ya que el término de la derecha en (\ref{eqn:LKextendida}) puede sumar al de la izquierda y permite que pueda superar el límite que la capacidad de carga impone, lo que se traduce en especies dominantes que regulan el sistema y provocan que llegue al atractor en un tiempo menor. Otro aspecto de la red de competencias es que cuando es más conectada el tiempo en estabilizarse también es mayor. La cantidad de fluctuaciones posibles es cada vez mayor, y el sistema no lineal es sensible en presencia de estas fluctuaciones, por lo tanto es esperable que el tiempo en llegar al equilibrio sea prolongado. Por el contrario, si la red de competencias tiene pocas conexiones ($p\leq 0.5$) entonces el sistema tarda menos en estabilizarse.\\
\\
En el Ejemplo \ref{eg:2x2CoopyDemás} del capítulo anterior, se observaba como las interacciones de cooperación $(--)$ en la matriz de incidencias $\Lambda$ genera la aparición de un atractor que se posiciona por arriba de la magnitud de la capacidad de carga (Figura (\ref{fig:CooperacionEspecies})). En el caso extendido a $N\gg 1$ especies ocurrirá algo semejante considerando un atractor $N$-dimensional. En este caso pueden haber especies que sobrepasen por mucho o poco la capacidad de carga, pero también cabe la posibilidad de que algunas no logren sobrepasarla y otras que lleguen a extinguirse. A continuación se muestran dos ejemplos diferentes
\begin{figure}[h!]
	\centering
	\includegraphics[scale=0.23]{../Imagenes/Series de Tiempo LK100}
	\caption{(\textbf{A}) Series de tiempo del sistema LV generalizado asociada a una matriz de incidencias de $N=100$, con $\sigma=0.2$ y $p=0.35$. (\textbf{B}) Series de tiempo para el sistema LV generalizado asociada a una matriz de incidencias de $N=100$, con $\sigma=0.2$ y $p=0.5$}
	\label{fig:SeriesdeTiempoLK100}
\end{figure}
\newpage
La diferencia evidente entre estas gráficas es debido la matriz de incidencias, sus parámetros son $\sigma=0.2$ con $p=0.35$ y $p=0.5$ respectivamente. El segundo caso corresponde con una red más conectada que el primero y eso se traduce en la oportunidad de tener más interacciones con signo negativo que propicien un mayor crecimiento. El tiempo en que llegaron a estabilizarse fue menor a $t=50$, y a su vez considerablemente menor que el sistema de competencia de especies de la Figura (\ref{fig:Seriesdetiempopostiva}). 


\section{Puntos fijos}\label{subsec:PuntosFijos2.0}

Algo interesante de notar en los sistemas de competencia es la tendencia del punto fijo. En la sección \ref{subsec:PuntosFijos} se propuso que una posible solución del punto fijo sería $X^*=\Lambda^{-1}\mathrm{K}$, se había discutido que su distribución podría ser de cola pesada con posibilidad de tener entradas negativas. Si $\Lambda$ tiene todos 
\begin{wrapfigure}{r}{0.5 \textwidth} \vspace{-25pt} \begin{center}
		\includegraphics[scale=0.13]{../Imagenes/PuntoFijoCompetencia}
	\end{center}
	\vspace{-20pt} 
	\caption{Distribución de punto fijo para caso particular del sistema de competencia de especies, escogiendo $p=0.5$ y considerando una distribución uniforme en los pesos de las interacciones.}
	\vspace{-10pt}
	\label{fig:PuntoFijoCompetencia}
\end{wrapfigure}
sus coeficientes positivos provenientes de la distribución uniforme en el intervalo $[0,1]$ entonces la distribución de $X^*$ podría ser diferente pero lo que se observa en la Figura (\ref{fig:Seriesdetiempopostiva}) es que la mayoría de las especies tienden a cero (se extinguen) mientras que unas cuantas logran sobrevivir por debajo de la capacidad de carga, empatando con una distribución de cola pesada tal y como se observa en la Figura (\ref{fig:PuntoFijoCompetencia}). Esta evidencia muestra un parteaguas en donde no necesariamente el punto fijo y su distribución estarán dados por $X^*=\Lambda^{-1}\mathrm{K}$, sino que el punto fijo favorable puede ser alguno en donde una o varias especies del sistema se extinguen, es decir, resolver las ecuaciones  del sistema (\ref{eqn:EcsPuntoFijo}) cuando se asume que existe un subconjunto que son cero: $S=\{x_k=0\ |\ k\in\mathcal{I}\, \wedge\,  \mathcal{I}\subset \{1,...,N\}\}$. Pero nuestro caso de mayor interés es lo que se logra observar en la Figura (\ref{fig:SeriesdeTiempoLK100}); se tienen dos sistemas estables que definitivamente muestran una distribución de cola pesada puesto que un número reducido de ellas se encuentran dispersas en valores mayores a la capacidad de carga pero es posible hallar a la mayoría concentradas en valores cercanos al cero (Ver Figura (\ref{fig:PuntoFijoCoop3}) \textbf{A}).
\\
\\
Anteriormente se ha discutido que mientras los valores de $\sigma$ y $p$ en la matriz de incidencias $\Lambda$ sean pequeños con respecto del otro, es decir, $\sigma\ll p$ ó $p\ll\sigma$: se obtendría una distribución alrededor del valor fijo de la capacidad de carga, en este caso $K=5$. Bajo esta premisa, se explorará para que escenarios la configuración $X^*=\Lambda^{´-1}\mathrm{K}$ representa correctamente el punto fijo y su distribución. A continuación se presentan una serie de histogramas que muestran gráficamente la comparación entre el resultado numérico y el teórico
\begin{figure}[h!]
	\centering
	\includegraphics[scale=0.16]{../Imagenes/PuntoFijoCoop1}
	\caption{Distribución del punto fijo para un sistema LV generalizado con $p=0.5$ y $\sigma=0.003$. En este caso se cumple $\sigma\ll p$ y por lo tanto la distribución coincide con $X^*=\Lambda^{-1}\mathrm{K}$.}
	\label{fig:PuntoFijoCoop1}
\end{figure}

\begin{figure}[h!]
	\centering
	\includegraphics[scale=0.16]{../Imagenes/PuntoFijoCoop2}
	\caption{Distribución del punto fijo para un sistema LV generalizado con $p=0.003$ y $\sigma=0.5$. En este caso se cumple $p\ll \sigma$ y por lo tanto la distribución coincide con $X^*=\Lambda^{-1}\mathrm{K}$.}
	\label{fig:PuntoFijoCoop2}
\end{figure}
En ambos escenarios se puede notar que la distribución del punto fijo se acumula alrededor de $K=5$ y realmente coincide con $X^*=\Lambda^{-1}\mathrm{K}$, se observará en los siguientes histogramas que cuando $\sigma$ y $p$ incrementen sus valores respectivamente, los puntos fijos ya no coincidirán con esta forma.\\
\\
\begin{figure}
	\centering
	\includegraphics[scale=0.16]{../Imagenes/PuntoFijoCoop3}
	\caption{Distribución del punto fijo para un sistema LV generalizado con $p=0.5$ y $\sigma=0.2$. }
	\label{fig:PuntoFijoCoop3}
\end{figure}

Con base en estos resultados se complementa la sección \ref{subsec:PuntosFijos} con el hecho de que para los casos $\sigma\ll p$ y $p\ll\sigma$, el punto fijo coincide con $X^*=\Lambda^{-1}\mathrm{K}$. Sin embargo, cuando $\sigma$ y $p$ son del mismo orden de magnitud entonces la distribución de $\Lambda^{-1}\mathrm{K}$ se ensancha hacia valores negativos mismos que no se pueden admitir en una dinámica estable (Figura (\ref{fig:PuntoFijoCoop3} \textbf{B})). Lo que se muestra en las simulaciones es que el punto fijo sigue una distribución de cola pesada con sesgo positivo alrededor del cero, la Figura (\ref{fig:PuntoFijoCoop3} \textbf{A}) indica que al menos un 65\% de las especies se han extinguido ó están cerca de extinguirse, mientras que el resto se ha estabilizado desde valores relativos que van del 10 al 200.\\
\\
Los puntos fijos con $p$ y $\sigma$ en el mismo orden de magnitud necesitan tener múltiples especies extintas para que pueda ser estable. Esta característica sigue dependiendo enteramente de las interacciones de $\Lambda$ y más adelante se observará como decae la estabilidad con base en el incremento de estos parámetros. Ya que la distribución del punto fijo es de cola pesada, tendrá un impacto directo en la distribución de valores propios de la matriz Jacobiana asociada a $\Lambda$. Cada valor de la diagonal será un tipo de atractor de valores propios en $\mathbb{C}$ y al estar distribuidos de forma heterogénea y re-escalados según (\ref{eqn:MatrizJacobiana}), la distribución de valores propios ya no será uniforme como en el caso de May o Allesina, sino que tendrá una forma más amorfa y/o dispersa. Por último se debe de aclarar que para obtener las diferentes distribuciones del punto fijo de los resultados numéricos, únicamente bastó con agarrar el último punto de la serie de tiempo y concentrarlo en un arreglo que representará al punto fijo.
\newpage

\section{Matriz Jacobiana}

Una vez que se obtienen los puntos fijos a partir de las series de tiempo, se evalúan en la Jacobiana (\ref{eqn:MatrizJacobiana}) y se valida su estabilidad con base en la parte real de sus valores propios asociados. En cambio, si la serie de tiempo diverge entonces se asume que el sistema es inestable. Un elemento importante a considerar de este proceso es que no es trivial determinar cuando se estabilizará el sistema. La serie de tiempo puede mostrar una referencia visual pero la información que entrega no es determinante. En general se ha observado que gran parte de las simulaciones realizadas corresponden a sistemas que se estabilizan en un tiempo menor o igual a $t=50$, sin embargo, existen sistemas que se estabilizan en un tiempo superior a este lo que genera inconsistencias en los resultados. Para evitar que esto ocurra se incluye una restricción en los algoritmos para considerar sistemas que se estabilicen en un tiempo menor o igual a $t=50$.
\\
\\
La forma de verificar la estabilidad recae en la parte real de los valores propios de la matriz $\mathcal{J}_\Lambda$, si por ejemplo la serie de tiempo aparenta estabilizarse pero los valores propios del sistema tienen parte real positiva, entonces quiere decir que el sistema no ha alcanzado a su punto fijo atractor. A continuación se presentan cuatro ejemplos arbitrarios de sistemas que resultaron estables para ir observando sus características. En este \href{https://github.com/rogve98/Tesis/tree/master/Notebooks/Datos/Ejemplo\%20Jacobianos}{enlace}\footnote{Consultar: \url{https://github.com/rogve98/Tesis/tree/master/Notebooks/Datos/Ejemplo\%20Jacobianos}} el lector tendrá acceso a 4 de estas matrices para los parámetros $N=100$, $\sigma=0.2$ y $p=\{0.3,0.4,0.5,0.6\}$. A continuación se presenta el código en \julia para validar como son sus diagonales.
\begin{tcolorbox}[colback=green!10!white, colframe=black, title=Entrada]
	\begin{minted}{julia}
using CSV, DataFrames
jacobianos = []
for i in 0.3:0.1:0.6
  ruta = "Datos/Ejemplo Jacobianos/Jacobiano100_p$i.s0.2.csv"
  df = CSV.read(ruta,DataFrame,header=false)
  push!(jacobianos,df)
end
jacobianos
	\end{minted}
\end{tcolorbox}

\begin{tcolorbox}[colback=red!10!white, colframe=black, title=Salida]
4-element Vector{Any}:\\
100×100 DataFrame
\end{tcolorbox}

\begin{tcolorbox}[colback=green!10!white, colframe=black, title=Entrada]
	\begin{minted}{julia}
using LinearAlgebra
print(all(x -> x<=0,diag(Matrix(jacobianos[1]))),", ")
print(all(x -> x<=0,diag(Matrix(jacobianos[2]))),", ")
print(all(x -> x<=0,diag(Matrix(jacobianos[3]))),", ")
print(all(x -> x<=0,diag(Matrix(jacobianos[4]))))
	\end{minted}
\end{tcolorbox}

\begin{tcolorbox}[colback=red!10!white, colframe=black, title=Salida]
true, true, true, true
\end{tcolorbox}

Si las diagonales de estas matrices son todas negativas o cero quiere decir que el punto fijo asociado tiene entradas mayor o igual que cero ya que $\mathcal{J}_{ii}=-\frac{r_i}{K_i}x^*_i$ con $x_i^*\in X^*$. Por lo tanto, la distribución de la diagonal de $\mathcal{J}_\Lambda$ será de cola pesada pero ahora con sesgo negativo. Esta distribución se ensancha en función de los parámetros, en particular con este ejemplo se observará que cuando la red es más conectada: la distribución será más dispersa. 
\begin{figure}[h!]
	\centering
	\includegraphics[scale=0.16]{../Imagenes/DistDiagonal}
	\caption{Distribuciones de diagonales de las matrices Jacobianas del ejemplo, considerando los parámetros: $N=100$, $\sigma=0.2$ y $p\in\{0.3,0.4,0.5,0.6\}$.}
	\label{fig:DistDiagonal}
\end{figure}

La distribución de los valores propios en $\mathbb{C}$ dependerá en gran medida de las entradas de la diagonal, entre más dispersa sea también lo serán los valores propios. En la proposición \ref{prop:DiagonalI} se introdujo la importancia de la diagonal de las matrices Jacobianas con respecto de la posición general de sus valores propios. Los discos de Gershgorin están centrados en cada valor de esta diagonal, por lo tanto los valores propios quedarán dispersos sus alrededores. En la proposición \ref{prop:paramMay} se mostró como surge el radio espectral (\ref{eqn:radioMayGirko}) cuando el centro de los discos esta fijado a un solo valor $-d$. Actualmente la pregunta será ver si existe un tipo de radio espectral para delimitar la distribución de valores propios de las matrices $\mathcal{J}_\Lambda$.

\section{Leyes Circulares}

Si la distribución de la diagonal es dispersa y no necesariamente simétrica (ver Figura (\ref{fig:DistDiagonal})) entonces no tiene sentido definir un radio espectral sino más bien concentrarse en el valor de la diagonal más cercano al cero y los valores propios de su alrededor para conocer las condiciones en $\Lambda$ que llevan a que $\mathcal{J}_\Lambda$ tenga valores propios positivos o negativos en esa región. Para ir desarrollando esta idea primero se revisarán las distribuciones de valores propios de las Jacobianas antes presentadas
\begin{figure}[h!]
	\centering
	\includegraphics[scale=0.24]{../Imagenes/DistEigenvalores}
	\caption{Distribución de valores propios de las matrices Jacobianas del ejemplo, considerando los parámetros $N=100$, $\sigma=0.2$ y $p\in\{0.3,0.4,0.5,0.6\}$.}
	\label{fig:DistEigenvalores}
\end{figure}

Se ha propuesto en cada distribución un disco con centro y radio con la parte real más negativa del conjunto de valores propios: en consecuencia se genera el disco con el radio más grande posible de cada sistema. Se puede observar que dichos discos si encierran a cada una de las distribuciones pero no se ajustan al círculo en concreto, tal y como sucede en los sistemas de May/Allesina. Sin embargo aún así se observa que toda la distribución se encuentra contenida en el semiplano negativo de $\mathbb{C}$. \\
\\
El ensanchamiento y/o dispersión de cada distribución de valores propios parece tener relación con la distribución de la diagonal (Figura (\ref{fig:DistDiagonal})), y ésta a su vez tiene relación directa con los puntos fijos asociados. Por lo tanto el punto fijo determina la dispersión de las distribuciones del sistema. Una forma de poder ajustar esta distribución amorfa es proponiendo $N$ discos cuyos centros y radios sean los valores de la diagonal de $\mathcal{J}_\Lambda$. Para dar un acercamiento a esta propuesta, se ocupará la distribución más ancha para $p=0.6$:
% Sería una propuesta semejante a los discos de Gershgorin y serviría para enfocarse principalmente en los valores propios que se encuentran cercanos al cero\footnote{Que son de gran interés ya que son los más propensos a desplazarse al semiplano positivo de $\mathbb{C}$, volviendo inestable al sistema.}.
\begin{figure}[h!]
	\centering
	\includegraphics[scale=0.17]{../Imagenes/LeyesCirculares}
	\caption{Distribución de valores propios del sistema generalizado para $N=100$, $\sigma=0.2$ y $p=0.6$. Se consideran $N$ discos cuyo centro y radio es cada valor de la diagonal de la matriz Jacobiana asociada. }
	\label{fig:LeyesCirculares}
\end{figure}


\begin{wrapfigure}{l}{0.5 \textwidth} \vspace{-40pt} \begin{center}
	\includegraphics[scale=0.13]{../Imagenes/LeyCircularParticular}
	\end{center}
	\vspace{-20pt} 
	\caption{Caso particular de la Figura (\ref{fig:LeyesCirculares}) para el valor de la diagonal $\alpha_{ii}=32.46\in\mathcal{J}_\Lambda$.}
		\vspace{-10pt}
	\label{fig:LeyCircularParticular}
\end{wrapfigure}
Esta propuesta se inspira de la Proposición \ref{prop:DiagonalI}, sin embargo, éstos discos no corresponden a los de Gershgorin debido a que los radios no necesariamente tienen la misma magnitud que el centro del disco, pueden ser mayores o menores. Se puede observar que esta elección de discos parece ajustarse a la distribución de valores propios, al menos de forma local para ciertos conjuntos. Si la varianza de los radios de Gershgorin (\ref{proof1:paramEstabilidad}) fuera menor o igual a 1, entonces esta propuesta sería más semejante a los discos de Gershgorin, pero en general esto solo se cumple bajo condiciones específicas. Un caso particular de la Figura (\ref{fig:LeyesCirculares}) considerando $\mathcal{J}_{ii}=32.46$ se puede observar en la Figura (\ref{fig:LeyCircularParticular}), cierto porcentaje de valores propios se distribuye por el círculo varios concentrados en el eje real, insinuando que al ser estructuralmente simétrica la matriz $\mathcal{J}_\Lambda$: tiene correlaciones positivas. Sin embargo existen algunos otros valores propios que se escapan del confinamiento particular y pasan al siguiente nivel o círculo. Para poder confirmar o refutar esta propuesta, se explorarán múltiples sistemas estables con sus respectivas distribuciones de diagonales y valores propios.

\subsection{Análisis para $N=50$}

Se tienen a disposición dos conjuntos de datos, ambos son de simulaciones realizadas para diferentes valores de $p$ y $\sigma$ considerando matrices $\Lambda$ estructuralmente simétricas. Cada conjunto esta conformado por 78 archivos .csv y son simulaciones de matrices \href{https://github.com/rogve98/Tesis/tree/master/Notebooks/Datos/Jacobianos}{Jacobianas} y de distribuciones de \href{https://github.com/rogve98/Tesis/tree/master/Notebooks/Datos/Diagonales}{Diagonales}\footnote{Para los Jacobianos acceder a: \url{https://github.com/rogve98/Tesis/tree/master/Notebooks/Datos/Jacobianos}. Para las Diagonales acceder a \url{https://github.com/rogve98/Tesis/tree/master/Notebooks/Datos/Diagonales}.}. Así mismo, en cada archivo se encuentran 100 simulaciones que resultaron estables y de los cuales se realizará el análisis correspondiente a la relación entre la distribución de las diagonales y de las partes reales de los valores propios de las matrices Jacobianas. En la siguiente tabla se muestra el esquema de simulaciones para cada $p$ y $\sigma$
\begin{table}[h!]
	\centering
	\begin{tabular}{|c|c|c|c|}
		\hline
		Fuerza promedio [$\sigma$] & Probabilidades [$p$] & Cantidad de archivos & Simulaciones realizadas \\ \hline
		$0.1-0.5$  & $0.1-1.0$  & 50 & 5000  \\ \hline
		0.6  & $0.1-0.9$  & 9 & 900 \\ \hline
		0.7  & $0.1-0.7$  & 7 & 700 \\ \hline
		0.8  & $0.1-0.5$  & 5 & 500 \\ \hline
		0.9  & $0.1-0.4$  & 4 & 400 \\ \hline
		1.0  & $0.1-0.3$  & 3 & 300 \\ \hline
		& \textbf{Total:} & 78& 7800\\ \hline
	\end{tabular}
	\caption{Cantidad de archivos generados para el banco de Diagonales y Jacobianos considerando $N=50$. A partir de $\sigma=0.6$ en adelante, los tiempos de compilación fueron muy prolongados por lo que no se obtuvieron los 10 archivos respectivos a diferencia de los valores promedio anteriores.}
	\label{tab:Simulaciones}
\end{table} 

La razón principal de escoger los sistemas para $N=50$ es por el costo computacional. Ha resultado muy prolongado el tiempo de compilación de sistemas más grandes (como el de $N=100$) y aún así para valores altos de la fuerza promedio ($\sigma\geq 0.6$) no se obtuvieron los 10 archivos por la misma razón. Con esta información se procede a continuar con el análisis de las $N$ Leyes Circulares, observando primero como es la distribución de los valores propios de cada conjunto así como la relación que existe entre la parte real y los valores de la diagonal de las $\mathcal{J}_\Lambda$.
\newpage
\begin{figure}[h!]
	\centering
	\includegraphics[scale=0.2]{../Imagenes/ReEvs-Diagonales}
	\caption{(\textbf{A}) Distribución de valores propios de 100 jacobianos para el caso $\sigma=0.6$, $p=0.7$. Se agrega una ley circular correspondiente al valor medio de la distribución de diagonales. (\textbf{B}) Relación entre la parte real de los valores propios con las diagonales de los Jacobianos considerados.}
	\label{fig:ReEvs-Diagonales}
\end{figure}

La distribución de los valores propios queda contenida por completo en el semiplano negativo de $\mathbb{C}$ asegurando que todas las simulaciones realizadas para el conjunto correspondiente ($\sigma=0.6$, $p=0.7$) están asociadas a sistemas estables. Se puede observar en la Figura (\ref{fig:ReEvs-Diagonales} \textbf{A}) que gran parte de los valores propios se quedan contenidos en el eje real; debido a que todas las simulaciones contempladas son para sistemas estructuralmente simétricos, esta observación refuerza estadisticamente el argumento de poseer correlaciones positivas en las interacciones de $\mathcal{J}_\Lambda$. 
\\
\\
Por otro lado en la Figura (\ref{fig:ReEvs-Diagonales} \textbf{B}) se observa una relación lineal prácticamente perfecta, insinuando que para este caso particular: hay una dependencia directa entre las entradas de la diagonal y la parte real de los valores propios de las Jacobianas $\mathcal{J}_\Lambda$. Sin embargo el ajuste parece ser mejor para los valores cercanos al cero y la razón puede ser la siguiente: Cada valor de la diagonal es un centro particular de una vecindad de Gershgorin y cada valor propio se encuentra en al menos una de estas vecindades; entonces cuando la magnitud de los centros es cada vez mayor, su radio también lo será y por lo tanto la vecindad es cada vez más amplia, haciendo que el valor propio pueda adoptar una posición considerablemente apartada de su centro y en consecuencia no tener una relación tan proporcional como en los casos donde el centro y radio de Gershgorin es cercano a cero.
\newpage
\begin{wrapfigure}{r}{0.5 \textwidth} \vspace{-30pt} \begin{center}
		\includegraphics[scale=0.13]{../Imagenes/RadsDiagsGersh}
	\end{center}
	\vspace{-20pt} 
	\caption{Relación entre centros y radios de Gershgorin de las Jacobianas asociadas a la Figura (\ref{fig:ReEvs-Diagonales}).}
	\vspace{-10pt}
	\label{fig:RadsDiagsGersh}
\end{wrapfigure}
En la Figura (\ref{fig:RadsDiagsGersh}) se puede observar la relación entre cada centro y radio de Gershgorin de las Jacobianas asociadas a la Figura (\ref{fig:ReEvs-Diagonales}). En esta gráfica hay dos características a destacar: primero que efectivamente existe una relación creciente entre la magnitud del centro con su radio asociado: son proporcionales; por lo tanto existen valores propios capaces de alejarse del centro de su vecindad de Gershgorin volviendo no tan proporcional la relación entre los valores de la diagonal y autovalores de mayor magnitud. En segundo se puede observar que hay un gran número de centros con radio cero. De acuerdo con la definición de la matriz Jacobiana (\ref{eqn:MatrizJacobiana}) las entradas del punto fijo afectan a cada fila entera de esta matriz, y como se ha visto en las series de tiempo (Figura (\ref{fig:SeriesdeTiempoLK100})) y la distribución del punto fijo (Figura (\ref{fig:PuntoFijoCoop3} \textbf{A})) existen múltiples especies que decaen a cero. Por lo tanto es natural observar este fenómeno, pues son la representación de la cantidad de especies que resultaron extintas.
%En la Figura (\ref{fig:ReEvs-Diagonales} \textbf{A}) se ha agregado una Ley Circular particular con centro y radio correspondiente al valor medio de la distribución de las partes reales de los valores propios, esto con el fin de observar que tan significativa es esta media en función de cuantos valores propios es capaz de encerrar. Si la distribución de autovalores fuera uniforme, la media podría capturar casi la totalidad de ellos (como en el caso de May), sin embargo al no ser uniforme solo será capaz de encerrar una porción del total. 
\begin{figure}[h!]
	\centering
	\includegraphics[scale=0.16]{../Imagenes/AjustesLinMeds}
	\caption{(\textbf{A}) Ajuste lineal de la relación entre las medias de las $Re(\overline{\lambda})$ con las medias de las $d(\sigma_i,p_j)$ de las Jacobianas del sistema. (\textbf{B}) Ajuste lineal de la relación entre las medianas $Re(\overline{\lambda})$ con las medianas de las $d(\sigma_i,p_j))$ de las Jacobianas del sistema. (\textbf{C}) Ajuste lineal de la relación entre las modas de $Re(\overline{\lambda})$ con las modas de las $d(\sigma_i,p_j)$ de los Jacobianas del sistema.}
	\label{fig:AjustesLinMeds}
\end{figure}

Para seguir explorando si la relación lineal existe también para el resto de las simulaciones de la Tabla (\ref{tab:Simulaciones}) se tomará la media, mediana y moda como cantidades representativas de cada una de las distribuciones de diagonales $d(\sigma_i,p_j)$ y partes reales de los conjuntos de valores propios $\overline{\lambda}$ para compararlos y realizando un ajuste lineal.\\
\\
Cada color se encuentra asociado a los valores de la fuerza promedio $\sigma$ y la cantidad de puntos del mismo color corresponde a cada una de las probabilidades de conectividad del sistema. Se ha visto en la Figuras (\ref{fig:DistDiagonal}, \ref{fig:DistEigenvalores}) que cuando esta probabilidad es cada vez mayor entonces las distribuciones son más dispersas en función de la cantidad de interacciones posibles, y a su vez de la dependencia que hay con los valores del punto fijo del sistema. Por lo tanto hacia la izquierda estarán las probabilidades más grandes e irán decreciendo hacia la derecha.\\
\\
En cada gráfica de la figura se contemplan los 78 conjuntos de simulaciones realizadas y se percibe una correspondencia entre los valores de la diagonal con las partes reales de los valores propios. Al igual que en la Figura (\ref{fig:ReEvs-Diagonales}), cada ajuste es mejor para valores cercanos al cero que para aquellos de magnitud máxima, esto debido al argumento antes mencionado sobre la amplitud del radio de Gershgorin que implica que los valores propios tengan la posibilidad de alojarse lejos del correspondiente centro $\mathcal{J}_{ii}$.
\begin{figure}[h!]
	\centering
	\includegraphics[scale=0.2]{../Imagenes/CoeficientesCorrelacionN}
	\caption{Coeficientes de correlación en función de $\sigma$ y $p$. Cada uno de los 78 coeficientes corresponde para un conjunto $\sigma_i$ y $p_j$ de simulaciones según la Tabla (\ref{tab:Simulaciones}).}
	\label{fig:CoeficientesCorrelacionN}
\end{figure}

Para tener una confirmación estadística más robusta sobre la relación entre las $d(\sigma,p)$ y Re$(\lambda)$ se construye un diagrama de correlaciones de cada uno de los 78 conjuntos de simulaciones. El diagrama contiene un coeficiente de correlación entre las distribuciones mencionadas para cada caso. Las líneas verdes delimitan los conjuntos de simulaciones por cada $\sigma$ de acuerdo con la Tabla (\ref{tab:Simulaciones}). En general se puede apreciar una correlación muy alta en entre ambas distribuciones, implicando que la propuesta de las Leyes Circulares funciona en un primer nivel: caracterizando al menos los centros de las vecindades que encierran los valores propios. Lo que faltaría sería definir una regla para cada radio de estas vecindades.
\begin{wrapfigure}{r}{0.5 \textwidth} \vspace{-30pt} \begin{center}
		\includegraphics[scale=0.135]{../Imagenes/AnomaliaSim11}
	\end{center}
	\vspace{-20pt} 
	\caption{Relación entre cantidades de la simulación 11, caso con $\sigma=0.2$ y $p=0.1$.}
	\vspace{-10pt}
	\label{fig:AnomaliaSim11}
\end{wrapfigure}
Este diagrama puede servir para tener un mapa general de como es la relación entre cada una de las distribuciones. Se puede observar un caso excepcional para los valores $\sigma=0.2$ y $p=0.1$, en este caso aunque la correlación sigue siendo muy alta, es la menor de todo el conjunto de simulaciones. Corroborando con la Figura (\ref{fig:AnomaliaSim11}) se puede apreciar un desfase en la linealidad justamente par aquellos valores de mayor magnitud, donde es posible que los radios de sus vecindades sean tan grandes que puedan alojarse lejos del centro de la misma.

\subsubsection*{Breve discusión sobre el radio}

El radio que se propuso al principio de la sección no posee un sustento matemático sólido para poder considerarlo, es únicamente una propuesta heurística para darle forma a la distribución de los valores propios. Una propuesta plausible serían los radios de Gershgorin, dados como la suma de los valores absolutos de los renglones sin considerar al elemento de la diagonal, sin embargo, esta opción contempla solo una cota superior de la distribución de los valores propios, es decir, las vecindades de Gershgorin no controlan la posición de los valores propios: únicamente determinan el área en donde podrían estar posicionados (ver Proposición \ref{prop:DiagonalI} y \ref{prop:paramMay}).\\
\\
Precisamente en la Proposición \ref{prop:paramMay} se ha mostrado que la norma euclidiana define el desplazamiento y confinamiento de los valores propios en el plano complejo. Entonces parece que la filosofía empleada en dicha proposición nos puede dar una respuesta concreta sobre los radios espectrales locales de los valores propios de esta distribución. El límite de esta opción es que actualmente no se posee información sobre la distribución de los valores fuera de la diagonal. La matriz (\ref{eqn:MatrizJacobiana}) indica que hay un producto entre una variable normal ($\alpha_{ij}\sim\mathcal{N}(0,\sigma^2)$) y el elemento del punto fijo $x_i^*\in X^*$ que viene de una distribución de cola pesada (Figura (\ref{fig:PuntoFijoCoop3} \textbf{A})). Incluso obteniendo dicha información es posible que no sea tan útil, pues no tendrá relación directa con lo que verdaderamente define la estabilidad del sistema: las configuraciones de $\Lambda$ (en función de $\sigma$ y $p$) que generan puntos fijos estables. Por lo tanto, la caracterización de estas distribuciones de valores propios queda solamente al nivel de los centros de las vecindades.

\newpage
\section{Transiciones de estabilidad: LV generalizado}

Para darle seguimiento a los resultados de esta tesis, esta sección se va a centrar en explorar de forma cuantitativa las condiciones de estabilidad de los sistemas de Lotka-Volterra generalizado en función de $\sigma$ y $p$. Posteriormente se presentará un análisis sobre las condiciones de la matriz de incidencias para que exista o no estabilidad en los sistemas relacionados, con miras en presentar un primer esbozo del parámetro crítico de transición. Para el caso más robusto ($N=100$) se va a presentar la diferencia entre usar matrices estructuralmente simétricas y puramente aleatorias (como el caso de May), en miras de observar cambios de la dinámica como los que se reportaron al final del capítulo anterior. %A diferencia del caso anterior, este se trata más de como abre el cono convexo considerando que ahora los vectores columna presentan cierta simétría.\\
\\
Se van a considerar tres tipos de sistemas, para $N=\{100,50,25\}$. Se pretende ver como cambia la estabilidad con diferentes tamaños del sistema (\ref{eqn:LK}). El proceso para llevar a cabo la construcción de los diagramas de transición es semejante al que se presentó en la sección de las Transiciones de May, se va a integrar el sistema fijando la fuerza de interacción en los siguientes valores ${\Sigma}=\{0.1,0.2,...,1.0\}$ mientras que para las probabilidades de conectividad se va a considerar la siguiente partición del intervalo $[0,1]$ para cada $N$:
\begin{equation}\label{eqn:particionLin}
	p = \{x_i\, |\, x_i=i\cdot\Delta x, i=0,1,2,...,100\},\qquad\text{con }\Delta x=0.01
\end{equation}
Con esta configuración de valores se permitirá generar diagramas de estabilidad para cada $N$ en función de $p$ considerando que es una partición equidistante. Hay otros dos diagramas de estabilidad que son de interés; siguiendo en función de la probabilidad de conectividad se define la siguiente partición sobre el intervalo $[0,1]$
\begin{equation}\label{eqn:particionLog}
	p_{\log}=\left \{ x_i\, |\, x_i=10^{-\left (10-\frac{10i}{99}\right )},\ i=0,1,...,99\right \}
\end{equation}
Se ha considerado particularmente este conjunto de probabilidades ya que para ciertas $\sigma\in\Sigma$ y para $N=100$ particularmente, la transición ocurre de manera abrupta para valores muy pequeños de $p$ de modo que no se aprecia con claridad cuando ocurre; la partición $p_{\log}$ resuelve este detalle. Por último pero no menos importante, se van a considerar diagramas en donde se deja fija la $p$ para cada $N$ y se define una partición equidistante en el intervalo $[0,1]$ en función de la fuerza de interacción, es decir:
\begin{equation}\label{eqn:particionSigma}
	\sigma = \{x_i\, |\, x_i=i\cdot\Delta x, i=0,1,2,...,100\},\qquad\text{con }\Delta x=0.01
\end{equation}
Para esta configuración de valores se obtendrán los diagramas de transición en función de $\sigma$ y se podrá tener otra óptica de la dinámica enfocándose en la fuerza de las interacciones del sistema, tal y como se observó en los sistemas de May. Este enfoque considerará 10 tamaños diferentes y definidos de los sistemas (para cada $p$ fija) y se observará su cambio de estabilidad cuando dicha fuerza de interacción aumente. Por otro lado, los sistemas en función de $p$ consideran 10 fuerzas diferentes (para cada $\sigma$ fija) y se observará el cambio de la estabilidad cuando se le aumenten enlaces/interacciones al sistema, partiendo de los nodos aislados ($p=0$) hasta la red completamente conexa ($p=1$). \\
\\
Teniendo las particiones y los ajustes de parámetros necesarios para generar los diagramas de transición, solamente falta ver cómo y de qué manera se van a generar estos diagramas: se integrará el sistema cierto número de veces para cada configuración $(N,\sigma,p)$ con sus correspondientes particiones; cada elemento de la partición contempla una configuración particular la cual será integrada el siguiente número de veces con base en $N$:
\begin{table}[h!]
	\centering
	\begin{tabular}{|c|c|}
		\hline
		Tamaño del sistema $N$ & Cantidad de simulaciones por cada elemento de la partición  \\ \hline
		100 & 3000 \\ \hline
		50  & 6000 \\ \hline
		25  & 12000 \\ \hline		
 	\end{tabular}
	\caption{Cantidad de simulaciones realizadas por cada $N$ y para cada elemento de la partición definida.}
	\label{tab:SimulacionesTransicion}
\end{table} 

De esta manera se va a contabilizar el número de simulaciones estables por cada configuración para posteriormente guardar la información y visualizarla en los diagramas de transición. En el diagrama se podrá interpretar un porcentaje de estabilidad por cada configuración efectuada, viendo como cambia con el avance de los valores de la partición. El número de simulaciones efectuadas por configuración, para cada uno de los sistemas contemplados $N=\{25,50,100\}$, responde a la intención de suavizar las curvas siguiendo la ley de los grandes números. Se ha observado que entre más chico sea el tamaño del sistema se requieren mas simulaciones para poder suavizar dicha curva.\\
\\
%En una primera versión de resultados se consideró un método que generó algunos problemas los cuales se visualizaron en la sección anterior, en la discusión de los sistemas asintótica-mente estables: los cuales no alcanzaron su estabilidad o quizás no eran estables y todo gracias al tiempo de integración de $t_f=50$. Sin embargo se han generado otros resultados en donde se pone una restricción para que solamente se consideren sistemas que se integren en un tiempo máximo de $t_f=50$. Entre ambos tipos de resultados existen algunas desviaciones que requerirán de su análisis más adelante. Habiendo conseguido los diagramas de transición mediante todo un proceso computacionalmente arduo, merece preguntarse si acaso estas transiciones  de estabilidad corresponden con transiciones de fase, y de ser cierto ¿cuál sería su parámetro crítico? Se tiene una pista en la relación (\ref{eqn:paramMay}) que re acomodando se tiene
%\begin{equation}\label{eqn:ParametroCritico}
%	\sigma\sqrt{NC}<d
%\end{equation}
%donde en el caso de May corresponde con la diagonal fijada en $-d=-1$. Se explorará si en los diagramas de transición generados existe alguna dependencia/relación con algún valor representativo de las $d_{kk}(\sigma_i,p_j)$ de los Jacobianos del sistema.
Ya que el parámetro de May no se ajusta a la forma de la distribución heterogénea de los valores propios de los sistemas LV, éste no será un parámetro crítico que determine la transición de estabilidad de dichos sistemas. Además, la propuesta sobre las $N$ leyes circulares solamente contempla los centros de las vecindades argumentando que para obtener los radios hay que caracterizar la distribución de los elementos fuera de la diagonal, mismos que se conforman del producto de una variable normal ($\alpha_{ij}$) y la entrada í-esima del punto fijo que para ciertas configuraciones $(N,\sigma,p)$ resulta ser de cola pesada con sesgo positivo hacia el cero. Por lo tanto el parámetro de transición deberá obtenerse de otros medios: principalmente enfocándose en el punto fijo que se deriva de la ecuación $\Lambda X^*=\mathrm{K}$.
\\
\\
En un primer nivel, nos concentraremos en el punto fijo y sus configuraciones $(N,\sigma,p)$, ya que se necesita delimitar los escenarios en donde $X^*\geq 0$ es decir que no posea entradas negativas. Los escenarios en donde existe alguna $x_k\in X^*$ negativa normalmente devienen en dinámicas inestables, incluso si no fuera así: no hay interpretación física para poblaciones negativas. Se dice que este es un primer nivel ya que actualmente no se sabe si existirán puntos fijos $X\geq 0$ que resulten inestables.


\subsection{Para $N=100$}

\subsubsection*{En función de $p$}

Este escenario es el más robusto de los 3 tamaños a analizar, por ende se le pondrá más atención que a los otros dos tamaños. Como bien se mencionó anteriormente, se necesitó una cantidad de simulaciones considerablemente inferior en comparación con los sistemas $N=50,25$ (Tabla (\ref{tab:SimulacionesTransicion})), lo que por sí mismo demuestra que la cantidad de interacciones esta relacionada con la ley de los grandes números, entre más grande sea el sistema: mayor será su capacidad de ``promediar'' el ruido estocástico interno y por lo tanto sus fluctuaciones irán disminuyendo permitiendo que el promedio converja cada vez mejor al valor esperado.
\begin{figure}[h!]
	\centering
	\includegraphics[scale=0.2]{../Imagenes/EstabilidadLKpLin.png}
	\caption{Transiciones de estabilidad en el Sistema de Lotka-Votlerra generalizado estructuralmente simétrico en función de la probabilidad de conectividad para $N=100$.}
	\label{fig:EstabilidadLKpLin}
\end{figure}

En este diagrama al igual que en el May (Fig. \ref{fig:TransicionMayDirLin}) también se observa una transición temprana para los casos $\sigma>0.3$. La diferencia con respecto de los sistemas de May radica en la forma suave de la transición; especialmente para $\sigma=0.2$ no se observa un cambio de fase abrupto delimitado por una probabilidad crítica, sino que la estabilidad va bajando gradualmente conforme el número de enlaces va aumentando. Esta suavidad es semejante a las transiciones de los sistemas semi Allesina del capítulo anterior; más adelante se realizará una comparación entre los escenarios contemplados hasta ahora. En las Figuras (\ref{fig:EstabilidadLKpLin}, \ref{fig:EstabilidadLKpLog}) se muestran los cambios de estabilidad considerando matrices $\Lambda$ estructuralmente simétricas. 
\newpage
Se ajusta el eje $x$ en escala logarítmica para lograr visualizar las transiciones con $\sigma>0.3$. Se puede observar que en casos se sigue presentando una notoria suavidad en la transición con respecto de los sistemas de May (Fig. \ref{fig:TransicionMayDirLog}), y son sustancialmente similares a los sistemas de semi Allesina, por lo tanto la simetría estructural incide significativamente en la forma de las transiciones.
\begin{figure}[h!]
	\centering
	\includegraphics[scale=0.2]{../Imagenes/EstabilidadLKpLog}
	\caption{Transiciones de estabilidad en el Sistema de Lotka-Votlerra generalizado estructuralmente simétrico en función de la probabilidad con escala logarítmica para $N=100$.}
	\label{fig:EstabilidadLKpLog}
\end{figure}

Sin embargo, se ha dado un hallazgo que puede aportar información a la estabilidad de los sistemas estructuralmente simétricos. Exclusivamente se da para interacciones de cooperación en la matriz $\Lambda$, es decir: $(--)$. El hallazgo se da para el caso $N=2$ dando apertura a que se pueda generalizar
\begin{wrapfigure}{l}{0.5 \textwidth} \vspace{-20pt} \begin{center}
		\includegraphics[scale=0.135]{../Imagenes/Heuristico}
	\end{center}
	\vspace{-20pt} 
	\caption{Sistema inestable que no cumple con la relación (\ref{eqn:relacionHeuristica})}
	\vspace{10pt}
	\label{fig:Heuristico}
\end{wrapfigure}
a dimensiones mayores. La relación heurística hallada (considerando $\alpha_{ij},\alpha_{ji}<0$) es la siguiente:
\begin{equation}\label{eqn:relacionHeuristica}
	\alpha_{ij}>\frac{1}{\alpha_{ji}},\quad \text{con }i\neq j
\end{equation}
si esta condición se cumple, entonces el sistema (\ref{eqn:LK}) tiene posibilidades de ser estable (más no garantiza estabilidad para sistemas $N\gg 1$); en cambio si no se cumple esta relación entonces el sistema será inestable tal y como se muestra en la Figura (\ref{fig:Heuristico}): se ha integrado un sistema que no cumple esta condición considerando un intervalo de tiempo que va de 0 a 10, se observa que en apenas 0.6 unidades de tiempo el sistema se disparó a $-\infty$. En la Figura (\ref{fig:Heuristico2}) se presenta otro sistema que cumple al límite con la condición (\ref{eqn:relacionHeuristica}), por lo tanto el sistema si se estabiliza aunque lo hace en valores sumamente elevados, mostrando incluso una \\

\begin{wrapfigure}{r}{0.5 \textwidth} \vspace{-30pt} \begin{center}
		\includegraphics[scale=0.135]{../Imagenes/Heuristico2}
	\end{center}
	\vspace{-20pt} 
	\caption{Sistema estable que cumple al límite con la relación (\ref{eqn:relacionHeuristica})}
	\vspace{-10pt}
	\label{fig:Heuristico2}
\end{wrapfigure}
fase metaestable en la que oscila hasta llegar a su atractor. En los siguientes diagramas se realizarán una serie de comparaciones entre sistemas LV generalizados con $\Lambda$ estructuralmente simétrica y puramente aleatoria, también se realizarán comparaciones con respecto de los sistemas de May antes mencionados. En la primer comparación se busca observar la desviación entre la elección del tipo de $\Lambda$; en la segunda comparación se busca observar la desviación con respecto de sistemas que consideran la diagonal fija. El tipo de $\Lambda$ utilizada para observar la transición de estabilidad influye mucho en como se comporta, en principio se observa en la Figura (\ref{fig:ComparacionIncidenciasp}) un comportamiento semejante al caso de May con respecto de los semi Allesina. El argumento para este caso mencionado eran las posibles correlaciones que emergen por tener simetría estructural, esto orillaba a que los sistemas semi Allesina tuvieran una distribución achatada de valores propios en el eje real que orillaba a sus partes reales a convertirse en positivas primero que en los sistemas de May. En nuestro caso con $\Lambda$, hay que hallar otras posibles explicaciones: ya que el eje de la estabilidad en sistemas LV son los puntos fijos asociados. ¿De que forma la simetría estructural impacta en la definición de los puntos fijos? La desigualdad (\ref{eqn:relacionHeuristica}) nos puede dar pistas al respecto, ya que las desviaciones son más marcadas conforme $\sigma\geq 3$, escenarios donde es más probable que se cumpla dicha desigualdad. 
\begin{figure}[h!]
	\centering
	\includegraphics[scale=0.16]{../Imagenes/ComparacionIncidenciasp}
	\caption{Comparación de la estabilidad entre sistemas de Lotka-Volterra generalizados con $\Lambda$ aleatoria y estructuralmente simétrica.}
	\label{fig:ComparacionIncidenciasp}
\end{figure}
\newpage
Para notar que distribuciones comienzan a no cumplir la desigualdad, se realiza una simulación (Ver apéndice\footnote{Cita}) que utiliza matrices cuadradas similares a $\Lambda$ solo con la diferencia de que todas sus entradas (incluso la diagonal) se mapean de una distribución normal que toma valores discretos de $\sigma$ tal como en la partición (\ref{eqn:particionSigma}). La simulación considera 10,000 matrices de tamaño $N=100$, y por cada $\sigma$ evalúa cuantos de ellos no cumple la desigualdad y devuelve el resultado en forma de porcentaje. 
\begin{tcolorbox}[colback=green!10!white, colframe=black, title=Entrada]
	\begin{minted}{julia}
for sigma in 0.1:0.01:1.0
   if simula_probabilidad_cumplimiento(sigma) != 1.0
      print(sigma," ",simula_probabilidad_cumplimiento(sigma))
      break
  end
end
	\end{minted}
\end{tcolorbox}

\begin{tcolorbox}[colback=red!10!white, colframe=black, title=Salida]
0.27 0.9999998383552714
\end{tcolorbox}
El hallazgo es que a partir de $\sigma=0.27$ se observa la primer distribución normal que no cumple la desigualdad (\ref{eqn:relacionHeuristica}) con menos de 10 casos. Cuando $\sigma$ va aumentando, el número de casos no favorables también lo hace hasta tener 2,585,708 (79\%) de ellos. Y tal cual se nota en la Figura (\ref{fig:ComparacionIncidenciasp}), a partir de $\sigma=0.3$ en adelante se va observando una desviación cada vez más marcada hasta $\sigma=1.0$. Por lo tanto, la relación heurística puede tener un peso significativo en la estabilidad de los sistemas LV. Ahora se mostrarán las diferencias que existen entre ambos sistemas LV con respecto de May y semi Allesina.
\begin{figure}[h!]
\centering
\includegraphics[scale=0.16]{../Imagenes/ComparacionLKMay.png}
\caption{Comparación de la estabilidad entre sistemas de Lotka-Volterra y May en función de $p$ con matrices aleatorias.}
\label{fig:ComparacionLKMay}
\end{figure}

En esta ocasión se percibe que la estabilidad de los sistemas de May y semi Allesina decaen primero que los LV, para ambos casos distinguiendo entre las matrices aleatorias y estructuralmente simétricas. En los diagramas con escala logarítmica se puede apreciar que cada decaimiento de estabilidad en sistemas LV es considerablemente similar su respectivo análogo, al menos de forma cualitativa. Por tanto, cada decaimiento muestra que existe una clara relación entre los sistemas LV y de May y/o semi Allesina. 

\begin{figure}[h!]
	\centering
	\includegraphics[scale=0.16]{../Imagenes/ComparacionLKMayESim.png}
	\caption{Comparación de la estabilidad entre sistemas de Lotka-Volterra y May en función de $p$ con matrices estructuralmente simétricas.}
	\label{fig:ComparacionLKMayESim}
\end{figure}

\subsubsection*{En función de $\sigma$}

Hasta ahora se ha visto como cambia la estabilidad en función de ir agregando enlaces a la red de incidencias considerando 10 diferentes fuerzas de interacción. Ahora el siguiente análisis será contemplar el cambio de estabilidad considerando 10 tamaños diferentes de red variando la fuerza de interacción promedio mediante la partición (\ref{eqn:particionSigma}). En el capítulo anterior se ha observado que las desviaciones más notorias entre sistemas de May y semi Allesina se daban cuando el número de enlaces era escaso y menos notorio conforme dicha cantidad fuera aumentando. El argumento principal se centraba en los tamaños fijos de las redes y que el aumento gradual de la magnitud de sus interacciones volvía las matrices estructuralmente simétricas indistinguibles de las totalmente aleatorias.\\
\\
En nuestro caso con $\Lambda$ puede que ocurra algo semejante, pero se deberá centrar el análisis en los puntos fijos que se generan en cada configuración para desenvolver en el nivel de estabilidad correspondiente. Al igual que en la Figura (\ref{fig:ComparacionIncidenciasp}), se va a mostrar como cambia la estabilidad cuando se consideran matrices de incidencias $\Lambda$ aleatorias y estructuralmente simétricas. Así mismo se observarán las desviaciones con respecto de los escenarios de May y semi Allesina respectivamente, tal como en las Figuras (\ref{fig:ComparacionLKMay}, \ref{fig:ComparacionLKMayESim}).
\begin{figure}[h!]
	\centering
	\includegraphics[scale=0.2]{../Imagenes/EstabilidadLKσLin}
	\caption{Transiciones de estabilidad en el Sistema de Lotka-Volterra generalizado estructuralmente simétrico en función de la fuerza de interacción promedio para $N=100$.}
	\label{fig:EstabilidadLKσLin}
\end{figure}

Se observa que las transiciones en función de $\sigma$ también se muestran más graduales tal y como en los sistemas de May y semi Allesina. El cambio gradual de la magnitud de las interacciones es proporcional al procentaje de estabilidad, ya no es tan abrupto como en el caso anterior en función de $p$. Algo importante de destacar es que conforme el tamaño de la red aumenta: las transiciones se van estrechando cada vez más, de modo que casi no exista diferencia entre el caso $p=0.9$ y $p=1.0$. \\
\\
Otro elemento que se hace evidente es lo que ocurre para $\sigma\geq 0.27$, para el caso $p=0.1$ prácticamente marca la pauta de la estabilidad pero para el resto de las redes, el decaimiento de la estabilidad comienza desde antes de esta cota. Es por ello que la relación heurística (\ref{eqn:relacionHeuristica}) es un ingrediente más a considerar en la estabilidad de los sistemas estructuralmente simétricos más no es enteramente determinante.\\
\\
Al comparar $\Lambda$ estructuralmente simétricas y aleatorias se puede percibir que el decaimiento de las estructuralmente simétricas es primero con respecto de las $\Lambda$ aleatorias, tal y como se observó entre sistemas de May y semi Allesina. En la Figura (\ref{fig:ComparacionIncidenciasσ}) también se muestra evidente que conforme la red es más grande: las desviaciones entre ambas $\Lambda$ va disminuyendo hasta mostrar el mismo decaimiento. En términos de la matriz de incidencias, se puede argumentar lo mismo que en el caso de May y semi Allesina: conforme la magnitud de las interacciones aumente para cada sistema de tamaño fijo, la matriz $\Lambda$ estructuralmente simétrica va siendo indistinguible de su homóloga aleatoria. Por lo tanto, es congruente que las desviaciones vayan siendo mínimas. Será de interés analizar cómo son sus puntos fijos y sus diferencias.
\newpage
\begin{figure}[h!]
	\centering
	\includegraphics[scale=0.22]{../Imagenes/ComparacionIncidenciasσ}
	\caption{Comparación de la estabilidad entre sistemas de Lotka-Volterra generalizados en función de $\sigma$ con $\Lambda$ aleatoria y estructuralmente simétrica.}
	\label{fig:ComparacionIncidenciasσ}
\end{figure}

Al comparar la estabilidad de los sistemas LV con respecto de May y semi Allesina, las diferencias son similares a las Figuras (\ref{fig:ComparacionLKMay}, \ref{fig:ComparacionLKMayESim}): las transiciones de May y semi Allesina se muestran hacia la izquierda con respecto de las transiciones LV. Sin embargo se puede observar que también siguen la tendencia de cambiar su estabilidad entorno a valores que se encuentran alrededor en el mismo orden de magnitud, lo que muestra una relación entre ambos sistemas. En realidad, la única diferencia entre las Jacobianas aleatorias y estructuralmente simétricas con respecto de May y semi Allesina es la injerencia que tiene el punto fijo sobre ellas (\ref{eqn:MatrizJacobiana}).
\begin{figure}[h!]
	\centering
	\includegraphics[scale=0.16]{../Imagenes/ComparacionLKMayσ}
	\caption{Comparación de la estabilidad entre sistemas de Lotka-Volterra y May en función de $\sigma$ con matrices aleatorias y estructuralmente simétricas.}
	\label{fig:ComparacionLKMayσ}
\end{figure}
\newpage
Para concluir el caso $N=100$ se puntualiza a partir de las Figuras (\ref{fig:EstabilidadLKpLin}, \ref{fig:EstabilidadLKσLin}) que genera más impacto en la estabilidad pequeños cambios en $p$ que en $\sigma$, es decir, en términos de estabilidad es más agresivo agregar enlaces a una red de interacciones que aumentar la magnitud de las mismas para redes de tamaño fijo.

\subsection{Para $N=50$}

\subsubsection*{En función de $p$}

Se ha utilizado la misma metodología para generar los diagramas de transición para este caso y para $N=25$. Las particiones y los parámetros del sistema (\ref{eqn:LK}) son los mismos. Lo único que ha cambiado es el número de simulaciones por punto (Tabla (\ref{tab:SimulacionesTransicion})), pues entre menos especies se contemplen se debe de compensar con el número de simulaciones para suavizar las curvas con base en la ley de los grandes números. Algo que se podrá observar en estas transiciones es que son cada vez más suaves y tienen mayor rango de estabilidad (ver caso $\sigma=0.2$). Esto sugiere que es más probable hallar sistemas estables cuando su tamaño es cada vez menor.
\begin{figure}[h!]
	\centering
	\includegraphics[scale=0.16]{../Imagenes/EstabilidadLKp50}
	\caption{Transiciones de estabilidad en el Sistema de Lotka-Volterra generalizado estructuralmente simétrico en función de la probabilidad de conectividad para $N=50$.}
	\label{fig:EstabilidadLKp50}
\end{figure}

Se puede observar que el caso $\sigma=0.2$ tiene hasta un 60\% de probabilidad de ser estable aún cuando la red es completamente conectada, hecho que contrasta con el caso $N=100$. Sin embargo aún así se sigue observando que el cambio de estabilidad para $\sigma$ cercano a 1.0 se da para valores de $p$ del orden de $10^{-4}$. Al realizar la simulación (Apéndice\footnote{Cita}) para observar el cumplimiento de la relación heurística (\ref{eqn:relacionHeuristica}) se encuentra que para $\sigma=0.27$ comienzan a emerger casos que no cumplen dicha desigualdad. Sin embargo solos se toma en cuenta como elemento adicional a considerar para las transiciones de estabilidad.

\subsubsection*{En función de $\sigma$}

En las transiciones que dependen de $\sigma$ también se aprecia que son más suaves y desplazadas un poco a la derecha con respecto del caso $N=100$. Además se sigue observando que las diferencias entre los sistemas $p=0.9$ y $p=1.0$ son muy mínimas en comparación con las desviaciones entre los sistemas $p=0.1$ y $p=0.2$. En este sentido se logra ver que los cambios de estabilidad son más sustanciales cuando se tienen redes escasamente pobladas en comparación con redes densas y casi conexas.
\begin{figure}[h!]
	\centering
	\includegraphics[scale=0.2]{../Imagenes/EstabilidadLKσ50}
	\caption{Transiciones de estabilidad en el Sistema de Lotka-Volterra generalizado estructuralmente simétrico en función de la fuerza de interacción para $N=50$.}
	\label{fig:EstabilidadLKσ50}
\end{figure}

\subsection{Para $N=25$}

\subsubsection*{En función de $p$}

Finalmente para el último caso se encuentran transiciones de estabilidad todavía más suaves, a su vez con un rango de estabilidad mayor con respecto de los sistemas anteriores. Se contempla un número mayor de simulaciones por punto (Tabla \ref{tab:SimulacionesTransicion}) para poder suavizar las curvas. En este caso se puede observar que el sistema con $\sigma=0.2$ es 99\% estable aún con la red totalmente conexa. Se observa que la cantidad de interacciones influye significativamente en el cambio de estabilidad de los sistemas: tienen mayor probabilidad de ser estables. En la proposición \ref{prop:paramMay} se ha mencionado que el radio de Girko (\ref{eqn:radioMayGirko}) tiene un valor esperado $\sigma$\footnote{Considerando matrices de Ginibre, que se encuentran normalizadas por el factor $1/\sqrt{N}$.} con fluctuaciones de orden $1/\sqrt{N}$; eso implica que entre más grande sea el sistema: es más estrecha la incertidumbre de ser estable o no. Por esta razón los sistemas $N=100$ se les observa una transición abrupta. Sin embargo si el sistema es de menor tamaño, entonces las fluctuaciones son más grandes y la probabilidad de ser estables es mayor.

\begin{figure}[h!]
	\centering
	\includegraphics[scale=0.16]{../Imagenes/EstabilidadLKp25}
	\caption{Transiciones de estabilidad en el Sistema de Lotka-Volterra generalizado en función de la probabilidad de conectividad para $N=25$.}
	\label{fig:EstabilidadLKp25}
\end{figure}

En el caso $N=100$ se ha observado que existe una relación sutil entre los sistemas con diagonal fija y las Jacobianas de los LV. No se puede concluir directamente que los sistemas LV tienen fluctuaciones de orden $1/\sqrt{N}$, puesto que no los describe una ley circular por no tener una distribución uniforme de valores propios. Sin embargo se puede suponer que tienen fluctuaciones similares debido a la forma de las transiciones vistas hasta ahora, y para garantizarlo se deberá tener una estimación del parámetro crítico de transición.
\begin{figure}[h!]
	\centering
	\includegraphics[scale=0.2]{../Imagenes/EstabilidadLKσ25}
	\caption{Transiciones de estabilidad en el Sistema de Lotka-Votlerra generalizado en función de la fuerza de interacción para $N=25$.}
	\label{fig:EstabilidadLKσ25}
\end{figure}
\newpage
Las transiciones en función de $\sigma$ también se muestran más suaves con respecto de los sistemas anteriores incluso se logra observar que el sistema con la red totalmente conexa ($p=1.0$) es estable con un 30\% de probabilidad para $\sigma=0.1$, hecho que no se logró observar en los casos anteriores. El mapeo de este caso particular sugiere que se requerirían interacciones de mayor magnitud para poder observar el decaimiento de la estabilidad a cero, es decir, considerar valores $\sigma\geq 1.0$ para el caso $p=0.1$. La simulación del apéndice(\footnote{Cita}) para la relación heurística (\ref{eqn:relacionHeuristica}) indica que para $\sigma=0.29$ se cumple la desigualdad y comienza a no cumplirlos para valores posteriores.
\\
\\
Con esto se concluye la presentación de los resultados más importantes de esta tesis, en este punto se tiene un mapa general de como se comporta la estabilidad de los sistemas LV para diversos configuraciones $(p,\sigma,N)$. Hace falta reconocer si puede existir o no un parámetro crítico de transición además de revisar si la relación (\ref{eqn:relacionHeuristica}) tiene alguna relevancia en dicho parámetro de transición específicamente para los sistemas estructuralmente simétricos. En la siguiente y última sección de este trabajo se estará discutiendo las posibles implicaciones de esto.

\section{Un acercamiento al parámetro crítico}

Existen antecedentes sobre transiciones de fase aplicando el modelo de Lotka-Volterra \cite{bunin2017ecological, biroli2018marginally, altieri2021properties}, específicamente dados en el enfoque de \textit{spin glasses}. Cada uno de los trabajos construye de forma particular el sistema y lo conlleva a resultados interesantes, en esta última sección se explorará brevemente cada enfoque y se abrirá la discusión sobre nuestra transición de fase entre régimen estable de un solo punto de equilibrio al inestable, tal y como se mostró en los diagramas de transición antes presentados.
\\
\\
Para el trabajo aportado por Bunin en 2017 \cite{bunin2017ecological} se consideran las ecuaciones comunes del sistema de Lotka-Volterra (\ref{eqn:LK}) considerando que las interacciones se mapean de forma independiente cumpliendo $\mu\equiv N\langle\alpha_{ij}\rangle$ y $\sigma^2\equiv N\, \Var(\alpha_{ij})$. Además de ello, se define un parámetro $\gamma$ que muestra la correlación entre cada una de las interacciones intervinientes, es decir:
$$\gamma = \text{corr}(\alpha_{ij},\alpha_{ji})\in[-1,1]$$
donde para cada $\gamma\neq 0$ se estará considerando el escenario de Allesina (valores propios distribuidos en una Ley Elíptica), mientras que en $\gamma=0$ se consideran los sistemas dados por May (valores propios distribuidos en una Ley Circular). Bunin propone una reparametrización de las ecuaciones considerando abundancias normalizadas para considerar tamaños relativos de las poblaciones consideradas, por tanto define
$$n_i=\frac{x_i}{\frac{1}{N}\sum_{j=1}^N x_j},\qquad\qquad \frac{1}{S}\sum_i n_i=1$$
Además de definir convenientemente las interacciones de la siguiente manera
$$\alpha_{ij}=\frac{\mu}{S}+\sigma\alpha_{ij}$$
con $\langle a_{ij}\rangle=0$, $\langle a_{ij}^2\rangle=\frac{1}{S}$ y $\langle a_{ij}a_{ji}\rangle=\frac{\gamma}{S}$ y los $a_{ij}$ pueden venir de cualquier distribución estadística que cumpla con dichas media y varianza. Por otro lado se consideran las capacidades de carga $K_i$ provenientes de una distribución normal (tal que $\langle K_i\rangle=1$ y $\sigma_K\equiv\Var(K_i)$) e independientes de las $\alpha_{ij}$, de modo que las $x_i\to x_i/\langle K_i\rangle$. Al explorar las condiciones de equilibrio $\frac{dx_i}{dt}=0$ con base en la reparametrización se obtiene
$$0=n_i\left (\lambda_i - un_i-\sum_{j\neq i}a_{ij}n_j + h\right) $$
considerando las siguientes cantidades: $u=\frac{1-\mu/N}{\sigma}$, $\lambda_i=\frac{K_i-1}{\sigma \langle x_i\rangle}$ y $h=\frac{1}{\langle x_i\rangle}-\frac{\mu}{\sigma}$, además se define $\sigma_\lambda=\Var(\lambda_i)=\frac{\sigma_K^2}{(\sigma\langle x_i\rangle)^2}$. Con esta reparametrización, Bunin establece que el sistema es invariante ante ciertas transformaciones, y toda la información relevante se encuentra contenida en $u$, $\gamma$ y $\sigma_\lambda^2$. Posteriormente se encarga de implementar el \textit{método de cavidad} para hallar la distribución de abundancias y la fracción de especies coexistentes hallando que es una gaussiana truncada
$$n=\max\left (0,\frac{h+\sqrt{q+\sigma_\lambda}z}{u-\gamma v}\right ),\qquad z\sim \mathcal{N}(0,1)$$
donde $\phi$ corresponde con la fracción de especies con $n_i>0$, $q=\langle n_i^2\rangle$ y $v=\langle\partial n_i/\partial \xi_i\rangle$ equivalente a la susceptibilidad. Ello conlleva a determinar la condición crítica de transición entre un único punto de equilibrio a múltiples de ellos, obedeciendo
$$\phi=(u-\gamma v)^2$$
Finalmente concluye que en la fase del único punto de equilibrio, los sistemas ecológicos son predecibles y convergen a dicho punto de equilibrio independientemente de su trayectoria. En la fase de los múltiples puntos fijos, emergen diferentes comunidades con distintas composiciones, dependiendo de condiciones iniciales y el orden de llegada de las especies. La transición se controla gracias a la heterogeneidad de las interacciones y la simetría de la matriz de interacción.\\
\\
Este trabajo continúa y se extiende en 2018 \cite{biroli2018marginally} enfocándose en sistemas de Lotka-Volterra que se encuentran en el borde de la estabilidad, es decir, explorando las implicaciones de un equilibrio marginalmente estable\footnote{Cuando el mínimo valor propio de la matriz Jacobiana es cero}. Las consecuencias de este estado del sistema es que se vuelve extremadamente sensible a perturbaciones, implicando que cualquiera de ellas aleje al sistema de su estabilidad sin que diverga del mismo, sino que más bien lo orbite. Las fluctuaciones pueden ser incluso ambientales (aunque eso recae a interpretación) y amplifican las correlaciones de largo alcance entre especies (en analogía con puntos críticos de transiciones de fase). Esto permite mantener alta diversidad sin perder estructura y refleja un equilibrio entre orden y caos.\\
\\
Más adelante, en 2021 surge una continuación sobre este trabajo pero con un ligero cambio de perspectiva. Altieri \textit{et al.} \cite{altieri2021properties} propone tomar el sistema de Lotka-Volterra generalizado añadiéndole ruido (demográfico) gaussiano\footnote{Correspondiente a nacimientos y decesos que ocurren en las poblaciones.}. El objetivo principal que buscan resolver es ver en que escenarios el tamaño del sistema es susceptible al ruido y como se conecta esto con las fases de tipo \textit{spin glasses}. Para ello definen el sistema de la siguiente manera
\begin{equation}\label{eqn:LKAltieri}
	\frac{dx_i}{dt}=x_i\left( 1-x_i-\sum_{j\neq i}\alpha_{ij}x_j\right )+\eta_i(t)
\end{equation}
Se considera una martiz aleatoria y simétrica con elementos $\alpha_{ij}=\alpha_{ji}$ i.i.d, y $\eta_i(t)$ es el ruido gaussiano con media diferente de cero y con covarianza: $\langle \eta_i(t)\eta'_i(t)\rangle=2TN_i(t)\delta_{ij}\delta(t-t')$ con $T=\frac{1}{N}$; éste término multiplicativo indica que mientras más grande sea la población, menor será el impacto del ruido demográfico sobre el sistema y viceversa; cuando es débil entonces converge a un único punto de equilibrio determinado por los coeficientes de la matriz aleatoria, mientras que cuando el ruido es fuerte las trayectorias generadas no convergen a un punto en especial, sino que se vuelven erráticas y/o atrapadas en zonas metaestables del espacio de estados, lo que conlleva a la fase vítrea.\\
\\
En este contexto la fase vítrea corresponde con una dinámica ecológica lenta, con tiempos de relajación extremadamente largos, atrapada en una jerarquía de valles y subvalles que imposibilita la exploración de todo el espacio posible (ruptura de ergodicidad). Además es altamente dependiente de las condiciones iniciales y del historial de las trayectorias, generando que cualquier perturbación conlleve a otro valle/subvalle; por lo tanto existe una proliferación de estados marginales. Este comportamiento emergerá de la alta heterogeneidad de la matriz aleatoria ($\sigma$) o de la alta intensidad del ruido demográfico reflejado en el tamaño efectivo del sistema.\\
\\
El punto central de las fases consideradas en este trabajo emerge del hecho de poder definir un Hamiltoniano que a su vez define una función de partición. Pensando en ello el sistema sigue una dinámica tipo gradiente descendiente, es decir:
\begin{equation}\label{eqn:Gradiente}
	\frac{dx_i}{dt}=-\frac{\partial H}{\partial x_i}
\end{equation}
Altieri \textit{et al.} establecen que los mínimos locales de la energía del sistema corresponden con los puntos de equilibrio del mismo. Por lo tanto, serán los mínimos de la energía los que describan las fases del sistema de acuerdo con el impacto de $\sigma$ y $T$: si es tal que solo genera un único mínimo, el sistema se organiza en torno al mismo. Al ir aumentando la intensidad de las condiciones mencionadas emergen múltiples puntos de equilibrio y hasta una cantidad exponencial de ellos (fase vítrea) con la cualidad de ser marginalmente estables. \\
\\
Al haber definido un Hamiltoniano asociado al sistema, se puede generar una distribución de estados estacionarios de Boltzmann $P(\{ x_i\})=e^{-\frac{H(\{x_i\})}{T}}$ el cual es posible generar gracias a que el sistema es conservativo\footnote{Cumpliendo
$\frac{\partial}{\partial x_j}\left (\frac{F_i}{x_i}\right )=\frac{\partial}{\partial x_i}\left (\frac{F_j}{x_j}\right )$
con $F_i=-x_i\frac{\partial H}{\partial x_i}$ y al mismo tiempo las $F_i$ son las ecuaciones del sistema (\ref{eqn:LKAltieri}).}. Al dividir por $x_i$ y aplicar la derivada parcial cruzada se llega al siguiente resultado
\begin{equation}\label{eqn:Conservativo}
	\frac{\partial}{\partial x_j}\left (\frac{F_i}{x_i}\right )=-\alpha_{ij},\qquad\frac{\partial}{\partial x_i}\left (\frac{F_j}{x_j}\right )=-\alpha_{ji}
\end{equation}
el cual es verdadero desde que se consideró simétrica a la matriz aleatoria, es decir, $\alpha_{ij}=\alpha_{ji}$ con sus respectivos valores de la distribución estadística considerada. Considerando estos elementos es posible definir una función de partición para explorar el promedio de cantidades observables, pero el espacio de integración se muestra muy complicado además de que se sale de los objetivos del trabajo de investigación. En resumen, los trabajos analizados brevemente hasta ahora tuvieron un enfoque compartido de analizar la dinámica ecológica del sistema (\ref{eqn:LK}) con puntos fijos estables desde una perspectiva estocástica (Altieri \textit{et al.}) y determinista (Bunin \textit{et al.}), y cada uno aportando conclusiones muy valiosas sobre el tema. Lo que seguirá a continuación será la argumentación de esta tesis al considerar el escenario que antecede a los mencionados: la transición del sistema (\ref{eqn:LK}) de puntos fijos estables a inestables.
% Esto entra en contacto con otro concepto que es equivalente pero que ya no voy a incluir por aquí: el análisis de la matriz Hessiana que son las segundas derivadas del Hamiltoniano sobre x_i y x_j, que es en realidad la matriz de interacciones simétrica que consideran en el trabajo. Al ver que sus valores propios son positivos quiere decir que existen mínimos en el Hamiltoniano que llevan a puntos de equilibrio en el sistema; hay que ver cuantos son los mínimos que generan para relacionarlo a la fase que se espera.
\subsection{Discusión}

El gran acierto de Altieri \textit{et al.} en su trabajo antes mencionado, ha sido el hecho de haber restringido las interacciones del sistema (\ref{eqn:LK}) de forma simétrica, es decir, $\alpha_{ij}=\alpha_{ji}$ para toda entrada de su matriz aleatoria\footnote{Que en este trabajo se le ha bautizado como \textit{Matriz de Incidencias}.}. En (\ref{eqn:Conservativo}) se ha observado como las derivadas cruzadas del sistema generan dichos coeficientes que al ser iguales implica que el rotacional del sistema es igual a cero y por lo tanto se puede definir un campo escalar que toma forma de gradiente descendiente (\ref{eqn:Gradiente}). En su trabajo han definido el Hamiltoniano del sistema mismo que induce a una distribución de Boltzmann y a su respectiva función de partición (si es que se requiriera). \\
\\
Lo más importante de estos elementos reunidos (a criterio de este autor) es la forma en que los mínimos del Hamiltoniano representan los puntos fijos (estables) del sistema. En el trabajo se describen distintas fases que no son sino la cantidad de mínimos de energía locales que puede haber en función de la heterogeneidad de las interacciones y el ruido demográfico. En este sentido se puede definir la \textit{Hessiana} del sistema como
\begin{equation}\label{eqn:Hessiana}
	\mathcal{H}_{ij} = \frac{\partial H}{\partial x_i\partial x_j}
\end{equation}
misma que guardará información de la curvatura del Hamiltoniano de tal modo que sus valors propios determinen ``que tan profundos'' son sus mínimos y que tan susceptibles son ante fluctuaciones. Si los valores propios de la matriz Hessiana son positivos, es decir, Re$(\lambda_{\min}(\mathcal{H}))>0$ implica que existe un único mínimo de energía al cual tiende el sistema. Sin embargo, cuando el mínimo valor propio de esta matriz alcanza el cero entonces la curvatura del Hamiltoniano se aplana y emergen modos blandos que representa una estabilidad marginal, de aquí emergen múltiples mínimos de energía y la fase Gardner. Si los valores propios de la Hessiana son negativos, entonces la curvatura del Hamiltoniano pasa a ser cóncava o máxima lo que desenvuelve en un sistema inestable\footnote{Todos aquellos puntos fijos estables o inestables se dan para $\nabla H=0$.}. \\
\\
Esto es crucial ya que en este contexto, la matriz Hessiana es el inverso aditivo de la Jacobiana, es decir, $\mathcal{J}=-\mathcal{H}$\footnote{Solo es posible gracias a que el sistema es conservativo.} indicando que hay una conexión directa entre el soporte espectral de la matriz Jacobiana (que define si las perturbaciones crecen o se mitigan) y la forma de la curvatura del Hamiltoniano (definiendo si el sistema tiene un mínimo o varios mínimos marginales). En ese sentido el mínimo valor propio de la Hessiana equivale al máximo valor propio de la Jacobiana. En el segundo caso, se considera que mientras Re$(\lambda_{\max}(\mathcal{J}))<0$, el sistema será estable; en cambio cuando es igual a cero se ingresa al terreno de la estabilidad marginal y cuando es positivo el sistema se vuelve inestable.
\newpage
Al intentar determinar las derivadas cruzadas del sistema (\ref{eqn:LK}) considerando que la matriz de incidencias $\Lambda$ es únicamente estructuralmente simétrica se encuentra
$$\frac{\partial}{\partial x_j}\left (\frac{F_i}{x_i}\right )=-\frac{r_i\alpha_{ij}}{K_i},\qquad\frac{\partial}{\partial x_i}\left (\frac{F_j}{x_j}\right )=-\frac{r_j\alpha_{ji}}{K_j}$$
considerando que todas las capacidades de carga $K_i$ y todas las tasas de crecimiento $r_i$ son iguales (en esta tesis) aún así $\alpha_{ij}\neq\alpha_{ji}$ y por lo tanto no se puede definir un campo escalar que represente a la energía del sistema: esta configuración es \textit{no conservativa} y en este caso no puede definirse la Hessiana del sistema. Sin embargo, aún la matriz Jacobiana posee el borde espectral que define la estabilidad del sistema, donde Re$(\lambda_{\max}(\mathcal{J}))=0$ es su frontera.
\\
\\
Ya que el sistema de este trabajo es no conservativo, el tipo de transición que se estará por argumentar no sigue la clasificación de Ehrenfest puesto que no se encuentra definida la segunda derivada de la energía libre, únicamente queda el análisis espectral de la matriz Jacobiana del sistema. Esto implica que se estará hablando de una \textit{transición de fase fuera de equilibrio} ya que las cantidades macroscópicas del sistema (principalmente la energía) no se mantienen constantes en el tiempo, sino que existe un flujo o intercambio de energía dentro del mismo. \\
\\
Para esta y toda transición de fase, es necesario definir sus elementos: El parámetro de control será la probabilidad de conectividad $p$ y/o la fuerza de las interacciones dada por $\sigma$. Se puede armar un parámetro de control conjunto como el de May (\ref{eqn:paramMay}) pero para ello se deberá definir el radio espectral de la distribución de valores propios el cual se discutirá más adelante. El parámetro crítico de transición será aquel punto mencionado Re$(\lambda_{\max}(\mathcal{J})=0)$ y las condiciones del parámetro de control para obtener una estabilidad marginal. Por último el parámetro de orden: será el porcentaje de estabilidad del sistema que cambiará cualitativamente a medida que transite hacia el parámetro crítico, manteniendo una fase ordenada (estable) y otra desordenada (inestable).\\
\\
Como se ha mencionado anteriormente, la diferencia entre un sistema conservativo y uno no conservativo es que en el primero las cantidades macroscópicas, especialmente la energía, se estabilizan y dejan de cambiar en el tiempo, a diferencia del no conservativo que puede llegar a estabilizarse a costa de constantes flujos o intercambios de energía que bien podrían venir desde el exterior del sistema. En esta tesis no se ha mencionado nada acerca de los integrantes del sistema, es decir, no se ha especificado de que especies son y su contribución dentro de la cadena trófica, por ello no se hablarán de flujos externos de energía hacia el sistema, pero si se podrá hablar del flujo de energía a causa de la simetría de las interacciones.
\newpage
El trasfondo conceptual de que las interacciones sean simétricas, es decir $\alpha_{ij}=\alpha_{ji}$, es el efecto que ejerce la especie $x_i$ sobre $x_j$ es la misma en sentido opuesto, eso implica la generación de una función de energía como se ha visto antes. Eso significa que no hay ciclos de flujos netos\footnote{La energía fluye en una dirección preferencial sin que exista una compensación exacta en sentido inverso.} que generen disipación de energía y se genera una tendencia hacia el equilibrio termodinámico. En contraste con el caso estructuralmente simétrico, las interacciones no necesariamente son iguales ($\alpha_{ij}\neq\alpha_{ji}$), entonces de aquí si pueden generarse ciclos de flujos netos: En la depredación una de las especies pierde biomasa mientras que otra la gana, en la competencia desigual una de las especies puede tener ventaja sobre la otra, en el mutualismo una de las especies puede beneficiarse más sobre la otra; por lo tanto gracias a este flujo de energía en dirección preferencial el sistema de Lotka-Volterra no conservativo se concibe \textit{fuera del equilibrio termodinámico}.\\
\\
Durante la tesis se ha mencionado que el soporte espectral define estabilidad del sistema; pero además de ello ¡también definirá su transición de fase! Anteriormente se ha revisado que el sistema de May obedece una Ley Circular que esta sustentada por el trabajo espectral de Girko: el centro del disco esta fijado en el valor constante $-d$ que tiene la diagonal de sus matrices aleatorias y el tamaño del radio espectral obedece (\ref{eqn:radioMayGirko}). Sin embargo, las matrices Jacobianas resultantes de resolver el sistema (\ref{eqn:LK}) tienen una diagonal heterogénea dadas por una distribución de cola pesada con sesgo negativo. Aunque se quisiera determinar un ajuste de la Ley Circular de Girko contemplando las $N$ Leyes Circulares propuestas en la Fig. (\ref{fig:LeyesCirculares}), no sería adecuado ya que los parámetros $\sigma$ y $p$ no inciden directamente sobre la Jacobiana sino sobre la matriz de Incidencias $\Lambda$.\\
\\
Anteriormente se ha observado que la distribución espectral de valores propios no sigue una forma simétrica, sino que es amorfa ajustándose a $N$ círculos. Por esta razón es probable que el radio espectral que defina la transición de fase no sea fácil de determinar. En la sección (\ref{subsec:PuntosFijos}) se han introducido las ideas para poder llegar al objetivo. Lo que se plantea es que las interacciones de la matriz de incidencias $\Lambda$: definen la estabilidad del punto fijo. Por lo tanto la misión es hallar las condiciones $(p,\sigma,N)$ tal que la solución del sistema (\ref{eqn:PuntoFijo}) induzca un punto fijo estable. Esta tarea no es trivial de resolver, pues el sistema tiene al menos $N$ soluciones diferentes y no se sabe de su estabilidad hasta que se evalúe en la Jacobiana. La relación heurística comentada anteriormente (\ref{eqn:relacionHeuristica}) puede ser una pista dentro de esta búsqueda pero por ahora se dejará como un problema abierto que llega al límite de esta tesis.
\newpage
\subsubsection*{Interpretación ecológica}

Al considerar matrices estructuralmente simétricas se esta restringiendo al sistema con la condición de que toda especie tena una de tres interacciones posibles ($++$, $--$, $+-$, $-+$). Esto implica que exista una dirección preferencial en el flujo de energía, pues las interacciones no son balanceadas. En este sentido, el sistema se estabilizará cuando el acoplamiento de todas las interacciones resulte en una configuración ordenada y el meollo del asunto es poder caracterizar dicha configuración con base en parámetro de control (acoplamiento de $(p,\sigma, N)$) aplicado a $\Lambda$. En este aspecto, sería interesante poder explorar si el flujo de energía neta (dadas por las interacciones) esta relacionado con la estabilidad del sistema.\\
\\
Hasta ahora se puede interpretar de manera directa las diferentes fases de estabilidad (al menos matemáticamente) pero físicamente ¿qué es lo que representan? Para el caso de la estabilidad es un punto $N$-dimensional al que todas las especies del sistema llegan y perduran, estos puntos son resistentes ante perturbaciones, cualquiera de ellas las puede mitigar con el tiempo hasta volver a alcanzar su punto estable. En el caso del régimen marginalmente estable representa el punto crítico en que las especies comienzan a ser sensibles en presencia de fluctuaciones y no vuelven a su punto estable pero tampoco divergen, más bien presentan un comportamiento oscilatorio cerca del punto fijo. \\
\\
En el caso de la inestabilidad, matemáticamente representa la sensibilidad que tiene el sistema a ser pertubado y diverger. Pero físicamente las poblaciones no crecen exponencialmente; este caso representa que la especie ya no tiene la capacidad de volver a su punto estable. Dependerá de que tan conectada se encuentre esta especie ``inestable'' para trasmitir ese comportamiento al resto del sistema, y eso dependerá de su grado. El modelo de red aleatoria descrita por una distribución de grado binomial implica de alguna manera las especies están relacionadas íntimamente de forma directa o indirecta, por lo que si al menos una especie no logra regresar a su punto estable entonces eso desencadena en la inestabilidad del sistema completo.\\
\\
Una de las observaciones de los diagramas de transición antes presentados, es que es más probable tener configuraciones ordenadas cuando el número de especies $N$ es cada vez menor...
\newpage
\section{Conclusiones}

En esta tesis se ha investigado sobre la estabilidad del sistema de Lotka-Volterra generalizado abarcando desde su construcción hasta su integración y sus implicaciones con el fin de entender los mecanismos que determinan la persistencia o colapso del sistema. Se ha encontrado que la distribución del punto fijo tiene cola pesada con sesgo positivo. Esto indica que existen pocas especies dominantes que prosperan más que el resto de especies del sistema. La distribución del punto fijo induce a la diagonal de las matrices Jacobianas con la diferencia que ahora tiene sesgo negativo. En este sentido, el soporte espectral de las Jacobianas estará controlado por cada una de los valores de su diagonal generando un conjunto de $N$ Leyes Circulares. Finalmente la estabilidad estará dominada por la tupla $(p,\sigma,N)$ que induce puntos fijos estables en la matriz de incidencias $\Lambda$. \\
\\
