\chapter{Introducción}

\setlength{\parindent}{0cm} La física clásica ha sido desarrollada históricamente a partir del estudio de sistemas simples, donde se considera la dinámica de uno o dos cuerpos. El marco de la mecánica newtoniana posee las herramientas necesarias para determinar la evolución temporal de estos sistemas, mismos que serán completamente dependientes de las condiciones iniciales y leyes de movimiento.\\
\\
El planteamiento de sistemas cada vez más complicados de describir orilló a la creación de fomulaciones más generales y estructuradas. La mecánica lagrangiana y hamiltoniana permitieron realizar una síntesis de dicha descripción y extender a sistemas con un mayor número de grados de libertad. Sin embargo, en el límite de un gran número de componentes, la perspectiva conduce al estudio de la mecánica estadística donde el enfoque pasa de trayectorias individuales hacia la descripción de propiedades macroscópicas emergentes.\\
\\
Al considerar sistemas con un número intermedio de grados de libertad, surge un escenario conceptualmente distinto. Las ecuaciones de movimiento que emergen de la perspectiva clásica muestran ser no lineales y no admiten soluciones analíticas cerradas en general. Al mismo tiempo, el número de componentes es insuficiente como para emplear el marco de la descripción estadística basada en promedios macroscópicos.\\
\\
Escenarios ampliamente estudiados sobre este régimen son el problema de los 3 cuerpos y el sistema del péndulo doble. En ambos escenarios se pueden derivar sus ecuaciones de movimiento pero su dinámica exhibe una fuerte sensibilidad en sus condiciones iniciales, lo que limita la predictibilidad a largo plazo. Este comportamiento no señala una falla del marco teórico clásico, sino la aparición de fenómenos con dinámicas no triviales y asociadas al \textit{caos} determinista.\\
\\
Los sistemas que se estudian bajo el marco teórico de \textit{sistemas complejos} no son una clase definida de manera unívoca, sino constituyen un conjunto amplio de modelos y fenómenos que comparten ciertas características estructurales y dinámicas. Resulta más apropiado brindar una definición con base en un conjunto de propiedades comunes que condicionen su comportamiento colectivo.\\
\\
Una propiedad fundamental es la presencia de interacciones no lineales entre las componentes del sistema, implicando que la respuesta global no puede obtenerse como la suma de contribuciones independientes. La no linealidad es consecuencia del acoplamiento de los grados de libertad que dificultan la reducción del sistema a descripciones simples. Así mismo, los sistemas complejos suelen estar formados por una gran cantidad de componentes con interacciones acopladas, dando lugar a una estructura relacional que desempeña un papel central en la dinámica. La representación mediante redes resulta la forma natural para describir los patrones de interacción sin necesariamente tomar en cuenta la naturaleza física de las componentes \cite{posfai2016network,newman2018networks}.\\
\\
Como consecuencia de estas características, los sistemas complejos presentan propiedades macroscópicas emergentes que no pueden inferirse de manera directa a partir de la dinámica microscópica. Estas propiedades implican que su análisis requiera enfoques distintos tanto de la descripción dinámica de pocos grados de libertad como de los métodos estadísticos tradicionales, lo que motiva el desarrollo de herramientas específicas orientadas a capturar el comportamiento colectivo del sistema.
\\
\\
La amplia variedad de sistemas que exhiben las propiedades antes descritas motiva la adopción de un enfoque interdisciplinario que permita transferir marcos conceptuales entre contextos distintos. Fenómenos asociados a interacciones acopladas son recurrentes en sistemas físicos, ecológicos, biológicos y sociales, a pesar de las diferencias sustanciales entre la naturaleza de sus componentes \cite{Koppen2005Interdisciplina}. \\
\\
Esta recurrencia sugiere la existencia de comportamientos colectivos universales que se manifiestan de forma consistente en diferentes escalas, desde lo microscópico hasta lo macroscópico \cite[p. 97--103]{ramirez1999perspectivas}. En consecuencia, la interdisciplinariedad se convierte en una necesidad para la búsqueda de descripciones generales para fenómenos colectivos emergentes. Así el estudio de los sistemas complejos articula distintos campos del conocimiento a partir del interés común en comprender cómo surgen patrones colectivos a partir de interacciones locales.\\
\\
La dinámica de los sistemas complejos tiende a evolucionar alrededor de estados colectivos que actúan como configuraciones preferenciales. A este fenómeno se le conoce como \textit{autoorganización}  y es considerada una cualidad emergente de las interacciones locales entre las componentes del sistema, sin necesidad de la intervención de fuerzas externas \cite{miramontes2005sistemas}. Los estados autoorganizados se manifiestan como soluciones invariantes de las ecuaciones de movimiento; ejemplos de estos estados son los \textit{puntos fijos} y \textit{ciclos límite} del sistema. La relevancia de dichos estados no radica en su existencia sino en su \textit{estabilidad dinámica}, la cual determina si el sistema permanece o no en su vecindad ante la presencia de ligeras perturbaciones del entorno.\\
\\
En consecuencia, la estabilidad dinámica desempeña un papel central en la caracterización del comportamiento colectivo. En particular, es de gran interés cuando se varían los parámetros que controlan las interacciones, ya que pueden inducir a cambios cualitativos en la estabilidad de estos estados, dando lugar a reorganizaciones críticas de la dinámica global asociadas a la emergencia o desaparición de patrones colectivos. Tales cambios se interpretan como \textit{transiciones dinámicas}, en analogía con las transiciones de fase de sistemas termodinámicos \cite{strogatz2001nonlinear}.
\\
\\
La estabilidad dinámica en sistemas complejos puede abordarse de forma sistemática a partir del estudio local de la dinámica en la vecindad de soluciones invariantes. En sistemas gobernados por ecuaciones diferenciales no lineales, el análisis conduce a la \textit{linealización} del sistema alrededor de algún estado estacionario. Esto permite el estudio de su dinámica a través de la aproximación a un sistema lineal efectivo.\\
\\
De este modo, la información esencial sobre la estabilidad dinámica queda codificada en el espectro de valores propios del operador linealizado, usualmente representado por la matriz Jacobiana evaluada en un punto crítico. Dicho espectro determina si pequeñas perturbaciones crecen o decaen en el tiempo y, por tanto, establece criterios precisos de estabilidad local. Al variar los parámetros que controlan las interacciones del sistema, el espectro puede experimentar cambios cualitativos, como el cruce de valores propios a través del eje imaginario, lo que impacta directamente en la estabilidad del estado y señala la ocurrencia de transiciones dinámicas. \\
\\
En este marco general, el modelo de Lotka–Volterra generalizado constituye un ejemplo paradigmático de sistema complejo de alta dimensionalidad con interacciones acopladas y no lineales, ampliamente utilizado en el estudio de dinámicas poblacionales en ecología. La combinación de no linealidad y elevada dimensionalidad presenta una estructura rica de estados estacionarios, cuya estabilidad puede analizarse directamente a partir de la estructura espectral del Jacobiano asociado.\\
\\
En función del marco conceptual elaborado y de la relevancia de la estabilidad dinámica en sistemas no lineales con alta dimensionalidad, el presente trabajo se propone estudiar la estabilidad del modelo de Lotka-Volterra generalizado desde un enfoque espectral, concretamente desde la posición del espectro del Jacobiano en el plano complejo. El análisis se restringe a regímenes en los cuales el sistema es claramente estable o inestable. 
\newpage
El estudio detallado del régimen crítico, en donde el sistema es marginalmente estable, así como la caracterización de modos marginales y fenómenos críticos asociados, queda fuera del alcance de esta tesis. En particular se plantean los siguientes objetivos:
\\
\\
\textbf{\textit{Objetivo principal:}} Investigar las transiciones dinámicas de estabilidad en el modelo de Lotka-Volterra generalizado mediante el análisis espectral de sistemas linealizados entorno a estados estacionarios.

\textbf{\textit{Objetivos específicos:}}
\begin{itemize}
	\item[1.] Caracterizar numéricamente el soporte espectral de un ensamble de matrices Jacobianas asociadas a estados estacionarios del modelo de Lotka-Volterra generalizado.
	\item[2.] Analizar la estabilidad dinámica del sistema a partir del comportamiento de la parte real máxima del espectro, identificando su dependencia con los parámetros de interacción y tamaño del sistema.
	\item[3.] Estudiar como la variación de los parámetros de control induce a cambios continuos en la probabilidad de estabilidad del sistema, interpretados como transiciones dinámicas suaves entre regímenes distintos.
	\item[4.] Proponer un parámetro de orden heurístico basado en una relación señal-ruido para caracterizar la cercanía a una transición de estabilidad.
\end{itemize}

La tesis se organiza de la siguiente manera. En el Capítulo 1 se presentan los antecedentes del problema, contextualizados en el trabajo de Robert May. En el Capítulo 2 se introduce el modelo de Lotka–Volterra generalizado, se establecen las ecuaciones que gobiernan su dinámica y se presenta la linealización del sistema alrededor de estados estacionarios. En el Capítulo 3 se discuten criterios generales de estabilidad inspirados en los trabajos de May y Allesina, los cuales se utilizan como marco de referencia para el análisis espectral. Posteriormente, se describe la metodología numérica empleada y se presentan los resultados sobre transiciones dinámicas de estabilidad, así como la propuesta de un indicador espectral asociado a dichas transiciones, cuya interpretación se discute en contraste con los enfoques clásicos. Finalmente, en el Capítulo 4 se presentan las conclusiones y perspectivas del trabajo.