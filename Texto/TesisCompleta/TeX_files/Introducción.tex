\chapter{Introducción}

\setlength{\parindent}{0cm} Los seres vivos como entes complejos, tienen el privilegio de percibir el mundo a través de sus sentidos; la experiencia consciente y/o inconsciente les conlleva a explorar el mundo de diversas maneras para aprenderlo y vivirlo. Sin embargo, es posible elevar el nivel de entendimiento de la experiencia a través del \textit{lenguaje}. De esta manera la percepción ya no se restringe a la experiencia en primera persona, sino que ahora es capaz de abrirse y trasmitirse con la finalidad de compartir con otros seres lo que nuestros sentidos son capaces de percibir.\\\\
\setlength{\parindent}{0cm} Desde la \textit{filosofía natural} hasta la \textit{ciencia moderna}, la humanidad se ha encargado de sentir y explorar el mundo a través de sus sentidos para plasmarlo en un lenguaje riguroso que sea capaz de explicar dicha experiencia con objetividad. Tal es el caso particular de la física clásica que se centró desde un inicio en explorar y explicar el movimiento de las cosas. Aunque la premisa es muy ambigua, se comenzó por explorar el fenómeno del movimiento a partir de sistemas simples y no necesariamente triviales. \\
\\
La construcción teórica del lenguaje que permite entender estos sistemas es una historia que se aborda rigurosamente en varios semestres de la carrera en Física en la Facultad de Ciencias: contemplando el estudio de la mecánica a partir de la formulación \textit{Newtoniana} que contempla las herramientas para resolver dichos sistemas simples; y posteriormente la formulación \textit{Lagrangiana} y \textit{Hamiltoniana} que representaron los esfuerzos por sintetizar el lenguaje y expresarlo de forma cada vez más elegante. Incluso se llega a abordar el escenario que concibe el límite de una gran cantidad de grados de libertad para conseguir la descripción de propiedades macroscópicas resultantes del sistema. \\
\\
¿Pero qué sucede en medio? cuando se contempla la interacción entre múltiples grados de libertad. En el ámbito de la mecánica clásica se deriva un conjunto de ecuaciones de movimiento no lineales, incapaces de entregar una solución analítica. Y en el ámbito de la formulación estadística, suficientes interacciones para no generar un promedio óptimo que resuelva la conjetura. Ejemplos clave de este contexto son el problema de los 3 cuerpos propuestos por Isaac Newton y el sistema del péndulo doble. \\
\\
Ambos sistemas comparten la característica de tener más de un grado de libertad que interacciona con el resto y entre ellos mismos. Aunque las ecuaciones de movimiento no sean integrables, no significa que las soluciones no existan; tampoco implica que el marco teórico clásico presente una falla técnica, más bien supone el hecho de algo más profundo: muestra la emergencia de comportamientos colectivos no triviales y completamente ajenos a lo que ya se concebía en la física clásica. \\
\\
Por esta razón surge la necesidad de implementar el marco teórico de los \textit{sistemas complejos} donde se analizan comportamientos \textit{no lineales} con cuidadosa rigurosidad. A finales del siglo \textrm{XVIII}, Henri Poincaré presentó lo que en su época se consideró como una solución suficiente al problema de los 3 cuerpos. Su enfoque no se centró en la ``predicción'' de la posición de los astros, sino más bien en la estabilidad del sistema; si los astros permanecen siempre en órbita o si son capaces de salirse y escapar hacia la nada. Poincaré fue la primer persona en vislumbrar la posibilidad del \textit{caos}, en donde un sistema determinista presenta comportamientos aperiódicos y fuertemente dependientes de sus condiciones iniciales, imposibilitando su predicción a largo plazo.\\
\\
Dar una definición de sistema complejos resulta intrínsecamente complejo, puesto que no existe una forma unívoca de representarlos, más bien corresponden con un conjunto amplio de modelos y fenómenos que comparten propiedades estructurales y dinámicas. Las componentes que conforman estos sistemas suelen interactuar entre sí; sin embargo la suma de sus modos independientes es incapaz de describirlos completamente. Este comportamiento es conocido como \textit{no lineal} y es el responsable de que esta clase de sistemas no tengan solución analítica. \\
\\
Otra propiedad que poseen es la \textit{emergencia}, la cual define que los sistemas complejos presentan propiedades macroscópicas no evidentes a nivel microscópico, es decir, no se obtienen por simple superposición de las componentes que lo conforman. Un caso particular de emergencia es la \textit{criticalidad}, el sistema se encuentra en la vecindad de un punto crítico en la cual ocurre una ruptura de simetría para transitar de una fase a otra.\\
\\
