\chapter{Conclusiones}

En esta tesis se ha investigado sobre la estabilidad del sistema de Lotka-Volterra generalizado abarcando desde su construcción hasta su integración y sus implicaciones con el fin de entender los mecanismos que determinan la persistencia o colapso del sistema. Se ha encontrado que la distribución del punto fijo tiene cola pesada con sesgo positivo. Esto indica que existen pocas especies dominantes que prosperan más que el resto de especies del sistema. La distribución del punto fijo induce a la diagonal de las matrices Jacobianas con la diferencia que ahora tiene sesgo negativo. En este sentido, el soporte espectral de las Jacobianas estará controlado por cada una de los valores de su diagonal generando un conjunto de $N$ Leyes Circulares. Finalmente la estabilidad estará dominada por la tupla $(p,\sigma,N)$ que induce puntos fijos estables en la matriz de incidencias $\Lambda$. \\
\\
