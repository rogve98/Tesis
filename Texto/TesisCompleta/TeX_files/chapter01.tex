\chapter{¿Será estable un gran sistema complejo?}

Dentro del marco de los sistemas complejos se manejan varias ramas muy interesantes que le dan su esencia, desde los sistemas dinámicos discretos, dinámica no lineal, teoría de redes complejas, termodinámica fuera de equilibrio, modelos basados en agentes, entre otras. Cada una de ellas aporta un valioso contenido al sistema complejo que se quiera estudiar y analizar dependiendo de sus componentes. Delimitar el área de los sistemas complejos aún resulta una labor complicada debido a su gran \textit{diversidad}, sin embargo, se sabe de la existencia de ciertas características que todo sistema complejo comparte. Los sistemas complejos cuentan con entes: \textit{conectados, interdependientes, dependientes del camino, emergentes} entre otros. El presente trabajo tiene como propósito mostrar al lector cada una de estas características con el objeto de estudio que se va a proponer como piedra angular.\\
\\
Para llegar a conocer nuestra piedra angular primero será necesario delimitar las áreas que intervendrán en la discusión constante de este texto. Se ocupará un \textit{sistema dinámico no-lineal} bajo el soporte de una \textit{red compleja}. La Dinámica no lineal es la rama de los sistemas dinámicos continuos en donde el comportamiento del sistema no se rige por la suma de los comportamientos de sus descriptores. Por ejemplo, una neurona y la suma del comportamiento de las neuronas de un cerebro no puede explicar la emergencia de la consciencia. Por otro lado las redes complejas es la extensión de la \textit{teoría de grafos} aplicada a escenarios comunes de la naturaleza y de la vida cotidiana, tales como redes ecológicas, redes sociales, redes comerciales etc. Su importancia radica en las propiedades que se le pueden extraer para interpretar información sobre la estructura de la red y de la red misma.

\section{Antecedentes}

Por primera vez en 1978, Robert May realizó un trabajo trascendente para los sistemas complejos en el ámbito de la estabilidad [cita\footnote{cita}], si se considera una red ecológica en la que participan $n\gg 1$ especies, la pregunta central es ¿de qué dependerá de que dicho sistema pueda ser estable o no? esta pregunta pretende indagar las condiciones que hacen que el sistema sea estable; en términos matemáticos se refiere a que exista una tendencia hacia un atractor global, donde convergen todas las soluciones. En otras palabras, cómo deben ser los parámetros que gobiernan al sistema y que hacen que las poblaciones que lo constituyen se estabilicen al cabo de cierto tiempo y que además sean resistentes ante fluctuaciones externas. Para definir estas redes ecológicas, primero May define un sistema no lineal de carácter
$$\frac{dX(t)}{dt}=\textbf{F}(X(t))$$
Donde $\textbf{F}:\mathbb{R}^n\to\mathbb{R}^n$ son las ecuaciones que constituyen la dinámica de cada población del sistema y $X(t)=(x_1(t),...,x_n(t))$ son las poblaciones en sí. Usando teoría de perturbaciones, se quiere explorar que sucede alrededor de un punto crítico que es aquel que cumple $F(X^*(t))=0$, para ello se tiene la ecuación
$$X(t)=X ^*+\textbf{p}(t)$$
donde $\textbf{p}(t)$ es en concreto el conjunto de perturbaciones alrededor de $X^*$. Realizando una expansión en series de Taylor se tiene 
$$\frac{d\textbf{p}(t)}{dt}=F(X^*)+\left .\frac{dF}{dX}\right  |_{X^*}\textbf{p}(t)+\mathcal{O}(\textbf{p}^2)$$
Considerando que $\textbf{p}(t)$ son pequeñas perturbaciones, se pueden despreciar los términos no lineales y quedarnos únicamente con la primera derivada, finalmente se tiene 
$$\frac{d\textbf{p}(t)}{dt}=A\textbf{p}(t)$$
donde $A$ es una matriz cuadrada a la que May denomina como \textit{community matrix}\footnote{También conocida como matriz de interacciones.} y sus elementos son tal que $a_{ij}=\left .\frac{\partial F_i}{\partial X_j}\right |_{X^*}$. Esta es una forma de ver al sistema de manera local alrededor del punto de equilibrio, realizar esta aproximación es definir un plano tangente en $X^*$ y así tener la capacidad de poder investigar que sucede aquí, en términos de la dinámica y estabilidad. Para ello se propone una solución del sistema linealizado, el cual puede ser resuelto de manera analítica
$$p_i(t)=\sum_{j=1}^n c_{ij}e^{\lambda_{j}t}\vec{v}_j$$
en este caso los $\lambda_{j}$ son el conjunto de eigenvalores de $A$ y los $\vec{v}_j$ sus eigenvectores asociados. De esta ecuación se puede percibir que el signo de los eigenvalores es sustancial para que el sistema sea estable o no. Si todos ellos tienen parte real negativa, entonces será estable de lo contrario (con al menos uno que no lo sea) será inestable. Esto es crucial porque no será necesario resolver el sistema de ecuaciones diferenciales para determinar si es estable o no, basta con conocer el signo de la parte real de sus eigenvalores y con ello determinar si el sistema es estable o no. Por esta razón May ha dejado de lado el proceso de linerización del sistema y se enfoca exclusivamente en las cualidades de la matriz de interacciones $A$.
\\
\\
\setlength{\parindent}{0cm} May tiene una forma de definir estas matrices de interacción, primeramente establece que la diagonal de $A$ tiene que estar fijado a un valor $-d$ ya que considera que estos valores fungen el papel de auto-regulación de cada una de las especies; el sistema sea estable o no, es importante que considere esta característica. Posteriormente, el resto de las entradas de la matriz $a_{ij}\in A$ con $i\neq j$ están sampleadas a partir de una distribución estadística, centrada en $\mu=0$ y con desviación estándar $\sigma$, de manera directa se decide utilizar una función de densidad de probabilidad normal. La forma de samplear estos valores se va a dar mediante un parámetro $C $ llamado \textit{conectancia}, y la regla para definirlos es que para cada entrada de la matriz con $i\neq j$, si se cumple $r<C$ entonces $a_{ij}=x$, donde $x$ es un valor aleatorio de la FDP normal. En caso contrario ($r\geq C$) el valor será cero $ a_{ij}=0$.\\
\\
Al parámetro $\sigma$ también se le conoce como fuerza de interacción promedio, y no es más que el peso de los enlaces que se relacionan entre especies. Estos parámetros, en conjunto con el número de entes interactuantes $n$: define la complejidad del sistema, y en concreto las condiciones necesarias que se deben de cumplir para que el sistema sea estable. Estas condiciones están bien estudiadas y definidas, y las conclusiones se pueden consultar en [cita\footnote{cita}], además en los siguientes capítulos se hablarán de ello con cierta precisión. 
\\
\\
Definir un sistema estable significa hallar todos los eigenvalores del sistema linearizado con parte real negativa; deja de ser estable cuando existe al menos un eigenvalor con parte real positiva. Al visualizar la distribución de eigenvalores en el plano complejo, May establece que ésta se ajusta adecuadamente a un círculo con centro y radio $-d$, tal y como esta definida la diagonal de la matriz de interacciones $A$. Si el sistema es estable: la distribución de eigenvalores se ajusta a esta \textit{Ley Circular}, de lo contrario los eigenvalores se salen de este confinamiento haciendo énfasis en la parte real de los mismos.
\\
\\
En resumen, la estabilidad de los sistemas dados por las matrices de interacción de May dependen exclusivamente de los parámetros que las construyen ($\sigma$, $C$ y $N$). Por lo tanto es posible que exista un parámetro crítico que relaciona los anteriores en uno solo, capaz de definir un régimen de estabilidad y otro de inestabilidad siendo éste el punto de inflexión entre ambos regímenes. El parámetro en concreto se encuentra definido por la siguiente desigualdad $\sigma<(NC)^{-1/2}$ y del que se puede obtener $\sigma\sqrt{NC}<1$. \\
\\
Este parámetro sugiere que existirán diversas condiciones para que un sistema sea estable y dependerá de la relación de los 3 parámetros. Existe una relación estrecha entre $\sigma$ y $C$, sin embargo, el parámetro $N$ define complejidad del sistema en términos del tamaño del mismo, entre más grande sea más influyente es la relación entre $\sigma$ y $C$. La desigualdad lleva a concluir que la estabilidad en los sistemas depende de que $\sigma$ sea inversamente proporcional a $C$ y $N$; un sistema será estable si existen pocas conexiones en la red y que las que existen tengan una fuerte interacción. En caso contrario, una red bien conectada podrá ser estable únicamente con interacciones débiles. La interacción débil o fuerte en este contexto es ambigua y ciertamente subjetiva: las interacciones débiles y fuertes se considerarán en los valores promedio $\sigma\approx 0$ y $\sigma\approx 1$ respectivamente de la FDP que samplea las matrices de interacción de May. Este concepto adquiere mayor significado cuando estos valores se comprometen con la desigualdad antes mencionada.

\section{Planteamiento del problema}

Como bien podría pensarse, el trabajo de May esta publicado con todo y sus conclusiones ¿Qué se podría extraer de aquí? La propuesta de esta tesis es poder mostrar al lector una alternativa de construcción de estos sistemas desde el principio, considerando una serie de pasos que May asume \textit{a priori}. Primeramente se piensa trabajar con el sistema de Lotka-Volterra generalizado (\ref{eqn:LK}), mismo que describe dinámica de las especies en términos de su población. Este es un sistema de ecuaciones diferenciales no lineales y se piensa construir para $N\gg 1$, por lo que se contempla utilizar integración numérica mediante algoritmos computacionales para lograr este cometido. \\
\\
Para ello debe de modelarse con una serie de interacciones que se consideran en el sistema y que viene dado por los coeficientes $\alpha_{ij}$ que multiplican a las especies interactuantes $x_j$ con la especie $x_i$. Estos coeficientes formarán parte de una matriz que más adelante se definirá denominada como \textit{matriz de incidencias}, esta matriz únicamente guarda la información de como son las interacciones del sistema (\ref{eqn:LK}). La ventaja de esta matriz es que en su construcción se pueden considerar diferentes topologías de red, en esta tesis exclusivamente se ha empleado el modelo de red aleatoria de Erdös–Rényi. Al emplear este modelo, la matriz de incidencias guarda en sí un parámetro $p$ que es la probabilidad de conexión entre nodos. Además se añaden pesos a los enlaces para representar las interacciones del sistema; cada uno de estos enlaces también estarán sampleados a partir de una FDP normal.\\
\\
De esta forma se construye apenas el modelo del sistema, el cual será integrado numéricamente con el algoritmo Runge-Kutta de orden 4. Realizar esto nos dará oportunidad de conocer las series de tiempo resultantes y visualizar la dinámica del sistema además de indicarnos si son estables o no. Sin embargo, lo que nos interesa es conocer el atractor al que tienden los sistemas que resultaron estables (bajo los parámetros antes mencionados) para poder determinar el Jacobiano del sistema y evaluarlo en este punto crítico ($X^*$). \\
\\
Este proceso es todo aquello que May ha asumido \textit{a priori}, al construir el Jacobiano mediante la serie de pasos antes mencionados, y al evaluarlo en $X^*$ se podrá llegar a la matriz de interacciones del sistema que nos indicará como es la dinámica del sistema de manera local alrededor del atractor. Con el resultado de esta matriz interacciones se pueden hacer varias cosas: se investigará como es la distribución de eigenvalores en el plano complejo y si coincide con los resultados de May. Sin embargo el principal objetivo de esta tesis es explorar la cualidad de los parámetros que forman la matriz de incidencias para determinar que el sistema sea estable o no. Dicho de otra forma es hallar las condiciones de estabilidad en función de $p$, $\sigma$ y $N$  para las matrices de interacción provinientes de (\ref{eqn:LK}).\\
\\
Para lograr este cometido se va a repetir todo el proceso para diferentes valores de $\sigma$ y $p$ concentrados en el intervalo $[0,1]$ con un paso de 0.1 entre cada valor del intervalo. Y estos valores para los siguientes números de especies $N=25,\,50,\, 100$. Al lograr esta misión, se busca comparar estos resultados con lo que ha estipulado May y si se relacionan de alguna forma. En esta dirección, explorar si el parámetro crítico y la forma de las transiciones entre el régimen estable e inestable se siguen respetando o presenta variaciones.\\
\\
Con estos elementos se presenta la siguiente hipótesis: Los sistemas de Lotka-Volterra generalizados bajo la topología del modelo de red aleatoria, presentan una transición de fase entre un régimen estable y otro inestable, así mismo, ésta depende de los parámetros que dan origen a la matriz de incidencias ($p$, $\sigma$ y $N$).
