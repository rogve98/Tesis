\chapter{Estabilidad en sistemas complejos de gran dimensión.}
\setlength{\parindent}{0cm} El análisis de la estabilidad dinámica constituye un problema central en el estudio de sistemas complejos de gran dimensión. Cuando el número de grados de libertad es elevado, la dinámica local alrededor de los estados estacionarios depende de la estructura global de las interacciones, lo que dificulta la aplicación directa de criterios clásicos de estabilidad. En este contexto, pequeñas variaciones pueden alterar drásticamente su comportamiento dinámico induciendo a posibles transiciones cualitativas del sistema. Este capítulo se enfoca en el análisis de sistemas modelados mediante matrices de interacción, sentando las bases para los resultados que se discutirán posteriormente.

\section{Antecedentes}\label{sec:Antecedentes}

El estudio de la estabilidad de sistemas dinámicos con un gran número de grados de libertad plantea dificultades que no están presentes en sistemas de baja dimensión. En este contexto, Robert May en la década de 1970 introdujo un enfoque estadístico para analizar la estabilidad de sistemas complejos, marcando un punto de referencia en el área \cite{may1972will}. Considera el uso de redes ecológicas con $N\gg 1$ especies para investigar la posibilidad de que el sistema sea estable o no. La clave que brinda dicha posibilidad se encuentra en la estructura de las interacciones, y es preciso delimitar su condición para distinguir entre un régimen resistente y susceptible a perturbaciones. Para abordar el problema, May introduce un sistema dinámico que considera \textit{a priori} interacciones acopladas
$$\frac{dX(t)}{dt}=\textbf{F}(X(t))$$
Donde $\textbf{F}:\mathbb{R}^n\to\mathbb{R}^n$ contiene funciones no lineales que constituyen la dinámica de cada población del sistema y $X(t)=(x_1(t),...,x_n(t))$ son las poblaciones en sí. Haciendo uso de teoría de perturbaciones \cite[p. 19]{may2019stability} se explora la dinámica alrededor de un punto estacionario $X^*(t)$, es decir, que cumple $\textbf{F}(X^*(t))=\vec{0}$, entonces se considera la ecuación
$$X(t)=X ^*+\textbf{p}(t)$$
donde $\textbf{p}(t)$ es en concreto el conjunto de perturbaciones alrededor de $X^*$. Realizando una expansión en series de Taylor se tiene 
$$\frac{d}{dt}\left (X^*(t)+\textbf{p}(t)\right )=\textbf{F}(X^*)+\left .\frac{d\textbf{F}}{dX}\right  |_{X^*}\textbf{p}(t)+\mathcal{O}(\textbf{p}^2)$$
Considerando que $\textbf{p}(t)$ son pequeñas perturbaciones, se pueden despreciar los términos no lineales y quedarnos únicamente con la primera derivada. Al reducir esta ecuación, finalmente se tiene 
$$\frac{d\textbf{p}(t)}{dt}=A\textbf{p}(t)$$
donde $A$ es una matriz cuadrada a la que May denomina como \textit{community matrix}\footnote{También conocida como matriz de interacciones. Es considerada una red ecológica en donde cada renglón corresponde con una especie y las columnas representa su respectiva interacción con otras especies.} y sus elementos son tal que $a_{ij}=\left .\frac{\partial f_i(X)}{\partial X_j}\right |_{X^*}$ con $f_i\in\textbf{F}$. De esta manera, se podrá analizar la estabilidad del sistema de forma local alrededor de algún estado estacionario. Sin embargo, el punto de equilibrio en consideración es arbitrario y su determinación no forma parte del análisis, únicamente se asume la matriz de interacciones $A$ con ciertas características. Este ejercicio permite explorar sistemas no lineales de forma local en un sistema que si es lineal. Por tanto, se pueden determinar sus valores propios e incluso su solución general que tiene la siguiente forma
$$\textbf{p}(t)=\sum_{j=1}^N c_{j}e^{\lambda_{j}t}\vec{v}_j$$
En este caso los $\lambda_{j}$ son el conjunto de valores propios de $A$ y los $\vec{v}_j$ sus vectores propios asociados. De esta ecuación se puede percibir que el signo de los valores propios es sustancial para que el sistema sea estable o no. Si los valores propios tienen parte real negativa implica que el sistema no saldrá de su vecindad local ante perturbaciones suficientemente pequeñas, en caso contrario las posibles perturbaciones se irán amplificando hasta que el sistema sea incapaz de regresar a su vecindad local cuando $t\to\infty$. \\
\\
Al centrarse en la matriz de interacciones, May modeló el sistema mediante matrices aleatorias con entradas independientes e idénticamente distribuidas de media cero y varianza finita, lo que permite caracterizar el espectro del sistema linealizado. En el límite de gran dimensión, los valores propios se distribuyen de acuerdo con la \textit{Ley Circular}, dando lugar a un umbral crítico que separa regímenes estables e inestables \cite{may1972will}. No obstante, este enfoque asume la existencia de un estado estacionario que no surge como solución de ecuaciones no lineales específicas, sino como una condición abstracta del equilibrio. 

\section{Planteamiento del problema}

Las matrices de interacción de May se construyen con base en varias características: primero se establece que los valores de la diagonal se mantienen fijos en un valor real $-d$, estos valores funcionan como auto reguladores de cada una de las especies y es muy importante que sea un real negativo ya que se verá directamente reflejado en el soporte espectral. El resto de las entradas provienen de una distribución con media cero y varianza finita, y la forma del muestreo será con base en una probabilidad de conectividad $C$, considerando que el elemento $a_{ij}$ no necesariamente es igual a su transpuesto $a_{ji}$.\\
\\
El parámetro $\sigma$ de la distribución de las interacciones cuantifica la intensidad típica de los acoplamientos entre las componentes del sistema y puede interpretarse como la escala de los pesos de los enlaces en una red de interacciones. Junto con el número de entes interactuantes $N$ y la conectividad $C$, estos parámetros determinan la complejidad del sistema desde el punto de vista de su estabilidad dinámica. En particular, existen combinaciones de $C$ y $\sigma$ para las cuales el estado estacionario es estable, así como regímenes en los que pequeñas perturbaciones se amplifican. La variación de estos parámetros de control permite analizar las transiciones dinámicas del sistema, condiciones que se encuentran en detalle en el marco del modelo de May \cite{may1972will}.
\\
\\
En sistemas de alta dimensión, el soporte espectral del operador linealizado se distribuye aproximadamente sobre un disco centrado en el término disipativo de la diagonal $-d$. El radio espectral depende de los parámetros de control del sistema y determina si el espectro cruza el eje imaginario, es decir, si aparecen valores propios con parte real positiva, lo cual es clave para la estabilidad del sistema. En este sentido, May introduce un parámetro crítico que delimita el umbral entre regímenes estables e inestables
\begin{equation}\label{eqn:paramMay}
	\sigma<(NC)^{-1/2}
\end{equation}
Al considerar el caso $d=1$, la condición de estabilidad se reduce a $\sigma\sqrt{NC}$. Para un sistema de tamaño fijo $N$, esta desigualdad define un umbral que puede analizarse variando la conectividad $C$ a $\sigma$ fija, o bien variando $\sigma$ a conectividad fija. Sin embargo, al aumentar el tamaño del sistema, la condición de estabilidad se vuelve más restrictiva. En el régimen $N\gg 1$, mantener la estabilidad requiere que la conectividad escale como $C\propto N^{-1}$ si $\sigma$ se mantiene fija, o bien que la intensidad de las interacciones escale como $\sigma\propto N^{-1/2}$ para conectividad fija. Cabe notar que, debido a los distintos escalamientos en $N$, las transiciones asociadas a variaciones en $C$ tienden a ser más abruptas que aquellas controladas por la intensidad de las interacciones, un aspecto que se retomará más adelante.
\newpage
\begin{figure}[h!]
	\centering
	\includegraphics[scale=0.16]{../Imagenes/TransicionesdeMay}
	\caption{Transiciones de estabilidad con sistemas de May de tamaño $N=100$. (\textbf{A}) Para la conectancia variable fijando $\sigma=0.3$. (\textbf{B}) Para la intensidad de las interacciones variable fijando $C=0.3$.}
	\label{fig:TransicionesdeMay}
\end{figure}

En términos cualitativos, el resultado central de May establece que la estabilidad de un sistema complejo emerge de un compromiso entre la conectividad de la red y la intensidad de las interacciones: sistemas con pocas conexiones pueden sostener interacciones fuertes sin perder estabilidad, mientras que redes densamente conectadas solo permanecen estables cuando las interacciones son suficientemente débiles. \\
\\
Sin embargo, esta caracterización proviene de la construcción estadística del sistema linealizado, sin incorporar explícitamente la dinámica no lineal que conduce a los estados estacionarios sobre los cuales se evalúa la estabilidad. En contraste, esta tesis aborda esta limitación considerando explícitamente la dinámica no lineal de un sistema de Lotka–Volterra generalizado \cite[p. 86]{may2007theoretical}, a partir del cual se identifica un estado estacionario bien definido y se analiza la estabilidad del sistema linealizado asociado.\\
\\
El sistema de Lotka-Volterra generalizado considera un gran número de especies interactuantes $N\gg 1$. Este marco permite formular una dinámica no lineal bien definida y analizar como la estructura de las interacciones influye en la estabilidad de los estados estacionarios resultantes. Debido a la alta dimensionalidad del sistema, el estudio de su comportamiento dinámico y de las transiciones de estabilidad se apoya en herramientas computacionales.\\
\\
En este contexto, se propone modelar la estructura de las interacciones a partir de redes aleatorias tipo Erdös–Rényi, con posibilidad de extender el análisis a otras topologías. Al incorporar pesos estadísticos a los enlaces, se obtiene una representación matricial de las interacciones caracterizada por una intensidad típica $\sigma$, una probabilidad de conectividad $p$ y el tamaño del sistema $N$. Este enfoque permite estudiar como la estructura estadística de las interacciones influye en la estabilidad de los estados estacionarios del sistema dinámico y en la aparición de transiciones entre regímenes estables e inestables.\\
\\	
Finalmente, es importante señalar el alcance de este trabajo. El análisis desarrollado no pretende determinar un parámetro crítico en el sentido estricto del marco de May o de la mecánica estadística, sino proponer un criterio heurístico para identificar transiciones de estabilidad en sistemas de alta dimensión. Dicho criterio se construye a partir de una cantidad tipo \textit{signal-to-noise ratio} (SNR), que permite caracterizar de manera efectiva la competencia entre la estructura promedio de las interacciones y sus fluctuaciones. En este sentido, los resultados presentados deben interpretarse como un primer acercamiento al estudio del parámetro de transición cuando la dinámica no lineal del sistema es considerada de forma explícita.


