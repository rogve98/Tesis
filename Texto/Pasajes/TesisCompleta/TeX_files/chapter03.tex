\chapter{Transiciones de fase: Estabilidad}

Para analizar la dinámica del sistema de competencia de especies se van a considerar redes aleatorias con pesos, lo que en el capítulo pasado llamamos como \textit{matriz de incidencias} de 25, 50 y 100 nodos respectivamente. Cada una de las entradas de la matriz de incidencias representa un enlace en la matriz aleatoria y de acuerdo con su signo puede ser de ``cooperación" o de ``competencia". Por el hecho de tomar en cuenta distribuciones normales para los pesos de cada entrada de la matriz de adyacencia asociada a la red aleatoria, es posible obtener un $50\%$ de probabilidad de que sea positivo y $50\%$ de que sea negativo, al tener varios coeficientes de cooperación nos orilla a que existan especies que se beneficien tanto de esto que puede que rebasen la capacidad de carga del sistema, y no será por un tema de recursos en el sistema sino por el tema de la misma cooperación que hace que las especies puedan encontrar estabilidad en un punto mayor.
\\
\\
Esto se verá una vez que veamos como son las serie de tiempo de los sistemas. Por lo pronto hablemos de los parámetros empleados para la resolución. Con el fin de tener el mayor número de parámetros controlados, se han definido constantes para cada uno de los cálculos realizados, en concreto se definió una tasa de crecimiento $r=2$, una capacidad de carga de $K=5$ para cada especie, se ha integrado para un intervalo discreto de tiempo de $t\in[0,50]$ con un paso de integración $h=0.01$, por último se definió. Por último, los únicos parámetros que si podían variar para realizar el análisis exploratorio son $N$ en número de especies, $\sigma$ la desviación estándar y $p$ la probabilidad.\\
\\
A grandes rasgos lo que representa cada uno de estos 3 parámetros variables es lo siguiente: $N$ simplemente define el tamaño del sistema, entre más grande sea el sistema es cada vez menos probable que el sistema sea estable por la cantidad de interacciones que se encuentran disponibles; por otro lado la $p$ nos define la cantidad de enlace en la red que esta representada con las interacciones entre especies, por su parte entre más grande sea la $p$ es menos probable que sea estable el sistema; por último la $\sigma$ representa la fuerza de las interacciones entre especies, que tan predominante o cooperativo (dependiendo del signo) es la especie $j$ con la especie $i$, entre más fuertes sean las interacciones es menos probable que el sistema sea estable. Por lo tanto la estabilidad esta dada en función de la magnitud de estos 3 parámetros y su relación entre ellos tal y como lo establece May en su trabajo [\footnote{agregar cita}].\\
\\
\begin{figure}[h!]
	\centering
	\includegraphics[scale=0.23]{../../Imagenes/Series de Tiempo LK100}
	\caption{(\textbf{A}) Series de tiempo para el sistema de competencia de especies para una red de incidencias de 100 nodos, con $\sigma=0.2$ y $p=0.35$. (\textbf{B}) Series de tiempo para el sistema de competencia de especies con una red de incidencias de 100 nodos con $\sigma=0.2$ y $p=0.5$}
	\label{fig:SeriesdeTiempoLK100}
\end{figure}
